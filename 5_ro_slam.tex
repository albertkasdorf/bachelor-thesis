\begin{comment}
------------------------------------------------------------------------------------------
- \cite{kurth2003experimental}
	- Numerically, we can evaluate the performance of the dead reckoning and Kalman lter localization methods by considering the cross-track error (XTE). That is, for each pose we measure how far left or right of the true position our estimation is, orthogonal to the true heading. We compile these errors for every point along the path, then nd the maximum value along with the mean and standard deviation of the errors to produce the evaluative statistics in Table 1.
	- https://de.wikipedia.org/wiki/Querabweichung
- Diagramme
	- \cite{kurth2003experimental}
		- Fig. 5: (1) The ground truth path with tags indicated by circles. The numbers indicate how many range measurements were received from each tag over the duration of Test 1. (2) The path estimate from dead reckoning alone. (3) The path estimate from localization using a Kalman lter. The lter fuses data from odometry and a gyro with absolute measurements from RF tags to produce this path estimate. Numerical results are given in Table 1. (X: position in x(m), Y: position in y(m), Ground truth path with tag locations, Dead reckoning path, Kalman filter localization path)
\end{comment}
\chapter{RO-SLAM}\label{ch:ro_slam}


\begin{comment}
------------------------------------------------------------------------------------------
\end{comment}
\section{Roboterplattform [todo]}


\begin{comment}
------------------------------------------------------------------------------------------
\end{comment}
\section{Softwarearchitektur [todo]}


\begin{comment}
------------------------------------------------------------------------------------------
\end{comment}
\subsection{ROS Module [todo]}


\begin{comment}
------------------------------------------------------------------------------------------
\end{comment}
\subsection{MRPT Module [todo]}