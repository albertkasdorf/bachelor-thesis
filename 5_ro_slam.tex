\begin{comment}
------------------------------------------------------------------------------------------
- \cite{kurth2003experimental}
	- Numerically, we can evaluate the performance of the dead reckoning and Kalman lter localization methods by considering the cross-track error (XTE). That is, for each pose we measure how far left or right of the true position our estimation is, orthogonal to the true heading. We compile these errors for every point along the path, then nd the maximum value along with the mean and standard deviation of the errors to produce the evaluative statistics in Table 1.
	- https://de.wikipedia.org/wiki/Querabweichung
- Diagramme
	- \cite{kurth2003experimental}
		- Fig. 5: (1) The ground truth path with tags indicated by circles. The numbers indicate how many range measurements were received from each tag over the duration of Test 1. (2) The path estimate from dead reckoning alone. (3) The path estimate from localization using a Kalman lter. The lter fuses data from odometry and a gyro with absolute measurements from RF tags to produce this path estimate. Numerical results are given in Table 1. (X: position in x(m), Y: position in y(m), Ground truth path with tag locations, Dead reckoning path, Kalman filter localization path)
\end{comment}
\chapter{RO-SLAM}\label{ch:ro_slam}


\begin{comment}
------------------------------------------------------------------------------------------
\end{comment}
\section{Roboterplattform [todo]}


\begin{comment}
------------------------------------------------------------------------------------------
\end{comment}
\section{Softwarearchitektur [todo]}


\begin{comment}
------------------------------------------------------------------------------------------
\end{comment}
\subsection{ROS Module [todo]}


\begin{comment}
------------------------------------------------------------------------------------------
\end{comment}
\subsubsection{hector\_trajectory\_server}

Mit diesem Modul ist es möglich die gefahrene Trajektorie eine Roboters in einem bestimmten Koordinatensystem auszugeben. Hierfür muss im TF--Tree eine Verbindung zwischen dem \textit{source\_frame\_name}(base\_link) und dem \textit{target\_frame\_name}(odom oder map) bestehen. Die Trajektorie bezieht sich mit ihren Koordinaten auf das target\_frame und ist vom Datentyp nav\_msgs/Path.

Über einen Service lässt die sich Trajektorie auch abfragen: \textit{rosservice call /trajectory}

\url{http://wiki.ros.org/hector_trajectory_server}

\url{http://docs.ros.org/api/nav_msgs/html/msg/Path.html}

\begin{lstlisting}[
	frame=shadowbox,
	breaklines=true,
	caption={Konfiguration der hector\_trajectory\_server--Nodes.},
	captionpos=b,
	label={lst:hector_trajectory_server_node},
	columns=fullflexible,
	language=XML,r
	numbers=none,
	float,
]
<node 
  name="hector_trajectory_server"
  pkg="hector_trajectory_server"
  type="hector_trajectory_server"
  output="screen">

  <param name="target_frame_name" value="map"/>
  <param name="source_frame_name" value="base_link" />
  <param name="trajectory_update_rate" value="10.0" />
  <param name="trajectory_publish_rate" value="10"/>
</node>
\end{lstlisting}


\begin{comment}
------------------------------------------------------------------------------------------
\end{comment}
\subsubsection{rf2o\_laser\_odometry}

\url{http://wiki.ros.org/rf2o}


\begin{comment}
------------------------------------------------------------------------------------------
\end{comment}
\subsubsection{robotino\_node}


\begin{comment}
------------------------------------------------------------------------------------------
\end{comment}
\subsubsection{robotino\_odometry\_node}

Sorgt dafür, dass die Odometry--Nachrichten vom Robotino ins ROS--System veröffentlicht werden.


\begin{comment}
------------------------------------------------------------------------------------------
\end{comment}
\subsubsection{Vergleich der Trajektorie von Odom und rf2o}

- Trajektorie der Robotino--Odometry bestimmen.
- Herausfiltern der Odometry Nachrichten aus den Bag--Dateien
- Odometry aus den Laser--Scans bestimmen und Trajektorie aufzeichen.
- Trajektorien vergleichen




\begin{comment}
------------------------------------------------------------------------------------------
\end{comment}
\subsection{MRPT Module [todo]}