\chapter{Zusammenfassung und Ausblick [todo]}

\section{Zusammenfassung [todo]}

\section{Ausblick [todo]}



\begin{comment}
Eine aktuelle Recherche (3. Januar 2017) in den einschlägigen deutschen Bestellportalen (Ebay, Amazon, Conrad, Euronics, …) ergab, dass Geräte mit CWUSB-Unterstützung dort derzeit in Deutschland nicht bestellbar sind. Bei Amazon ließen sich in den Kundenrezensionen Spuren finden, dass im Jahr 2010 entsprechende Geräte auch vertrieben wurden. Zwei Entwicklungen machen es den Geräten schwer, sich am Markt zu behaupten: Einerseits wurde mit USB 3.0 die Datendurchsatzrate deutlich angehoben, was die Anforderungen an den Wireless-USB-Standard verschärft. Andererseits hat die Marktentwicklung bei den Smartphones die Verbreitung des Bluetooth-Standards stark ausgebaut. Während das Bluetooth-Konsortium seinen Standard laufend weiterentwickelt (zuletzt 2016 mit Version 5), datiert die letzte Version des USBCV-Tools für den Test und die Entwicklung von Wireless USB auf den 17. Juli 2009. Vor diesem Hintergrund erscheint es derzeit fraglich, ob CWUSB noch einmal aus der Versenkung auftauchen wird. [Wikipedia, Wireless USB]

Vielleicht sollte man sich diese Einschätzung für das Fazit aufgewahren? Komnsumer Markt nein, Spezial Markt ja.
\end{comment}