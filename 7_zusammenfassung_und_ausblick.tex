%%%%%%%%%%%%%%%%%%%%%%%%%%%%%%%%%%%%%%%%%%%%%%%%%%%%%%%%%%%%%%%%%%%%%%%%%%%%%%%%
%
%	- [Wikipedia, Wireless USB]
%		- Eine aktuelle Recherche (3. Januar 2017) in den einschlägigen deutschen Bestellportalen (Ebay, Amazon, Conrad, Euronics, …) ergab, dass Geräte mit CWUSB-Unterstützung dort derzeit in Deutschland nicht bestellbar sind.
%		- Bei Amazon ließen sich in den Kundenrezensionen Spuren finden, dass im Jahr 2010 entsprechende Geräte auch vertrieben wurden.
%		- Zwei Entwicklungen machen es den Geräten schwer, sich am Markt zu behaupten:
%			- Einerseits wurde mit USB 3.0 die Datendurchsatzrate deutlich angehoben, was die Anforderungen an den Wireless-USB-Standard verschärft.
%			- Andererseits hat die Marktentwicklung bei den Smartphones die Verbreitung des Bluetooth-Standards stark ausgebaut.
%			- Während das Bluetooth-Konsortium seinen Standard laufend weiterentwickelt (zuletzt 2016 mit Version 5), datiert die letzte Version des USBCV-Tools für den Test und die Entwicklung von Wireless USB auf den 17. Juli 2009. Vor diesem Hintergrund erscheint es derzeit fraglich, ob CWUSB noch einmal aus der Versenkung auftauchen wird. 
%	
%	- Vielleicht sollte man sich diese Einschätzung für das Fazit aufgewahren? Komnsumer Markt nein, Spezial Markt ja.
%	
%
%%%%%%%%%%
\chapter{Zusammenfassung und Ausblick}

Im Grundlagen-Kapitel wurden zu erste der Unterschied zwischen der Entfernungsmessung mittels Triangulation und der Trilateration beschrieben. Die Triangulation bestimmt die Entfernung durch das Messen der Winkel zwischen mehreren Referenzpunkten, während die Trilateration die Entfernungen anhand der Signallaufzeit bestimmt. Die Trilateration wird von den \glsuseri{uwbm} verwendet um Nachrichten auszutauschen. Durch den Nachrichtenaustausch ist es auch möglich die Entfernung zwischen zwei \glspl{uwbm} zu bestimmen. Dazu wird der Nachrichtenversand zu einem zukünftigen Zeitpunkt geplant, um den Zeitstempel des Sendevorgangs in die Nachricht einzubetten. Das empfangende \gls{uwbm} ist nun im Besitz aller Informationen um die Entfernung zu errechnen. Dies ist unter dem Namen \gls{sstwr}-Verfahren bekannt. Eine Verbesserung stellt das \gls{dstwr}-Verfahren dar, das für die Entfernungsmessung verwendet wird, da es den Fehler der lokalen Zeitgeber minimiert.

Im Geometrie-Abschnitt wurden die mathematischen Gleichungen für das Konstruieren und Berechnen der relevanten Längen des gleichseitigen Dreiecks und eines regelmäßigen Fünfecks beschrieben.

Die Wahrscheinlichkeitstheorie legt den Grundstein um die Funktionsweise der verschieden \gls{slam}-Varianten zu verstehen. Dabei wurden die Konzepte der Zufallsvariablen, der einfachen und mehrdimensionalen Normalverteilung und deren Gesetzmäßigkeiten wie die bedingte Wahrscheinlichkeit, die Abhängigkeiten zwischen Zufallsvariablen und der Satz von Bayes vorgestellt.

Mit einem Zustandschätzer ist es möglich, Abschätzungen über den zukünftigen Zustand eines Systems zu erstellen. Der Zustand beschreibt dabei alle Aspekte eines Roboters und seiner Umwelt die einen Einfluss auf die Zukunft haben können. Die Abschätzungen werden dabei durch die Steuerbefehle, die der Roboter an die Aktorik sendet, und die Wahrnehmungen der Umwelt gesteuert. Das interne Wissen des Roboters über den Zustand seiner Umwelt wird dabei als Belief bezeichnet. Die Basis aller Zustandschätzer bildet dabei der rekursive Bayes Filter, der jeden Zustand in zwei Schritten schätzt. Im Prognose-Schritt wird der nächste Zustand mit den Steuerbefehlen vorhergesagt und dann im Korrektur-Schritt mit den Wahrnehmungen korrigiert. Bei dem Bayes Filter handelt es sich um eine sehr abstrakten Algorithmus, der durch den Kalman Filter konkret umgesetzt wird. Der Kalman Filter nutzt dabei eine mehrdimensionale Normalverteilung um den Belief zu repräsentieren. Dadurch entstehen bei der Nutzung von nicht linearen Funktionen, die bei Rotationsbewegungen auftreten, aber auch Probleme. Diese werden durch den \gls{ekf} mittels einer Linearisierung der nicht linearen Funktion gelöst.

Bedingt durch die Verwendung einer Normalverteilung, können sowohl der Kalman Filter als auch der \gls{ekf} keine multimodalen Verteilungen darstellen. Diese Einschränkungen können durch den \gls{pf} umgangen werden. Dieser nutzt eine Menge von Partikeln um den Belief darzustellen. Dadurch ist es möglich eine beliebige Verteilung abzubilden. Jedes Partikel wird mit einem Gewicht versehen, um die Relevanz zu beschreiben. Im Resampling-Schritt werden die Partikel aussortiert, bei denen die Gewichtung unter einen festgelegten Grenzwert fällt, und durch neue Partikel ersetzt.

Wenn es notwendig wird, das nicht nur der Zustand des Roboters geschätzt, sondern auch gleichzeitig eine Karte der Umwelt erstellt wird, spricht man von einem \gls{slam}-Verfahren. Der \gls{ekf}-\gls{slam} gehört dabei zu den Standardverfahren der Robotik. Hierbei wird der Zustand des Roboters, als auch der der Umwelt mittels eines kombinierten Zustandsvektors modelliert. Das \gls{slam}-Verfahren kann auch mithilfe eines \gls{pf} gelöst werden, dazu muss der Zustandsvektor aufgespalten werden da \gls{pf} in großen Zustandsräumen nicht praktikabel sind. Der \gls{pf} stellt somit durch seine Partikel eine Hypothese des Pfades dar und jedem Partikel wird dann eine Menge von Landmarken zugewiesen.

Das Grundlagen-Kapitel schließt mit einer kurzen Betrachtung der \gls{ros}-Begrifflichkeiten ab. Hierzu gehört der \gls{ros}-Master, der die Verbindung zwischen verschieden Verarbeitungseinheiten, auch als Knoten bezeichnet, bereitstellt. Damit eine Kommunikation zwischen den Knoten stattfinden kann, werden Nachrichten über Datenbusse übertragen.




	
% Stand der Forschung und Technik
%- Probability Grids werden mit einander kombiniert
%- AUV
%	- Bereinigen der Daten über das Spectral Graph Partitioning Verfahren
%	- Eintragen der
%	- Voting-System
%	- EKF-SALM
%- PF mit Radialen Hilfspartikel filter.
%	-  Nach dem konvergierne wird ein ekf draus.
%- PF mit SOG
%	-  Nach dem konvergierne wird ein ekf draus.
%- Ringförmige Verteilung im Polarkoordinaten system

% Ulta-Wideband
%- Signale werden nicht auf eine sinusförmige Trägerfrequenz auf moduliert sondern im Basisband durch kurze Impulse
%- Große Bandbreite
%- Erstellte Hardware
%	- Anforderungen
%		- Gemeinsame Hardwareplatform, Separate STromversorgung, Kommunikation
%	- Hardware Zusammenstellung
%		- uwbt von decawave mit kommunikation über die SPI schnittstelle
%		- Warum ProTrinket anstatt ardunino
%		- Datenaustausch
%	- Prototypen
%		- Nachrichten austausch und Entfernungsmessung
%	- Platinendesign
%		- Berücksichtigen der Herstellerangaben
%	- Steuersoftware
%		- GitHub-Projekt
%		- DW1000Ranging
%		- Austausch Entfernungen, Übertragen an die Verarbeitungseinheit, Einspeisen ins ROS
%	- Kalibierung
%		- Antennenverzögerung
%		- DecaWave Verfahren
%		- LGS


% Umsetzung des RO-SLAM in ROS
%- Roboterplattform
%	- Robotino 2, 2D-Laser-Entfernungsmesser, holonomer Antrieb mit Inkrementalgeber, Verarbeitungseinheit
%	- Softwarearchitektur
%		- Transformatinosbaum mit Odometrie + Entfernungsmessungen
%	- ROS-Hauptmodule
%		- Robotino-Steuerung
%		- Koordindatentransformationen
%		- Teleoperation
%		- UWB-Entfernungsmessung
%	- ROS-Hilfsmodule
%		- 2D-Laser-Entfernungsmesser
%		- Belegtheitskarten
%		- Trajektorie
%		- Laser-Odometrie
%	- MRPT

% Evaluation
%- Batterielaufzeit
%- Kalibierung
%	- LGS
%	- DecaWave
%	- Normalverteilung der Messwerte
%	- Fünfeck
%- Entfernungsmessung
%	- LOS/NLOS
%	- 1-9m in 0.5 schritten
%	- Blech/Wasser
%- Trajektorie
%	- Testen der verschieden Odometriequellen
%	- Inkrementalgeber + laserbasierte
%- Positionsschätzung
%	- RF2O
%	- virtuelle + reale uwbm
%- Positionsschätzung der UWB-Module
%	- virtuelle werden auf ca. 0.15-0.97m geschätzt
%	- reale werden von ca 0.49-1,03m geschätzt
%- Konvergenz der WDF


%%%%%%%%%%%%%%%%%%%%%%%%%%%%%%%%%%%%%%%%%%%%%%%%%%%%%%%%%%%%%%%%%%%%%%%%%%%%%%%%
%
%	- UWB-Module
%		- Stärker Prozessor um den DWM mit maximaler SPI-Geschwindigkeit anzusteuern (20 Mhz)
%		- Identifier über DIP Schalter einstellbar
%		- Größerer Speicher um mehr Anchor/Tags verwalten zu können.
%		- Stromsparfunktionen
%		- vergleich mit dem kommerziellen Produkt
%	- RO-SLAM
%		- 
%
%%%%%%%%%%
\section{Ausblick}


%%%%%%%%%%%%%%%%%%%%%%%%%%%%%%%%%%%%%%%%%%%%%%%%%%%%%%%%%%%%%%%%%%%%%%%%%%%%%%%%
%
% Wurden alle Fragen aus dem Expose geklärt/beantwortet?
%
%%%%%%%%%%
\section{Fazit}

