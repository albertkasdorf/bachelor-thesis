%
% Spezielle Verzeichnisse (Glossar, Abkürzungsverzeichnis, Symbolverzeichnis) in LATEX mit glossaries und makeglossaries Vortrag bei DANTE 2015
% https://www.dante.de/events/dante2015/Programm/vortraege/vortrage-partosch.pdf
%
% How to combine Acronym and Glossary
% https://tex.stackexchange.com/questions/8946/how-to-combine-acronym-and-glossary
%
% LaTeX/Glossary
% https://en.wikibooks.org/wiki/LaTeX/Glossary
%


%
% \newacronym[plural={kurz-form-plural}, longplural={lang-form-plural}}{label}{kurz-form}{lang-form}
% \newacronym[plural={LEDs}, longplural={light-emitting diodes}]{led}{LED}{light-emitting diode}
% \newacronym[plural={}, longplural={}]{}{}{}
%
% \newacronym{<label>}{<abbrv>}{<full>}
% \newglossaryentry{<lable>}{name={}, sort={}, plural={}, description={}}
%


% ----------------------------------------------------------------------------------------
% A
\newacronym{aoa}{AOA}{Angle of Arrival}
\newacronym{auv}{AUV}{Autonomous Underwater Vehicle}

\newglossaryentry{anchor}
{
	name={Anker}, sort={Anker}, plural={Ankern},
	description={UWB-Modul das stationär befestig ist und sich nicht bewegt.}
}
\newglossaryentry{atmega}
{
	name={ATmega328/P}, sort={ATmega328/P},
	description={}
}

% ----------------------------------------------------------------------------------------
% B
\newglossaryentry{basisband}
{
	name={Basisband},
	description={Wikipedia: In der Nachrichtentechnik ist das Basisband der natürliche Frequenzbereich des Nutzsignals.},
}
\newglossaryentry{beacon}
{
	name={beacon}, sort={beacon}, plural={beacons},
	description={},
	user1={
	%https://en.wikipedia.org/wiki/Electric_beacon#Radio_beacons; https://de.wikipedia.org/wiki/Funkbake; http://www.spektrum.de/news/wo-bitte-gehts-zur-intelligenten-navigation/1410559
	}
}

% ----------------------------------------------------------------------------------------
% C
\newacronym{cua}{CUA}{Catholic University of America}
\newacronym{cs}{CS}{Chip Select}

% ----------------------------------------------------------------------------------------
% D
\newacronym{darpa}{DARPA}{Defense Advanced Research Projects Agency}
\newacronym{dod}{DoD}{Department of Defense}
\newacronym{doa}{DOA}{Direction of Arrival}
\newacronym{dstwr}{DS-TWR}{Double-sided Two-way Ranging}

% ----------------------------------------------------------------------------------------
% E
\newacronym{ekf}{EKF}{Extended Kalman Filter}

% ----------------------------------------------------------------------------------------
% F
\newacronym{fbw}{FBW}{Fractional Bandwidth}
\newacronym{fcc}{FCC}{Federal Communications Commission}
\newacronym{ftdi}{FTDI}{Future Technology Devices International}
\newglossaryentry{firmware}
{
	name={Firmware},
	sort={Firmware},
	description={Wikipedia: Software, die in elektronischen Geräten eingebettet ist.}
}

% ----------------------------------------------------------------------------------------
% G
\newacronym{gps}{GPS}{Global Positioning System}
\newacronym{gnss}{GNSS}{Global Navigation Satellite System}
\newacronym{gssi}{GSSI}{Geophysical Survey Systems Inc.}
\newacronym{gpr}{GPR}{Ground Penetrating Radar}
\newacronym{gpio}{GPIO}{General-purpose input/output}	% Allzweckeingabe/-ausgabe
\newglossaryentry{footprint}
{
	name={Footprint}, sort={Footprint}, plural={Footprints},
	description={Die Umrisse der Lötflächen von elektrischen Bauelementen auf einer Leiterplatte.}
} % TODO: Wikipedia: Footprint (engl. Fußabdruck, Fußspur) die Umrisse von Lötflächen von elektrischen Bauelementen auf einer Leiterplatte.

% ----------------------------------------------------------------------------------------
% H
\newacronym{hf}{HF}{Hochfrequenz}
\newglossaryentry{host}
{
	name={Host},
	sort={Host},
	description={Verarbeitungseinheit der Entfernungsdaten.}
}

% ----------------------------------------------------------------------------------------
% I
\newacronym{ins}{INS}{Inertial Navigation System}
\newacronym{ips}{IPS}{Indoor Positioning System}
\newacronym{ic}{IC}{Integrated Circuit}	% integrierter Schaltkreis
\newacronym{irq}{IRQ}{Interrupt Request}
\newacronym{ir}{IR}{Impulse Radio}
\newglossaryentry{inlier}
{
	name={inlier}, sort={inlier},
	description={In contrast, an inlier is an erroneous data value which actually lies in the interior of a statistical distribution, making it difficult to distinguish it from good data values.}
} % Quelle: http://ec.europa.eu/eurostat/statistics-explained/index.php/Glossary:Outlier

% ----------------------------------------------------------------------------------------
% J
\newacronym{json}{JSON}{JavaScript Object Notation}

% ----------------------------------------------------------------------------------------
% K

% ----------------------------------------------------------------------------------------
% L
\newacronym{los}{LoS}{Line of Sight}
\newacronym{llnl}{LLNL}{Lawrence Livermore National Laboratory}
\newacronym{lanl}{LANL}{Los Alamos National Laboratory}
\newacronym{led}{LED}{Light-emitting diode} % Leuchtdiode
%Lötpad; Lötpads sind die Lötflächen, welche die Verbindung zwischen Bauteil und Leiterbahnen darstellt.;http://www.kurtzersa.de/electronics-production-equipment/loetlexikon/begriff/loetpad.html

% ----------------------------------------------------------------------------------------
% M
\newacronym{mems}{MEMS}{Micro Electro Mechanical Sensors}
\newacronym{mir}{MIR}{Micropower Impulse Radar}
\newacronym{mosi}{MOSI}{Master Output Slave Input}
\newacronym{miso}{MISO}{Master Input Slave Output}
\newacronym{mc}{MC}{Monte Carlo}
\newacronym{mrpt}{MRPT}{Mobile Robot Programming Toolkit}
\newglossaryentry{multipath}
{
	name={multipath}, sort={multipath},
	description={},
	% Wikipedia: Mehrwegempfang oder Mehrwegeempfang (engl. Multipath) tritt an einem Empfänger auf, wenn elektromagnetische Wellen eines Senders von Reflektoren abgelenkt werden und auf verschiedenen Wegen beim Empfänger ankommen.
}

% ----------------------------------------------------------------------------------------
% N
\newacronym{nlos}{NLoS}{Non-line of Sight}

% ----------------------------------------------------------------------------------------
% O
\newacronym{osd}{OSD}{Office of the Secretary of Defense}
\newglossaryentry{outlier}
{
	name={outlier}, sort={outlier},
	description={An outlier is a data value that lies in the tail of the statistical distribution of a set of data values. In the distribution of raw data, outliers are often regarded as more likely to be incorrect.}
} % Quelle: http://ec.europa.eu/eurostat/statistics-explained/index.php/Glossary:Outlier
% http://mars.wiwi.hu-berlin.de/mediawiki/teachwiki/index.php/Ausrei%C3%9Fer: In der Statistik spricht man von einem Ausreißer, wenn ein Beobachtungswert nicht in eine erhobene Messreihe passt, also den Erwartungen widerspricht. Die Erwartung wird meistens als Streuungsbereich um den Erwartungswert herum definiert, in dem sich die meisten aller Messwerte nach der Messung befinden.

% ----------------------------------------------------------------------------------------
% P
\newacronym{pan}{PAN}{Personal Area Network}
\newacronym{pcb}{PCB}{Printed Circuit Board} % gedruckte Leiterplatte
\newacronym{p2p}{P2P}{Peer-to-Peer}
\newglossaryentry{pulldown_resistor}
{
	name={Pulldown--Widerstand},
	sort={Pulldown--Widerstand},
	description={Wikipedia: Der Begriff Pulldown bezeichnet in der Elektrotechnik einen (relativ hochohmigen) elektrischen Widerstand, der eine Signalleitung mit dem niedrigeren Spannungs-Potenzial verbindet, siehe Pulldown-Widerstand. (https://de.wikipedia.org/wiki/Pulldown)}
}
\newglossaryentry{payload}
{
	name={payload}, sort={payload},
	description={},
}
\newglossaryentry{propgrid}
{
	name={Probability Grid}, sort={Probability Grid}, plural={Probability Grids},
	description={TODO: Probability Grid},
}

% ----------------------------------------------------------------------------------------
% Q

% ----------------------------------------------------------------------------------------
% R
\newacronym{rtls}{RTLS}{Real-Time Location System}
\newacronym{rss}{RSS}{Received Signal Strength}
\newacronym{rssi}{RSSI}{Received Signal Strength Indication}
\newacronym{rf}{RF}{Radio Frequency}	% dt. Hochfrequenz (HF)
\newacronym{ros}{ROS}{Robot Operating System}
\newacronym{roslam}{RO--SLAM}{Range--Only Simultaneous Localization and Mapping}
\newacronym{rbpf}{RBPF}{Rao-Blackwellized Particle Filter}

% ----------------------------------------------------------------------------------------
% S
\newacronym{spi}{SPI}{Serial Peripheral Interface}
\newacronym{sclk}{SCLK}{Serial Clock}
\newacronym{ss}{SS}{Slave Select}
\newacronym{slam}{SLAM}{Simultaneous Localization and Mapping}
\newacronym{ss_radio}{SS}{Signal Strength}
\newacronym{sstwr}{SS-TWR}{Single-sided Two-way Ranging}

% ----------------------------------------------------------------------------------------
% T
\newacronym{tof}{ToF}{Time of Flight}
\newacronym{toa}{ToA}{Time of Arrival}
\newacronym{tdoa}{TDoA}{Time Difference of Arrival}
\newacronym{twr}{TWR}{Two Way Ranging}
\newglossaryentry{tag}
{
	name={Tag}, sort={Tag},
	description={UWB-Modul das am Roboter befestig ist und sich durch den Raum bewegt.}
}

% ----------------------------------------------------------------------------------------
% U
\newacronym{uwb}{UWB}{Ultra-Wideband} % Ultra WideBand (UWB),
\newacronym{usaf}{USAF}{United States Air Force}
\newacronym{usdod}{USDOD}{United States Department of Defense} % U.S. Department of Defense
\newacronym{uart}{UART}{Universal Asynchronous Receiver Transmitter}
\newacronym{usb}{USB}{Universal Serial Bus}
\newglossaryentry{uwbm}
{
	name={UWB--Modul}, sort={UWB--Modul}, plural={UWB--Module},
	user1={UWB--Modulen}, user2={UWB--Moduls},
	description={Erstelle Platine mit dem \gls{uwbt}, dem Mikrocontroller, der Energieversorgung, usw.}
}
\newglossaryentry{uwbt}
{
	name={UWB--Transceiver}, sort={UWB--Transceiver},
	description={DWM1000 \gls{ic}, also der DW1000 Mikrocontroller mit der notwendigen Beschaltung und der Antenne.}
}

% ----------------------------------------------------------------------------------------
% V

% ----------------------------------------------------------------------------------------
% W
\newacronym{wpan}{WPAN}{Wireless Personal Area Network}
% Bluetooth, also known as the IEEE 802.15.1 standard is based on a wireless radio system designed for short-range and cheap devices to replace cables for computer peripherals, such as mice, keyboards, joysticks, and printers. This range of applications is known as wireless personal area network (WPAN). \cite{lee2007comparative}
\newacronym{wlan}{WLAN}{Wireless Local Area Network}

% ----------------------------------------------------------------------------------------
% X

% ----------------------------------------------------------------------------------------
% Y

% ----------------------------------------------------------------------------------------
% Z
