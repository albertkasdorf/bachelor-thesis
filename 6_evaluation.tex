\begin{comment}
------------------------------------------------------------------------------------------
\end{comment}
\chapter{Evaluation}


\begin{comment}
------------------------------------------------------------------------------------------
\section{Versuchsaufbau [todo]}
\section{Ergebnisse und Auswertung [todo]}
\end{comment}


\begin{comment}
------------------------------------------------------------------------------------------
- TODO: Stimmen die 10Hz?
- Start 13:50-23:50, 12:50-20:00 => 10+7 => 17 Stunden
\end{comment}
\section{Batterielaufzeit}

Beim Test der Batterielaufzeit wurden zwei \gls{uwbm} in einem Abstand von \SI{4.7}{\metre} aufgestellt. Beide \gls{uwbm} hatten eine direkte \gls{los} zu einandern. Über den kompletten Zeitraum wurden Entfernungsmessungen mit einer Rate von \SI{10}{\hertz} durchgeführt. Als Testprogramme wurden dabei \textit{DW1000Ranging\_ANCHOR} und \textit{DW1000Ranging\_TAG} aus dem GitHub--Projekt \cite{Trojer2015} verwendet.

Der \Gls{anchor} hatte nach ca. \SI{17}{\hour} seinen Dienst eingestellt, wenig später folgte Ihm der \Gls{tag}. Deutlich höhere Batterielaufzeiten können dadurch erzielt werden, dass die Senderate reduziert wird und die Stromsparfunktionen sowohl des DWM1000 als auch des ATmega328/P genutzt werden.


\begin{comment}
------------------------------------------------------------------------------------------
- Mit welchen Einstellungen kommt man auf die Entfernungsmessung?
- Streuung?
- LOS/NLOS {Holz, Bücher, Menschlicher Körper}
	- Welcher Fehler ergibt zwischen LOS/NLOS?
- Wie verändert sich die Genauigkeit der Entfernungsmessung bei einer direkten Sichtverbindung (engl. Line--of--sight (LOS)) und indirekten Sichtverbindung (engl. Non--line--of--sight (NLOS))?
- isaacs2009optimal - Optimal sensor placement for time difference of arrival localization
- Diagramme
	- \cite{kurth2003experimental}
		- Fig. 2: Sample PDFs showing the true ranges associated with 20, 30, and 50 ft measured ranges. (X: true range, Y:count)
		- Fig. 3: The mean true distances to RF tags vs. measured distances (X:measured range, Y: true range)
		- Fig. 4: The variance in true distances to RF tags vs. measured distances (X:measured range (ft), Y: variance (ft^2))
	
- https://matheguru.com/stochastik/standardfehler.html
- https://de.wikipedia.org/wiki/Standardfehler
	
\end{comment}
\section{Entfernungsmessung}

Um die Charakteristik der Entfernungsmessung zu bestimmen, wurde der Versuchsaufbau aus der \figurename~\ref{fig:entfernungsmessung_versuchsaufbau} verwendet. Dabei wird der \Gls{tag} an einem fixen Ort befestigt und die Entfernung zu dem \Gls{anchor} gemessen. Es wurden dabei acht Entfernungen mit einem Abstand von einem Meter gemessen. Zu jeder Entfernung wurden \num{249} Messungen aufgezeichnet. Die tatsächliche Entfernung wird mit einem Laser Entfernungsmesser, der eine Genauigkeit von $\pm$~\SI{2}{\milli\meter} besitzt, bestimmt.

\begin{figure}[ht!]
  \centering
  \includegraphics[width=0.5\linewidth]{entfernungsmessung_versuchsaufbau}
	\caption{Versuchsaufbau der Entfernungsmessung.}
	\label{fig:entfernungsmessung_versuchsaufbau}
\end{figure}

Die Ergebnisse der Entfernungsmessung können der \tablename~\ref{tab:entfernungsmessung_stochastik} entnommen werden. Auffällig sind die zum Teil großen Abweichung der Mittelwerte von der tatsächlichen Entfernungen, siehe Entfernung \SI{3}{\meter} und \SI{7}{\meter}. Mit \SI{10}{\centi\meter} sind die größten Ausreißer vom Mittelwert bei \SI{7}{\meter} zu verzeichnen. Die restlichen liegen im Bereich von \SIrange{4}{8}{\centi\meter}. Die Standardabweichung liegt mit \SI{3}{\centi\meter} in einem sehr guten Bereich, siehe auch \figurename~\ref{fig:entfernungsmessung_punktwolke}.

\begin{table}[h!]
	\centering
	\begin{tabular}{||c||c|c|c|c|c|c||}
		\hline
		Entfernung [\si{\meter}] & $\overline{x}_{arithm}$ & $\sigma$ & $\sigma^2$ & $SE_{\overline{x}}$ & Min & Max\\\hline
		\hline
		\num{1.00} & \num{1.0401} & \num{0.0298} & \num{0.0009} & \num{0.0019} & \num{0.96} & \num{1.12}\\\hline
		\num{2.00} & \num{2.0766} & \num{0.0164} & \num{0.0003} & \num{0.0010} & \num{2.03} & \num{2.12}\\\hline
		\num{3.00} & \num{3.1288} & \num{0.0218} & \num{0.0005} & \num{0.0014} & \num{3.07} & \num{3.18}\\\hline
		\num{4.00} & \num{3.9104} & \num{0.0221} & \num{0.0005} & \num{0.0014} & \num{3.86} & \num{3.97}\\\hline
		\num{5.00} & \num{5.0746} & \num{0.0383} & \num{0.0015} & \num{0.0024} & \num{5.00} & \num{5.19}\\\hline
		\num{6.00} & \num{6.0965} & \num{0.0177} & \num{0.0003} & \num{0.0011} & \num{6.05} & \num{6.16}\\\hline
		\num{7.00} & \num{7.1509} & \num{0.0324} & \num{0.0010} & \num{0.0021} & \num{7.08} & \num{7.25}\\\hline
		\num{8.00} & \num{7.9356} & \num{0.0191} & \num{0.0004} & \num{0.0012} & \num{7.89} & \num{7.98}\\\hline
	\end{tabular}
	\caption{Stochastische Eigenschaften der Entfernungsmessungen.}
	\label{tab:entfernungsmessung_stochastik}
\end{table}

\begin{figure}[h!]
  \centering
  \includegraphics[width=0.5\linewidth]{entfernungsmessung_punktwolke}
	\caption{Verteilung der Messpunkte der ungeraden Entfernungsmessungen.}
	\label{fig:entfernungsmessung_punktwolke}
\end{figure}

In der \figurename~\ref{fig:entfernungsmessung_los_16440} wurden die ungeraden Entfernungsmessungen als Histogramm dargestellt. Gut zu erkennen ist die Normalverteilung der Messwerte um den Mittelwert.

\begin{figure}[h!]
	\centering
	\begin{subfigure}[b]{0.45\textwidth}
		\centering
		\includegraphics[width=\textwidth]{entfernungsmessung_los_1_16440}
		\caption{1 Meter}
		\label{fig:entfernungsmessung_los_1_16440}
	\end{subfigure}
	\hfill
	\begin{subfigure}[b]{0.45\textwidth}
		\centering
		\includegraphics[width=\textwidth]{entfernungsmessung_los_3_16440}
		\caption{3 Meter}
		\label{fig:entfernungsmessung_los_3_16440}
	\end{subfigure}
	\bigskip
	\begin{subfigure}[b]{0.45\textwidth}
		\centering
		\includegraphics[width=\textwidth]{entfernungsmessung_los_5_16440}
		\caption{5 Meter}
		\label{fig:entfernungsmessung_los_5_16440}
	\end{subfigure}
	\hfil
	\begin{subfigure}[b]{0.45\textwidth}
		\centering
		\includegraphics[width=\textwidth]{entfernungsmessung_los_7_16440}
		\caption{7 Meter}
		\label{fig:entfernungsmessung_los_7_16440}
	\end{subfigure}
	\caption{Histogramm und Wahrscheinlichkeitsdichtefunktion der ungeraden Entfernungsmessungen.}
	\label{fig:entfernungsmessung_los_16440}
\end{figure}


\begin{comment}
------------------------------------------------------------------------------------------
\end{comment}
\section{RO-SLAM [todo]}



