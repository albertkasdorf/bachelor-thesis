\begin{comment}
--------------------------------------------------------------------------------
\end{comment}
\chapter{Evaluation}


\begin{comment}
--------------------------------------------------------------------------------
\section{Versuchsaufbau [todo]}
\section{Ergebnisse und Auswertung [todo]}
\end{comment}


\begin{comment}
--------------------------------------------------------------------------------
- TODO: Stimmen die 10Hz?
- Start 13:50-23:50, 12:50-20:00 => 10+7 => 17 Stunden
\end{comment}
\section{Batterielaufzeit}

Beim Test der Batterielaufzeit wurden zwei \gls{uwbm} in einem Abstand von \SI{4.7}{\metre} aufgestellt. Beide \gls{uwbm} hatten eine direkte \gls{los} zu einandern. Über den kompletten Zeitraum wurden Entfernungsmessungen mit einer Rate von \SI{10}{\hertz} durchgeführt. Als Testprogramme wurden dabei \textit{DW1000Ranging\_ANCHOR} und \textit{DW1000Ranging\_TAG} aus dem GitHub--Projekt \cite{Trojer2015} verwendet.

Der \Gls{anchor} hatte nach ca. \SI{17}{\hour} seinen Dienst eingestellt, wenig später folgte Ihm der \Gls{tag}. Deutlich höhere Batterielaufzeiten können dadurch erzielt werden, dass die Senderate reduziert wird und die Stromsparfunktionen sowohl des DWM1000 als auch des ATmega328/P genutzt werden.


\begin{comment}
--------------------------------------------------------------------------------
- Mit welchen Einstellungen kommt man auf die Entfernungsmessung?
- Streuung?
- LOS/NLOS {Holz, Bücher, Menschlicher Körper}
	- Welcher Fehler ergibt zwischen LOS/NLOS?
- Wie verändert sich die Genauigkeit der Entfernungsmessung bei einer direkten Sichtverbindung (engl. Line--of--sight (LOS)) und indirekten Sichtverbindung (engl. Non--line--of--sight (NLOS))?
- isaacs2009optimal - Optimal sensor placement for time difference of arrival localization
- Diagramme
	- \cite{kurth2003experimental}
		- Fig. 2: Sample PDFs showing the true ranges associated with 20, 30, and 50 ft measured ranges. (X: true range, Y:count)
		- Fig. 3: The mean true distances to RF tags vs. measured distances (X:measured range, Y: true range)
		- Fig. 4: The variance in true distances to RF tags vs. measured distances (X:measured range (ft), Y: variance (ft^2))
	
- https://matheguru.com/stochastik/standardfehler.html
- https://de.wikipedia.org/wiki/Standardfehler
	
\end{comment}
\section{Entfernungsmessung}

Um die Charakteristik der Entfernungsmessung zu bestimmen, wurde der Versuchsaufbau aus der \figurename~\ref{fig:entfernungsmessung_versuchsaufbau} verwendet. Dabei wird der \Gls{tag} an einem fixen Ort befestigt und die Entfernung zu dem \Gls{anchor} gemessen. Es wurden dabei acht Entfernungen mit einem Abstand von einem Meter gemessen. Zu jeder Entfernung wurden \num{249} Messungen aufgezeichnet. Die tatsächliche Entfernung wird mit einem Laser Entfernungsmesser, der eine Genauigkeit von $\pm$~\SI{2}{\milli\meter} besitzt, bestimmt.

\begin{figure}[ht!]
  \centering
  \includegraphics[width=0.5\linewidth]{entfernungsmessung_versuchsaufbau}
	\caption{Versuchsaufbau der Entfernungsmessung.}
	\label{fig:entfernungsmessung_versuchsaufbau}
\end{figure}

Die Ergebnisse der Entfernungsmessung können der \tablename~\ref{tab:entfernungsmessung_stochastik} entnommen werden. Auffällig sind die zum Teil großen Abweichung der Mittelwerte von der tatsächlichen Entfernungen, siehe Entfernung \SI{3}{\meter} und \SI{7}{\meter}. Mit \SI{10}{\centi\meter} sind die größten Ausreißer vom Mittelwert bei \SI{7}{\meter} zu verzeichnen. Die restlichen liegen im Bereich von \SIrange{4}{8}{\centi\meter}. Die Standardabweichung liegt mit \SI{3}{\centi\meter} in einem sehr guten Bereich, siehe auch \figurename~\ref{fig:entfernungsmessung_punktwolke}.

\begin{table}[h!]
	\centering
	\begin{tabular}{||c||c|c|c|c|c|c||}
		\hline
		Entfernung [\si{\meter}] & $\overline{x}_{arithm}$ & $\sigma$ & $\sigma^2$ & $SE_{\overline{x}}$ & Min & Max\\\hline
		\hline
		\num{1.00} & \num{1.0401} & \num{0.0298} & \num{0.0009} & \num{0.0019} & \num{0.96} & \num{1.12}\\\hline
		\num{2.00} & \num{2.0766} & \num{0.0164} & \num{0.0003} & \num{0.0010} & \num{2.03} & \num{2.12}\\\hline
		\num{3.00} & \num{3.1288} & \num{0.0218} & \num{0.0005} & \num{0.0014} & \num{3.07} & \num{3.18}\\\hline
		\num{4.00} & \num{3.9104} & \num{0.0221} & \num{0.0005} & \num{0.0014} & \num{3.86} & \num{3.97}\\\hline
		\num{5.00} & \num{5.0746} & \num{0.0383} & \num{0.0015} & \num{0.0024} & \num{5.00} & \num{5.19}\\\hline
		\num{6.00} & \num{6.0965} & \num{0.0177} & \num{0.0003} & \num{0.0011} & \num{6.05} & \num{6.16}\\\hline
		\num{7.00} & \num{7.1509} & \num{0.0324} & \num{0.0010} & \num{0.0021} & \num{7.08} & \num{7.25}\\\hline
		\num{8.00} & \num{7.9356} & \num{0.0191} & \num{0.0004} & \num{0.0012} & \num{7.89} & \num{7.98}\\\hline
	\end{tabular}
	\caption{Stochastische Eigenschaften der Entfernungsmessungen.}
	\label{tab:entfernungsmessung_stochastik}
\end{table}

\begin{figure}[h!]
  \centering
  \includegraphics[width=0.5\linewidth]{entfernungsmessung_punktwolke}
	\caption{Verteilung der Messpunkte der ungeraden Entfernungsmessungen.}
	\label{fig:entfernungsmessung_punktwolke}
\end{figure}

In der \figurename~\ref{fig:entfernungsmessung_los_16440} wurden die ungeraden Entfernungsmessungen als Histogramm dargestellt. Gut zu erkennen ist die Normalverteilung der Messwerte um den Mittelwert.

\begin{figure}[h!]
	\centering
	\begin{subfigure}[b]{0.45\textwidth}
		\centering
		\includegraphics[width=\textwidth]{entfernungsmessung_los_1_16440}
		\caption{1 Meter}
		\label{fig:entfernungsmessung_los_1_16440}
	\end{subfigure}
	\hfill
	\begin{subfigure}[b]{0.45\textwidth}
		\centering
		\includegraphics[width=\textwidth]{entfernungsmessung_los_3_16440}
		\caption{3 Meter}
		\label{fig:entfernungsmessung_los_3_16440}
	\end{subfigure}
	\bigskip
	\begin{subfigure}[b]{0.45\textwidth}
		\centering
		\includegraphics[width=\textwidth]{entfernungsmessung_los_5_16440}
		\caption{5 Meter}
		\label{fig:entfernungsmessung_los_5_16440}
	\end{subfigure}
	\hfil
	\begin{subfigure}[b]{0.45\textwidth}
		\centering
		\includegraphics[width=\textwidth]{entfernungsmessung_los_7_16440}
		\caption{7 Meter}
		\label{fig:entfernungsmessung_los_7_16440}
	\end{subfigure}
	\caption{Histogramm und Wahrscheinlichkeitsdichtefunktion der ungeraden Entfernungsmessungen.}
	\label{fig:entfernungsmessung_los_16440}
\end{figure}


\begin{comment}
--------------------------------------------------------------------------------
- Diagramme
	- \cite{kurth2003experimental}
		- Fig. 5: (1) The ground truth path with tags indicated by circles. The numbers indicate how many range measurements were received from each tag over the duration of Test 1. (2) The path estimate from dead reckoning alone. (3) The path estimate from localization using a Kalman Filter. The Filter fuses data from odometry and a gyro with absolute measurements from RF tags to produce this path estimate. Numerical results are given in Table 1. (X: position in x(m), Y: position in y(m), Ground truth path with tag locations, Dead reckoning path, Kalman filter localization path)


- Versuchsdurchführung (Kalibierung)
	+ Warten bis alle Beacons registiert sind (10 Sekunden)
		+ rostopic echo "/beacon/sensed_data[0]/id"
	+ Anzahl der Messungen festlegen
		+ Wie dauert eine volle Messung
		+ 250=50s, 500=>100s, 1000=>200s
	+ ROS
		+ roscore
		+ beacon_publisher
		- beacon_writer	
			- filename
			- append
			- check measurement id
			- count per beacon
			- debug print
	+ Beacons mit einem Null Antennenverzögerung programmieren
		- Projekte erstellen mit Tag und Anchor
		- Programmierung von Linux aus
	+ Aufzeichen und Auswerten
		- Matlab Script erstellen
	+ Beacons mit der persönlichen Antennenverzögerung programmieren
		- Genauigkeit verifizieren
	- Ersatz Beacon erstellen
	- Gleichgewichtige Verkabelung für die Datenübertragung

- Versuchsdurchführung (LOS/NLOS)
	+ Nicht RF-Transparentes Material bestimmen
		+ Alumniumplatte (stark Dämpfung)
		+ Holzplatten (schwache Dämpfung)
		+ Behälter mit Wasser (mittlere Dämpfung)
	+ Messanzahl festlegen
		+ Dauer
			+ 1-1 Beacon => 8Hz
			+ 1-2 Beacon => 13Hz
			+ 1-3 Beacon => 17Hz
			+ 1-4 Beacon => 20Hz
		+ 250 => 31s, 500 => 62s, 1000 => 125s
	- Messabstände festlegen
		- 1:1:15 Meter? => 15 Messungen
		- 1:0.5:15 Meter? => 30 Messungen
		- 1:0.1:8 Meter? => 80 Messungen

- Aufzeichnungsrichtlinien (RO-SLAM)
	+ Robotino Odom zurücksetzen
		+ rosservice call /reset_odometry 0 0 0
	+ Warten bis alle Beacons registiert sind (10 Sekunden)
		+ rostopic echo "/beacon/sensed_data[0]/id"
	- Beacon Platzierung
		+ Beacons auf keinen Fall symmetrisch plazieren.
		+ Beacon-Position muss im Nachhinein bestimmt werden können.
		- Beacons erhöht positionieren, damit sie vom Laser erkannt werden.
		- Karte erstellen mit Beacon-Positionen
	- Robotino Platzierung
		+ Horizontale wird an dem Türrahmen ausgerichtet
		+ Rückkehren zum Startpunkt, sinnvoll wegen Drift.
		- Startpunkt festlegen auf der Karte
		- Startpunkt mit Tape abkleben
	- Skizze des Raums anfertigen mit Maßen
	- Platzierung von Kisten um Features für den LaserScan zu haben
	- Robotino Geschwindigkeit anpassen
		- Custom joystick teleop
			- [todo]
			- http://wiki.ros.org/teleop_twist_joy
			- http://yardbot.ca/2014/10/writing-custom-joystick-teleop-node-ros/
	- URDF-Modell anpassen
		- [todo] Position des Beacons anpassen
		
		
- Versuchsbeschreibung:
	- Warum wurden die uwbm da platziert wo sie jetzt stehen?
	
- Bauhaus
	- Rohr mit einer Höhe von x und einem Durchmesser 4-5 cm mit Kappe
	- Schrauben
	- Malerband?

\end{comment}
\section{RO-SLAM [todo]}


\begin{comment}
--------------------------------------------------------------------------------
\end{comment}
\subsection{Trajektorie [todo]}


\begin{comment}
--------------------------------------------------------------------------------
\end{comment}
\subsection{Trajektorie [todo]}


\begin{comment}
--------------------------------------------------------------------------------
\end{comment}
\subsection{Vergleich von MC und SOG [todo]}

