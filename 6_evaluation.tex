\begin{comment}
------------------------------------------------------------------------------------------
\end{comment}
\chapter{Evaluation [todo]}


\begin{comment}
------------------------------------------------------------------------------------------
\end{comment}
\section{Versuchsaufbau [todo]}


\begin{comment}
------------------------------------------------------------------------------------------
TODO: Stimmen die 10Hz?
\end{comment}
\subsection{Batterielaufzeit}

Beim Test der Batterielaufzeit wurden zwei \gls{uwbm} in einem Abstand von \SI{4.7}{\metre} aufgestellt. Beide \gls{uwbm} hatten eine direkte \gls{los} zu einandern. Über den kompletten Zeitraum wurden Entfernungsmessungen mit einer Rate von \SI{10}{\hertz} durchgeführt. Als Testprogramme wurden dabei \textit{DW1000Ranging\_ANCHOR} und \textit{DW1000Ranging\_TAG} aus dem GitHub--Projekt \cite{Trojer2015} verwendet.


\begin{comment}
------------------------------------------------------------------------------------------
\end{comment}
\section{Ergebnisse und Auswertung [todo]}


\begin{comment}
------------------------------------------------------------------------------------------
- Start 13:50-23:50, 12:50-20:00 => 10+7 => 17 Stunden
\end{comment}
\subsection{Batterielaufzeit}

Der \gls{anchor} hatte nach ca. \SI{17}{\hour} seinen Dienst eingestellt, wenig später folgte Ihm der \gls{tag}. Deutlich höhere Batterielaufzeiten können dadurch erzielt werden, dass die Senderate reduziert wird und die Stromsparfunktionen sowohl des DWM1000 als auch des ATmega328/P genutzt werden.