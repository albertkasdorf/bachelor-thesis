\begin{comment}
------------------------------------------------------------------------------------------
\end{comment}
\chapter{Evaluation}


\begin{comment}
------------------------------------------------------------------------------------------
\end{comment}
\section{Versuchsaufbau [todo]}


\begin{comment}
------------------------------------------------------------------------------------------
TODO: Stimmen die 10Hz?
\end{comment}
\subsection{Batterielaufzeit}

Beim Test der Batterielaufzeit wurden zwei \gls{uwbm} in einem Abstand von \SI{4.7}{\metre} aufgestellt. Beide \gls{uwbm} hatten eine direkte \gls{los} zu einandern. Über den kompletten Zeitraum wurden Entfernungsmessungen mit einer Rate von \SI{10}{\hertz} durchgeführt. Als Testprogramme wurden dabei \textit{DW1000Ranging\_ANCHOR} und \textit{DW1000Ranging\_TAG} aus dem GitHub--Projekt \cite{Trojer2015} verwendet.


\begin{comment}
------------------------------------------------------------------------------------------
- Versuchsaufbau
- Mit welchen Einstellungen kommt man auf die Entfernungsmessung?
- Streuung?
- LOS/NLOS {Holz, Bücher, Menschlicher Körper}
	- Welcher Fehler ergibt zwischen LOS/NLOS?
- Wie verändert sich die Genauigkeit der Entfernungsmessung bei einer direkten Sichtverbindung (engl. Line--of--sight (LOS)) und indirekten Sichtverbindung (engl. Non--line--of--sight (NLOS))?
- isaacs2009optimal - Optimal sensor placement for time difference of arrival localization
- Diagramme
	- \cite{kurth2003experimental}
		- Fig. 2: Sample PDFs showing the true ranges associated with 20, 30, and 50 ft measured ranges. (X: true range, Y:count)
		- Fig. 3: The mean true distances to RF tags vs. measured distances (X:measured range, Y: true range)
		- Fig. 4: The variance in true distances to RF tags vs. measured distances (X:measured range (ft), Y: variance (ft^2))
	
\end{comment}
\subsection{Entfernungsmessung [todo]}


\begin{comment}
------------------------------------------------------------------------------------------
\end{comment}
\section{Ergebnisse und Auswertung [todo]}


\begin{comment}
------------------------------------------------------------------------------------------
- Start 13:50-23:50, 12:50-20:00 => 10+7 => 17 Stunden
\end{comment}
\subsection{Batterielaufzeit}

Der \Gls{anchor} hatte nach ca. \SI{17}{\hour} seinen Dienst eingestellt, wenig später folgte Ihm der \Gls{tag}. Deutlich höhere Batterielaufzeiten können dadurch erzielt werden, dass die Senderate reduziert wird und die Stromsparfunktionen sowohl des DWM1000 als auch des ATmega328/P genutzt werden.