\begin{comment}
Fragestellung:
- Welche elektrische Beschaltung ist notwendig um das DWM1000 Modul von DecaWave in Betrieb nehmen zu können?
- Wie erfolgt die Entfernungsmessung zwischen den einzelnen UWB--Modulen?
- Wie erfolgt der Datenaustausch zwischen einem UWB--Modul und der Verarbeitungseinheit?
\end{comment}

\begin{comment}
------------------------------------------------------------------------------------------
\end{comment}
\chapter{Ultrabreitband}

\begin{comment}
- Was ist UWB überhaupt?
- Wie unterscheidet es sich von der bisherigen Verfahren?
- Auflisten einige Eigenschaften
\end{comment}


yavari2014ultra - Ultra wideband wireless positioning systems
	- The most important characteristic of UWB is large bandwidth in comparison with
prevalent narrow-band systems.
	- One result of the large bandwidth of UWB is that due to the inverse relatioship of
time and frequency, the life-time of UWB signals is very short. Consequently, the
time resolution of UWB signals is high and UWB is a good candidate for positioning
systems.

\cite{yang2004uwbcom}
	- Ultra-wideband communications: an idea whose time has come
	- UWB applications: short-range very high-speed broadband access to the Internet, covert communication links, localization at centimeter-level accuracy, high-resolution ground-penetrating radar, through-wall imaging, precision navigation and asset tracking, just to name a few.
	- UWB characterizes transmission systems with instantaneous spectral occupancy in excess of 500 MHz or a fractional bandwidth of more than 20%.
	- Such systems rely on ultra-short (nanosecond scale) waveforms that can be free of sine-wave carriers and do not require IF processing because they can operate at baseband. As information-bearing pulses with ultra-short duration have UWB spectral occupancy, UWB radios come with unique advantages that have long been appreciated by the radar and communications communities: i) enhanced capability to penetrate through obstacles; ii) ultra high precision ranging at the centimeter level; iii) potential for very high data rates along with a commensurate increase in user capacity; and iv) potentially small size and processing power.
	- This huge “new bandwidth” opens the door for an unprecedented number of bandwidth-demanding position-critical low-power applications in wireless communications, networking, radar imaging, and localization systems [64].
	- These include short-range, high-speed access to the Internet, accurate personnel and asset tracking for increased safety and security, precision navigation, imaging of steel reinforcement bars in concrete or pipes hidden inside walls, surveillance, and medical monitoring of the heart’s actual contractions.
	- For wireless communications in particular, the FCC regulated power levels are very low (below -41.3 dBm), which allows UWB technology to overlay already available services such as the global positioning system (GPS) and the IEEE 802.11 wireless local area networks (WLANs) that coexist in the 3.6--10.1 GHz band. Although UWB signals can propagate greater distances at higher power levels, current FCC regulations enable high-rate (above 110 MB/s) data transmissions over a short range (10--15 m) at very low power.
	- Wireless personal area networks (WPANs): Also known as in-home networks, WPANs address short-range (generally within 10--20 m) ad hoc connectivity among portable consumer electronic and communication devices. They are envisioned to provide high-quality real-time video and audio distribution, file exchange among storage systems, and cable replacement for home entertainment systems. UWB technology emerges as a promising physical layer candidate for WPANs, because it offers high-rates over short range, with low cost, high power efficiency, and low duty cycle.
	- Sensor networks: Sensor networks consist of a large number of nodes spread across a geographical area. The nodes can be static, if deployed for, e.g., avalanche monitoring and pollution tracking, or mobile, if equipped on soldiers, firemen, or robots in military and emergency response situations. Key requirements for sensor networks operating in challenging environments include low cost, low power, and multifunctionality. High data-rate UWB communication systems are well motivated for gathering and disseminating or exchanging a vast quantity of sensory data in a timely manner. Typically, energy is more limited in sensor networks than in WPANs because of the nature of the sensing devices and the difficulty in recharging their batteries. Studies have shown that current commercial Bluetooth devices are less suitable for sensor network applications because of their energy requirements [62] and higher expected cost [2]. In addition, exploiting the precise localization capability of UWB promises wireless sensor networks with improved positioning accuracy. This is especially useful when GPSs are not available, e.g., due to obstruction.
	- Imaging systems: Different from conventional radar systems where targets are typically considered as point scatterers, UWB radar pulses are shorter than the target dimensions. UWB reflections off the target exhibit not only changes in amplitude and time shift but also changes in the pulse shape. As a result, UWB waveforms exhibit pronounced sensitivity to scattering relative to conventional radar signals. This property has been readily adopted by radar systems (see e.g., [5] and references therein) and can be extended to additional applications, such as underground, through-wall and ocean imaging, as well as medical diagnostics and border surveillance devices [55], [57].
	- Vehicular radar systems: UWB-based sensing has the potential to improve the resolution of conventional proximity and motion sensors. Relying on the high ranging accuracy and target differentiation capability enabled by UWB, intelligent collision-avoidance and cruise-control systems can be envisioned. These systems can also improve airbag deployment and adapt suspension/braking systems depending on road conditions. UWB technology can also be integrated into vehicular entertainment and navigation systems by downloading high-rate data from airport off ramp, road-side, or gas station UWB transmitters.
		- By its rulemaking proposal in 2002, the Federal Communications Commission (FCC) in the United States essentially unleashed huge “new bandwidth’’ (3.6–10.1 GHz) at the noise floor, where UWB radios overlaying coexistent RF systems can operate using low-power ultra-short information bearing pulses.
	
\cite{win1998impulse}
	- Impulse radio: How it works
	
\cite{fontana2004recent}
	- Recent system applications of short-pulse ultra-wideband (UWB) technology	
	- In its infancy, UWB was commonly referred to as “carrier-free,” “baseband,” or “impulse,” reflecting the fact that the underlying signal generation strategy was the result of a broad-band extremely fast rise time, step, or impulse, which shock, or impulse, excited a wide-band antenna (e.g., TEM, mode horn).
	- The origins of UWB technology stem from work in time-domain electromagnetics begun in the early 1960s to fully describe the transient behavior of certain classes of microwave networks by examining their characteristic impulse response [7]–[12].
	- For up until 1962, there were no convenient means to observe, let alone measure, waveforms having subnanosecond durations, as were required to suitably approximate an ideal impulsive excitation. Fortuitously, at about the same time [15], Hewlett-Packard introduced the time-domain sampling oscilloscope, which greatly facilitated these measurements.
	- The last element that needed to be developed before real system development could begin was the short-pulse, or threshold, receiver. In the early 1970s, both avalanche transistor and tunnel diode detectors were constructed in attempts to detect these very short duration signals. The tunnel diode, invented in 1957 by Esaki who would later receive the Nobel Prize in physics in 1973 for this accomplishment, was the first known practical application of quantum physics. This unique device, with its extremely wide bandwidth (at the time, tens of gigahertz) permitted not only subnanosecond pulse generation essential for impulse excitation, but also could be used as a sensitive thresholding device for the detection of short-pulse waveforms.
	
Medical applications of ultra-wideband (UWB)
	- \cite{pan2007medical}
	
\cite{lakkundi2006ultra}
	- Ultra wideband communications: History, evolution and emergence
	- Figure 1 und 2 sind gut
	- The development of the sampling oscilloscope in the early 1960s and the corresponding techniques for generating sub-nanosecond baseband pulses speed up the development of UWB [1].
	
\cite{aiello2006ultra}
	- Ultra wideband systems: technologies and applications
	- The FCC's definition of the criteria for devices operating in the UWB spectrum purposely did not specify the techniques related to the generation and detection of RF energy; rather, it mandated compliance with emission limits that would enable coexistence and minimize the threat of harmful interference with legacy systems, thus protecting the Global Positioning System (GPS), satellite receivers, cellular systems, and others.
	- Interest in the technology has been steady, with more than 200 technical papers published in journals between 1960 and 1999 on the topic and more than 100 U.S. patents issued on UWB or UWB-related technology[3].


\begin{comment}
------------------------------------------------------------------------------------------
\end{comment}
\section{Historie}

Die Ursprunge der \ac{uwb} technologie liegen in dem ``spark gap transmission designs'' von Guglielmo Marconi in den späten 1890er Jahren und in der erfolgreichen ``cross-Atlantic'' Übertragung mit der Verwendung der ``spark'' Technik im Jahre 1901. \cite{fontana2004recent}

Vor der 1960er Jahren war die dominaten Funkübertragungsform sinusförmig. \cite{lakkundi2006ultra}

Die Erforschung der \ac{uwb} Technologie gehen auf die Anfänge der 1960er Jahre zurück. Zu diesem Zeitpunkt versuchte man die Ausbreitung elektromagnetischer Wellen nicht zur im Frequenz- sondern auch im Zeitbereich zu untersuchen. \cite{eltaher2004positioning, fontana2004recent, lakkundi2006ultra}

Grundlegende Erkenntnisse wurden dabei von \ac{llnl} und \ac{lanl} im bereich ``pulse transmitters'', ``receivers'' und ``antennas'' erforscht. 

Durch die Einführung der ``time-domain sampling oscilloscopes'' im Jahre 1962 durch Tektronix und Hewlett-Packard war es zum ersten Mal möglich eine UWB Wellenform aufzufangen und anzuzeigen. Dieses wurde erst durch den Einsatz von Avalanchetransistoren und Tunneldioden möglich. (high-speed triggering and detection) \cite{fontana2004recent, lakkundi2006ultra, aiello2006ultra}

Ab dem Jahre 1964 produzierten dann beide Hersteller Messgeräte für die Diagnose im Zeitbereich. \cite{barrett2001technical}

Im Jahre 1973 wurde das erste grundlegende Patent (3,728,632) von Ross für die \ac{uwb} Kommunikation eingereicht/gewährt. \cite{fontana2004recent, barrett2001technical, yang2004uwbcom}
\begin{comment}
US 3728632 A
Transmission and reception system for generating and receiving base-band pulse duration pulse signals without distortion for short base-band communication system
Veröffentlichungsnummer: US3728632 A
Publikationstyp: Erteilung
Veröffentlichungsdatum	: 17. Apr. 1973
Eingetragen: 12. März 1971
Erfinder:	Ross G
Ursprünglich Bevollmächtigter: Sperry Rand Corp
ZUSAMMENFASSUNG
An electromagnetic signal communication system utilizing short base-band pulse signals of sub-nanosecond duration employs dispersionless, broad band antenna transmission line elements for generating and preserving the character of the short base-band pulses in respective transmitter and receiver sub-systems.
\end{comment}

Die wichtigsten Patente für die \ac{uwb} Kommunikation- und Radarsysteme wurden von Harmuth an der \ac{cua}, Ross und Robbins bei der Sperry Rand Corp. and Paul van Etten an der \ac{usaf} im Rome Air Development Center eingereicht. \cite{barrett2001technical}

\begin{comment}
Rome Laboratory (Rome Air Development Center until 1991) is the US "Air Force 'superlab' for command, control, and communications"[4] research and development and is responsible for planning and executing the USAF science and technology program.
\end{comment}

Somit waren in den anfängen der 1970er Jahre alle Grundsteine für ein \ac{uwb} System für Kommunikation- bzw. Radaranwendungen gelegt.\cite{barrett2001technical}

Im Jahre 1974 fertige Morey ein \ac{uwb} Radarsystem für die ``Ground Penetration'' welche zu einem komerziellen Erfolg bei der \ac{gssi} führte. \cite{barrett2001technical}

In dem Zeitraum 1977 bis 1989 mehrere Workshops organisiert, darunter auch beim US-ame­ri­ka­nisch \ac{dod}, um das grundlegende Verständnis für die physik hinter der ``short pulse transmissions'' zu erörtern, die sich grundlegend von der ``continuous'' bzw. ``long pulse signals'' unterscheidet. \cite{barrett2001technical}

Ab dem Jahre 1989 wurde der Name \ac{uwb} durch das US-ame­ri­ka­nisch \ac{dod} geprägt. Diese Definition galt für alle Geräte die mindestens eine Bandbreite von \SI{1.5}{\GHz} bzw. mindestens \SI{25}{\percent} der ``Fractional Bandwidth'' belegt. Vorher war die Technologie nur unter den Synonymen ``baseband communication'', ``carrier free communication'', ``impulse radio'', ``large relative bandwidth communication'', ``nonsinusoidal communication'', ``orthogonal functions'', ``sequency theory'', ``time domain'', ``large-relative-bandwidth radio/radar signals'', ``video-pulse transmission'' und/oder ``Walsh waves communication'' bekannt. \cite{eltaher2004positioning, fowler1990assessment, yang2004uwbcom, aiello2006ultra, fontana2004recent}

Das Militär hatte dabei für sich die Anwendungsfälle im Bereich Radar und hochsicherheitskommunkation entdeckt.\cite{eltaher2004positioning}

Im Jahre 1994 wurde von T.E. McEwan an der \ac{llnl} das \ac{mir} erfunden. Hierbei handelte es sich um ein \ac{uwb} Radar, das mit den folgenden Eigenschaften aufwartete. Sehr geringer Energieverbrauch im Bereich von Mikrowatt, d.h. es stand einer batteriebetriebenen Anwendung nichts im Wege. Zusätzlich war es sehr komplakt und zum anderen kostengünstig in der Hergestellung. (The methods of reception of this design also permitted for the first time extremely sensitive signal detection.) \cite{barrett2001technical}

Ab dem Jahre 1998 beschäftigte sich die \ac{fcc} mit einer standartisierung der \ac{uwb} Nutzung. Im Jahre 2002 

Vor dem Jahre 2001 war die Verwendung von \ac{uwb} auf Radarssytem beschränkt die größtenteils in Militärischen Anwendungen verwendung fanden.\cite{yang2004uwbcom}

Das Änderte sich im Jahre 2002, als die \ac{fcc} in Amerika große Frequenzbereiche (\SIrange{3.6}{10.1}{\GHz}) für die kommerzielle Nutzung freigegeben hat, siehe First Report and Order (R\&O). Danach wurden auch Anwendungen im Bereich ``Imaging systems'', ``communication and measurement systems'' und ``vehicular radar systems'' möglich.\cite{yang2004uwbcom}

\begin{comment}
[12] FCC First Report and Order: In the matter of Revision of Part 15 of the
Commission’s Rules Regarding Ultra-Wideband Transmission Systems,
FCC 02–48, April 2002.
\end{comment}

Weitere Staten folgten der \ac{fcc} Regulierung und gaben ebenfalls große Frequenzbereiche für die \ac{uwb} Technologie frei. Details zu einzelnen Staten, siehe \cite{decawave2015uwbreg}.

\begin{comment}
Eine aktuelle Recherche (3. Januar 2017) in den einschlägigen deutschen Bestellportalen (Ebay, Amazon, Conrad, Euronics, …) ergab, dass Geräte mit CWUSB-Unterstützung dort derzeit in Deutschland nicht bestellbar sind. Bei Amazon ließen sich in den Kundenrezensionen Spuren finden, dass im Jahr 2010 entsprechende Geräte auch vertrieben wurden. Zwei Entwicklungen machen es den Geräten schwer, sich am Markt zu behaupten: Einerseits wurde mit USB 3.0 die Datendurchsatzrate deutlich angehoben, was die Anforderungen an den Wireless-USB-Standard verschärft. Andererseits hat die Marktentwicklung bei den Smartphones die Verbreitung des Bluetooth-Standards stark ausgebaut. Während das Bluetooth-Konsortium seinen Standard laufend weiterentwickelt (zuletzt 2016 mit Version 5), datiert die letzte Version des USBCV-Tools für den Test und die Entwicklung von Wireless USB auf den 17. Juli 2009. Vor diesem Hintergrund erscheint es derzeit fraglich, ob CWUSB noch einmal aus der Versenkung auftauchen wird. [Wikipedia, Wireless USB]

Vielleicht sollte man sich diese Einschätzung für das Fazit aufgewahren? Komnsumer Markt nein, Spezial Markt ja.
\end{comment}

	
\begin{comment}
------------------------------------------------------------------------------------------
\end{comment}
\section{Alternative Technologien}

\begin{comment}
Welche alternativen Technologien gibt es zu UWB?
\end{comment}

Einen guten Überblick über die Eigenschaften der Drahtlosen-Protokolle (engl. Wireless Protocols) Bluetooth, UWB, ZigBee und WiFi liefert die Arbeit \cite{lee2007comparative} von \citeauthor{lee2007comparative}.

qigao2015tightly - Tightly Coupled Model for Indoor Positioning based on UWB/INS


\begin{comment}
------------------------------------------------------------------------------------------
\end{comment}
\section{Gegenüberstellung}

\begin{comment}
Welche Eigenschaften haben die alternativen Technologien?
Warum hab ich mich für UWB entschieden?
\end{comment}


\begin{comment}
------------------------------------------------------------------------------------------
\end{comment}
\section{Erstelle Hardware}

Vor der Herstellung der UWB--Module werden die Produktspezifikationen des Herstellers untersucht um aus diesen die notwendige Beschaltung herzuleiten, siehe \cite{decawave2016dwm1kdatasheet, decawave2013power}. Zusätzlich werden Erfahrungsberichte aus dem Internet ausgewertet um die Beschaltung weiter zu verfeinern, siehe \cite{Trojer2015, Holder2016, Holder2016a}.

Der initiale Aufbau erfolgt zu Evaluationszwecken auf einem Steckboard und zusätzlich auf einer separaten Lochstreifenplatine um das Zusammenspiel zweier UWB--Module zu testen. Nach dem erfolgreichen Systemtest wird aus dem erstellten Schaltplan, ein PCB--Layout erstellt, mehrere PCB--Boards bestellt und nach der Lieferung zusammengebaut und noch mal getestet.


\begin{comment}
------------------------------------------------------------------------------------------
\end{comment}
\subsection{Elektrischer Aufbau}

\begin{comment}
Wie lange halten die Batterien durch?
	- 4.7 m, LOS, Start 13:50-23:50, 12:50-20:00
\end{comment}


\begin{comment}
------------------------------------------------------------------------------------------
\end{comment}
\subsection{Platinendesign}

\begin{comment}
\end{comment}


\begin{comment}
------------------------------------------------------------------------------------------
\end{comment}
\subsection{Steuersoftware}


\begin{comment}
------------------------------------------------------------------------------------------
\end{comment}
\subsection{Entfernungsmessung und Auswertung}

\begin{comment}
- Mit welchen Einstellungen kommt man auf die Entfernungsmessung?
- Streuung?
- LOS/NLOS {Holz, Bücher, Menschlicher Körper}
	- Welcher Fehler ergibt zwischen LOS/NLOS?
- Wie verändert sich die Genauigkeit der Entfernungsmessung bei einer direkten Sichtverbindung (engl. Line--of--sight (LOS)) und indirekten Sichtverbindung (engl. Non--line--of--sight (NLOS))?
\end{comment}

Der Überblick und Vergleich der verschiedenen Abstandsbestimmungsverfahren erfolgt über eine klassische Literatursuche, siehe \cite{lee2007comparative, herranz2010studying, zekavat2011handbook}.


isaacs2009optimal - Optimal sensor placement for time difference of arrival localization


\begin{comment}
------------------------------------------------------------------------------------------
\end{comment}
\subsection{Kalibrierung}

\begin{comment}
- Kalibierungsalgorithmus nach decaWave
	- Hab ich den Überhaupt richtig implementiert?
- Kalibierung über die Anpassung der einer Antennen Delay für alle.
- Kann über die Kalibrierung der Antennenverzögerung eine genauere Entfernungsmessung erreicht werden?
\end{comment}


Das Verfahren zur Kalibrierung der Antennenverzögerung kann ebenfalls der Hersteller--Dokumentation entnommen werden, siehe \cite{decawave2014calibration}. Hierfür muss ein Versuchsaufbau erstellt werden. Zusätzlich wird eine Anpassung der Steuer--/Auswerte-Software notwendig, um die Verzögerung zu berechnen.

Die Genauigkeitsbestimmung der Entfernungsmessung mit LOS und NLOS wird über einen Versuchsaufbau realisiert. Hierfür werden mehrere Messreihen in verschiedenen Abständen aufgenommen und mit der tatsächlichen Entfernung verglichen.