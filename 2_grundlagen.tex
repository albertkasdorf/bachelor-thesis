\begin{comment}
------------------------------------------------------------------------------------------
\end{comment}
\chapter{Grundlagen [todo]}


\begin{comment}
------------------------------------------------------------------------------------------
- Wie lange dauert es bis eine Nachricht ausgetauscht worden ist?
	- Beispiel mit einer konkreten Entfernung?
- Wie schnell drifted ein Quarz in einem µc?
	- What is the ppm in the crystal oscillator?
	- https://electronics.stackexchange.com/questions/15851/what-is-the-ppm-in-the-crystal-oscillator
	- In the 1930s, such precise time measurements simply weren't possible; a clock of the required accuracy was difficult enough to build in fixed form, let alone portable. A crystal oscillator, for instance, drifts about 1 to 2 seconds in a month, or 1.4x10−3 seconds an hour.[1] This may sound small, but as light travels 3x108 m/s, this represents a drift of 400 m per hour. Only a few hours of flight time would render such a system unusable, a situation that remained in force until the introduction of commercial atomic clocks in the 1960s.
	- Clock accuracy in ppm
	- http://www.best-microcontroller-projects.com/ppm.html

		
- \cite{decawave2014rtls}
	- Real time location systems - An Introduction
		- Unsynchronized Transmitter and Receiver: In this case the tagged object communicates with each of the fixed anchors in what’s called a two-way ranging exchange. The tag and each anchor exchange timing information so that the anchor can calculate the time-of-flight of the signal from the tag to the anchor without the necessity for tag and anchor to be synchronized in time. Once each anchor has this information, a location engine can calculate the position of the tagged object.
			- This requires that the tag be capable of receiving as well as transmitting which means that this method generally consumes more power than the Time Difference of Arrival solution we’ll see in the next section.
				 The anchor transmits a message to the tag and records the time the message left its antenna (let’s call it t1).
				 The tag receives the message and sends back a reply.
				 The anchor records the time it receives the reply (let’s call it t2)
				 The anchor then calculates the time difference Tr = t2 – t1
				 The anchor then calculates the distance using the formula d = cTr/2, where c is the speed of light.
			- This is what’s known as Two Way Ranging.
			
- \cite{decawave2015twr}
	- The implementation of two-way ranging with the DW1000
	- In this application note two-way ranging (TWR) scheme as used by Decawave’s example application (DecaRanging) is described. TWR is a basic concept to calculate the distance between two objects by determining the time of flight (TOF) of signals travelling between them.
	- The DW1000 uses mathematical and electronic techniques to implement a very precise clock. By recording the state of this clock when certain events occur during DW1000 transmission and reception of the radio wave signals, the DW1000 has the ability to ‘timestamp’ those events.
	- DW1000 based TWR
		- The initiator transmits a radio message to the responder and records its time of transmission (transmit timestamp) t1. The responder receives the message and transmits a response (a radio message) back to the initiator after a particular delay treply. The initiator then receives this response and records a receive timestamp t2.
		- If we assume the speed of radio waves through air is the same as the speed of light c, then the distance between the initiator and responder can be calculated by,
		- there are a number of sources of error due to clock drift and frequency drift [4]. Asymmetric double sided TWR method is used in Decawave’s implementation. It reduces the error due to clock and frequency drift.
	- Discovery and Ranging phase message exchanges
	
- \cite{decawave2016dw1kusermanual}
	- Single-sided two-way ranging (SS-TWR)
	- Double-sided two-way ranging (DS-TWR)
		- The four messages of DS-TWR, shown in Figure 37, can be reduced to three messages by using the reply of the first round-trip measurement as the initiator of the second round-trip measurement.
	- Error
		- Where the clock in device A runs at ka times the desired frequency and the clock in device B runs at kb times the desired frequency and both ka & kb are close to 1.
		- To give some idea of the size of this error, if devices A and B have clocks where each are 20 ppm away (the worst case specification) from the nominal clock in directions which make their combined error additive and equal to 40 ppm, then ka and kb might both be 0.99998 or 1.00002.
		- Even with a relatively large UWB operating range of say 100 m, the TOF is just 333 ns, so the error is 20 × 10-6 × 333 × 10-9 seconds, which is 6.7 × 10-12 seconds or 6.7 picoseconds which is approximately 2.2 mm.
		
	

https://de.wikipedia.org/wiki/Fading_(Elektrotechnik)
\end{comment}
\section{Verfahren für die Entfernungsbestimmung}

Bei der \textit{Triangulation} werden die Winkel zwischen mehreren Referenzpunkten bestimmt und dann die dazu gehörige Entfernung mittels trigonometrischer Funktionen berechnet. Dieses Verfahren ist auch unter den Namen \Gls{aoa} bzw. \Gls{doa} bekannt. Um eine genau Ortsbestimmung durchzuführen müssen die Winkel sehr genau bestimmt werden. Um das zu bewerkstelligen werden im Empfänger mehrere Antennen zu einem Feld (engl. Antenna Array) zusammengefasst. Jedoch ist diese Konstruktion sehr teuer und empfindlich für Mehrwegeempfang (engl. Multipath) bzw. Signalabschattungen. \cite{gezici2005localization, liu2007survey, decawave2014rtls}

Im Gegensatz dazu werden bei der \textit{Trilateration} die Entfernungen zwischen mehreren Referenzpunkten betrachtet. Es werden dabei die Verfahren \Gls{toa} und \Gls{tdoa} unterschieden.
Bei dem \Gls{toa}--Verfahren wird zuerst die Zeitdifferenz zwischen dem Senden und Empfangen eines Funksignals berechnet. Mittels der Signallaufzeit (engl. \acrfull{tof}) und der Ausbreitungsgeschwindigkeit des Funksignals kann die Entfernung berechnet werden. Die Ortsbestimmung erfolgt dann über die Schnittpunkte von drei Kreisen (2D) bzw. vier Kugeln (3D) miteinander. Um dieses Verfahren anwenden zu können, ist es erforderlich das das Funksignal mit einem Zeitstempel des Startzeitpunktes versehen ist. Daraus folgt aber auch, das die Zeit zwischen Sender und Empfänger sehr genau synchronisiert werden müssen um den Fehler möglich klein zu halten.
Bei dem \Gls{tdoa}--Verfahren werden die Zeitdifferenz zwischen dem Empfang des Funksignals an mehreren Empfängern ausgewertet. Dies hat den großen Vorteil das nur noch die Zeit zwischen den Empfängern synchronisiert werden muss. \cite{zekavat2011handbook, decawave2014rtls}

Neben der \textit{Triangulation} und \textit{Trilateration} besteht auch die Möglichkeit auf Grund der empfangenen Signalstärke (engl. \acrfull{ss_radio}, \acrfull{rss} oder auch \acrfull{rssi}) Rückschlüsse über die Entfernung zu ziehen. Dazu muss die ursprüngliche Signalstärke und die Ausbreitungscharakteristik der elektromagnetischen Welle in der spezifischen Umgebung bekannt sein. \cite{gezici2005localization, decawave2014rtls}

In den nächsten zwei Abschnitten werden die \textit{DecaWave} Entfernungsmessverfahren vorgestellt.

\begin{comment}
------------------------------------------------------------------------------------------

\end{comment}
\subsection{\acrlong{sstwr} [todo]}


\begin{comment}
------------------------------------------------------------------------------------------
\end{comment}
\subsection{\arclong{dstwr} [todo]}


\begin{comment}
------------------------------------------------------------------------------------------
- Theorie: Wahrscheinlichkeitsverfahren
	- Positionsschätzer in Form einer Wahrscheinlichkeitsverteilung über den Zustandraum.
	- Kalman fitering
		- Multivariate Gaussian distribution (Mehrdimensionale Normalverteilung)
		- \url{https://de.wikipedia.org/wiki/Mehrdimensionale_Normalverteilung}
		- Kompakte Beschreibung der Normalverteilung über den Erwartungswert $\mu$ und die Kovarianzmatrix $\Sigma$ ($\mu$ und $\sigma^2$)
		- \url{https://matheguru.com/stochastik/normalverteilung.html}
	- Markov methods
		- Probability Grid
		- Robot--Position ist diskretisiert
		- Nutzen von Bayes Rule um Grids zu kombinieren/neuerzeugen
	- Monte Carlo Lokalisierung
		- Multimodal Distribution for position estimation
		- Important Sampling
\end{comment}
\section{Wahrscheinlichkeitstheorie [todo]}


\begin{comment}
------------------------------------------------------------------------------------------
- Theorie: Lokalisierungsprobleme
	- Statische Lokalisierung
		- Akkurate Schätzung seiner globalen Position anhand der Sensordaten
		- Annahme: Umgebungskarte der Landmarken ist vorhanden
	- Position Tracking/Positionsverfolgung
		- Initiale Position ist gegeben
		- Verfogenden der Roboterposition
		- Annahme: Umgebungskarte der Landmarken ist vorhanden
	- SLAM
		- Verwenden der Sensordaten um sich zu Lokalisierung...
		- und eine Karte der Landmarken zu erzeugen.
		- Bisher Winkel und Entferung zu einer Landmarke gegeben
			- Computer Vision, Structure from Motion
		- Hier nur die Entfernung
\end{comment}
\section{Bayes/Kalman/Partikel Filter [todo]}


\begin{comment}
------------------------------------------------------------------------------------------
- \cite{kalman1960new}
- \cite{kurth2003experimental}
	- Originally introduced in 1960, the Kalman lter assumes a multivariate Gaussian distribution [6]. The Kalman lter has the advantage that the representation of the distribution is compact; a Gaussian distribution can be represented by a mean and a covariance matrix. The robot's pose estimation is maintained as a Gaussian distribution and sensor data from dead reckoning and landmark observations is fused to obtain a new position distribution.
	- Our results with Kalman ltering require an under-standing of the characteristics of the noise present in ranges reported by the radio tags. We gain this by look-ing at the probability distribution functions for each tag measurement.
	- We obtain the PDFs as follows: for every reported measurement, we nd the true range to the robot when that distance was reported. We do this by comparing the known location of the reporting tag to the times-tamped true location of the robot when the report was received.
	- the covariance matrix, which describes the uncertainty and correlation of the terms in the state estimate.
	- However, when the same initial noisy tag locations are used with Test 2, our SLAM technique fails to converge. Since the Kalman lter uses a linearization of the nonlinear range measurements, if the linearized estimate is too far away from the truth, the lter may be unable to recover and will diverge.
	-

\end{comment}
\section{Kalman Filter [todo]}




\begin{comment}
------------------------------------------------------------------------------------------
- \cite{kurth2003experimental}
	- Recent extensions of Kalman ltering allow for non-Gaussian, multimodal probability distributions through multiple hypothesis tracking. The result is a more versatile estimation technique that still preserves many of the computational advantages of the Kalman filter.
\end{comment}
\subsection{Extended Kalman Filter [todo]}

% Linearisierung


\begin{comment}
------------------------------------------------------------------------------------------
\end{comment}
\section{Partikel Filter [todo]}

\begin{comment}
------------------------------------------------------------------------------------------
- \cite{kurth2003experimental}
	- Monte Carlo localization, or particle ltering, provides a method of representing multimodal distri-butions for position estimation [4, 12], with the ad-vantage that the computational requirements can be scaled. The main advantage of these methods is their ability to recover robustly from a poor initial condition.
- \cite{fox1999monte}
\end{comment}
\subsection{Monte Carlo [todo]}


\begin{comment}
------------------------------------------------------------------------------------------
Rao-Blackwellized Particle Filtering
https://people.eecs.berkeley.edu/~pabbeel/cs287-fa12/slides/RBPF.pdf
\end{comment}
\subsection{Rao-Blackwellized [todo]}


\begin{comment}
------------------------------------------------------------------------------------------
- \cite{kurth2003experimental}
	- We are currently developing a batch localization method, which considers all the data collected by the robot and nds the best path estimate given all the data. Although time consuming computationally, this will produce the theoretically optimal result obtainable from the collected data; we can then evaluate the results of our online localization method by comparing to this optimal solution.
\end{comment}
\section{Batch optimization [todo,optional]}


\begin{comment}
------------------------------------------------------------------------------------------
- \cite{kurth2003experimental}
	- Additionally, we will extend the batch method to produce a variable dimension lter, as used by Deans for the case of bearing-only sensors [3], which would consider some window of previous robot states and optimize the position estimates based on the data in that window.
\end{comment}
\section{Variable Dimension Filter [todo,optional]}


\begin{comment}
------------------------------------------------------------------------------------------
Embodied Localisation and Mapping
http://elib.suub.uni-bremen.de/edocs/00103537-1.pdf

- \cite{sarkka2013bayesian}
	- Bayesian filtering and smoothing
- \cite{kurth2003experimental}
	- The Kalman lter approach described in Section 5 can be reformulated for the SLAM problem. To perform SLAM, we include position estimates for each tag in the state, producing a state vector of the form: q(k) = [xk; yk; k; xb1; yb1 ; :::; xbn; ybn]T , where n is the number of beacons.
\end{comment}
\section{SLAM [todo]}

% Zuerst wird die Position geschätzt und danach die Positionen der Landmarken.


\begin{comment}
------------------------------------------------------------------------------------------
- Was ist ROS?
- Von wem wurde es Entwickelt bzw. weiterentwickelt?
- Wie funktioniert ros?
- Welche Hauptkomponenten gibt es in ROS? ROSCORE, Messages (siehe ROS Summer School)
=> Maximal eine Seite
\end{comment}
\section{ROS [todo]}