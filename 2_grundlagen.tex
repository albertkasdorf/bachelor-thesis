\begin{comment}
------------------------------------------------------------------------------------------
\end{comment}
\chapter{Grundlagen}


\begin{comment}
------------------------------------------------------------------------------------------
\end{comment}
\section{Verfahren für die Reichweiten-Bestimmung}


\begin{comment}
------------------------------------------------------------------------------------------
- Theorie: Wahrscheinlichkeitsverfahren
	- Positionsschätzer in Form einer Wahrscheinlichkeitsverteilung über den Zustandraum.
	- Kalman fitering
		- Multivariate Gaussian distribution (Mehrdimensionale Normalverteilung)
		- \url{https://de.wikipedia.org/wiki/Mehrdimensionale_Normalverteilung}
		- Kompakte Beschreibung der Normalverteilung über den Erwartungswert $\mu$ und die Kovarianzmatrix $\Sigma$ ($\mu$ und $\sigma^2$)
		- \url{https://matheguru.com/stochastik/normalverteilung.html}
	- Markov methods
		- Probability Grid
		- Robot--Position ist diskretisiert
		- Nutzen von Bayes Rule um Grids zu kombinieren/neuerzeugen
	- Monte Carlo Lokalisierung
		- Multimodal Distribution for position estimation
		- Important Sampling
\end{comment}
\section{Wahrscheinlichkeitstheorie}


\begin{comment}
------------------------------------------------------------------------------------------
- Theorie: Lokalisierungsprobleme
	- Statische Lokalisierung
		- Akkurate Schätzung seiner globalen Position anhand der Sensordaten
		- Annahme: Umgebungskarte der Landmarken ist vorhanden
	- Position Tracking/Positionsverfolgung
		- Initiale Position ist gegeben
		- Verfogenden der Roboterposition
		- Annahme: Umgebungskarte der Landmarken ist vorhanden
	- SLAM
		- Verwenden der Sensordaten um sich zu Lokalisierung...
		- und eine Karte der Landmarken zu erzeugen.
		- Bisher Winkel und Entferung zu einer Landmarke gegeben
			- Computer Vision, Structure from Motion
		- Hier nur die Entfernung
\end{comment}
\section{Bayes/Kalman/Partikel Filter}


\begin{comment}
------------------------------------------------------------------------------------------
\end{comment}
\section{Kalman Filter}


\begin{comment}
------------------------------------------------------------------------------------------
\end{comment}
\subsection{Extended Kalman Filter}

% Linearisierung


\begin{comment}
------------------------------------------------------------------------------------------
\end{comment}
\section{Monte Carlo Partikel Filter}


\begin{comment}
------------------------------------------------------------------------------------------
\end{comment}
\section{SLAM}

% Zuerst wird die Position geschätzt und danach die Positionen der Landmarken.


\begin{comment}
------------------------------------------------------------------------------------------
\end{comment}
\section{ROS}