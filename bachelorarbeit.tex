\documentclass[
	%draft,
	11pt,
	a4paper,
	twoside,
	titlepage,
	openright,
]{scrbook}

\usepackage[utf8]{inputenc}
\usepackage[T1]{fontenc}
\usepackage{lmodern}
\usepackage{comment}
\usepackage[ngerman]{babel}
\usepackage[autostyle]{csquotes}

\usepackage{showframe}
\usepackage[
	a4paper,
	showframe,
% FH Vorlage :$
%	inner=35mm,
%	outer=25mm,
%	top=20mm,
%	bottom=25mm,
%	headsep=1em,
%	headheight=1.5em,
%	includehead,
%	footskip=3em,
%	includefoot
]{geometry}

% change normal text style to 1.5 lineheight and remove paragraph indent
% Das Setspace Paket ermöglicht es auch recht einfach Weise den Zeilenabstand zu ändern. Da es bei einer Vielzahl von Arbeiten gefordert wird, dass der Zeilenabstand zum Beispiel 1.5 betragen soll oder das ein doppelter Zeilenabstand gewünscht wird. Im Regelfall werden Seminar-, Bachelor-, Master- und Diplomarbeiten mit eineinhalbfachen Zeilenabstand gesetzt.
%\usepackage[onehalfspacing]{setspace}
%	\setlength{\parindent}{0em}
%	\setlength{\parskip}{0.5em}

% fix widows (Hurenkinder) and orphans (Schusterjungen)
% http://mirror.hmc.edu/ctan/macros/latex/contrib/nowidow/nowidow.pdf
%\usepackage[defaultlines=4,all]{nowidow}

% use color package and define fh logo colors
\usepackage{color}
	\definecolor{fh-black}{RGB}{0,0,0}
	\definecolor{fh-white}{RGB}{255,255,255}
	\definecolor{fh-turquoise}{RGB}{0,177,172}

% Quelltexte (Programmlistings)
\usepackage{listings}
  % Listings caption [duplicate]
  % https://tex.stackexchange.com/questions/54819/listings-caption
  \renewcommand{\lstlistingname}{Auflistung}
  
\usepackage{url}
\usepackage[
	breaklinks=true,
	colorlinks=true,
	linkcolor=black,
	urlcolor=black,
	citecolor=black,
]{hyperref}

% The biblatex Package - Programmable Bibliographies and Citations
% 	- http://ftp.math.purdue.edu/mirrors/ctan.org/macros/latex/exptl/biblatex/doc/biblatex.pdf
% Biblatex citation order
%	- https://tex.stackexchange.com/questions/51434/biblatex-citation-order
\usepackage[
	backend=biber,
	style=numeric-comp,
	% Sort by year (descending), name, title.
	%sorting=ydnt,
	% Do not sort at all. All entries are processed in citation order.
	sorting=none,
]{biblatex}
	\addbibresource{C:\\Users\\Albert\\Documents\\Studium\\Bachelor WS17\\sources.bib}

% How can I use @author, @date, and @title after maketitle?
% https://tex.stackexchange.com/questions/27710/how-can-i-use-author-date-and-title-after-maketitle
\usepackage{titling}
	\title{Range Only Simultaneous Localization and Mapping with Ultra-Wideband}
	\author{Albert Kasdorf}

% Page layout in LATEX
% http://www.ntg.nl/maps/16/29.pdf
%\usepackage{fancyhdr}
%	\pagestyle{fancy}
%	\lhead{}\chead{}\rhead{}
%	\lfoot{}\cfoot{\thepage}\rfoot{}
%	\renewcommand{\headrulewidth}{0.0pt}
%	\renewcommand{\footrulewidth}{0.4pt}

% siunitx — A comprehensive (SI) units package
% http://www.bakoma-tex.com/doc/latex/siunitx/siunitx.pdf
\usepackage{siunitx}

\usepackage{graphicx}
  \graphicspath{ {Bilder/} }
  
\usepackage{eso-pic}

%\usepackage{tikz}
%	\usetikzlibrary{positioning}
%	\usetikzlibrary{calc}
%	\usetikzlibrary{decorations.markings}
%	\usetikzlibrary{fit}

\usepackage{float}

%
% acronym – Abkürzungsverzeichnis mit LATEX
% http://namsu.de/Extra/pakete/Acronym.pdf
%
\usepackage[
	nohyperlinks,
	printonlyused,
	withpage,
	%footnote,
	%smaller,
]{acronym}
%\usepackage[numbers,nonamebreak,sort]{natbib}

% configure "fancy" page style
%\usepackage{fancyhdr}
%\pagestyle{fancy}
%\fancyhf{}
%\renewcommand\headrulewidth{0.2mm}
%\renewcommand\footrulewidth{0mm}
%\fancyhead[OR,EL]{\leftmark}
%\fancyfoot[OR,EL]{\thepage}
%\fancypagestyle{plain}{
%    \fancyhead{}% remove header at chapter beginning
%    \renewcommand{\headrulewidth}{0mm}
%}

% math
\usepackage{amssymb}
\usepackage{amsthm}

% other packages
\usepackage{enumitem}
\usepackage{lipsum}

%
% The tocloft package provides means of controlling the typographic design of the Table of Contents, List of Figures and List of Tables.
% http://mirrors.rit.edu/CTAN/macros/latex/contrib/tocloft/tocloft.pdf
%
%\usepackage{tocloft}

%
% http://ctan.math.illinois.edu/macros/latex/contrib/fncychap/fncychap.pdf
%
%\usepackage[Sonny]{fncychap}


\begin{document}

	% Dieser Befehl wird oftmals in großen Dokumenten für die ersten Seiten verwendet. Ziel ist es hierbei die Seiten wie Sperrklausel, Danksagung, Tabellen- oder Inhaltsverzeichnis von dem eigentlichen Inhalt des Dokumentes zu trennen. Dies geschieht durch Umschaltung der Seitennummerierung von arabisch auf römisch, wobei die Seitenzählung wieder bei 1 beginnt. 
	\frontmatter

	% titlepage
	%
% http://tobiw.de/tbdm/titelseiten
%
\begin{titlepage}
	\newlength
	\logoheight
	\settoheight
	\logoheight{\includegraphics[]{fh_logo_right_black.pdf}}
	\AddToShipoutPicture*{
		\makebox[\paperwidth][r]{
			\raisebox{\dimexpr(\paperheight-\height-20mm)}
			{\includegraphics[]{fh_logo_right_black.pdf}}
		}
	}
	\newgeometry{margin=20mm,bottom=25mm}
	\thispagestyle{empty}

	\vspace*{\fill}
	\centering
	\Huge
	\thetitle\par
	\normalsize
	Bachelorarbeit\par
	\vspace{0.25cm}
	\normalsize
	Fachhochschule Aachen\par
	Fachbereich Elektrotechnik und Informationstechnik\par
	Ingenieur-Informatik\par
	\vspace{0.5cm}
	Albert Kasdorf\par
	geb. am 29.12.1984 in Pawlodar\par
	Matr.-Nr.: 3029294\par
	%\vspace{0.5cm}
	%März -- Juli 2017\par
	%01.03.2017 bis 31.07.2017\par
	\vspace{0.5cm}
	Gutachter:\par
	Prof. Dr. rer. nat. Alexander Ferrein\par
	Prof. Dipl.-Inf. Ingrid Scholl\par
	Prof. Dr.-Ing. Thorsten Ringbeck\par
	\vspace{\fill}
\end{titlepage}
	\cleardoublepage
	
	% front matter
	%\input{abstract}

	\input{0_inhaltsverzeichnis}
	%
% How to add translation to long form of acronym?
% https://tex.stackexchange.com/questions/112222/how-to-add-translation-to-long-form-of-acronym
%
% Samples:
% - \acro{}[]{}
% - \acro{Kuerzel}[Kurzform]{Langform}
% Plural:
% - \acro{dr}[Dr.]{Doktor}
% - \acroplural{dr}[Dres.]{Doktoren}


\chapter*{Abkürzungsverzeichnis}
\begin{acronym}		
% A
	\acro{aoa}[AOA]{Angle of Arrival}
% B
% C
	\acro{cua}[CUA]{Catholic University of America}
% D
	\acro{darpa}[DARPA]{Defense Advanced Research Projects Agency}
	\acro{dod}[DoD]{Department of Defense}
% E
	\acro{ekf}[EKF]{Extended Kalman Filter}
% F
	\acro{fbw}[FBW]{Fractional Bandwidth}
	\acro{fcc}[FCC]{Federal Communications Commission}
	\acro{ftdi}[FTDI]{Future Technology Devices International}
% G
	\acro{gps}[GPS]{Global Positioning System}
	\acro{gnss}[GNSS]{Global Navigation Satellite System}
	\acro{gssi}[GSSI]{Geophysical Survey Systems Inc.}
	\acro{gpr}[GPR]{Ground Penetrating Radar}
	\acro{gpio}[GPIO]{General-purpose input/output}	% Allzweckeingabe/-ausgabe
% H
	\acro{hf}[HF]{Hochfrequenz}
% I
	\acro{ins}[INS]{Inertial Navigation System}
	\acro{ips}[IPS]{Indoor Positioning System}
	\acro{ic}[IC]{Integrated Circuit}	% integrierter Schaltkreis
	\acro{irq}[IRQ]{Interrupt Request}
% J
% K
% L
	\acro{los}[LoS]{Line of Sight}
	\acro{llnl}[LLNL]{Lawrence Livermore National Laboratory}
	\acro{lanl}[LANL]{Los Alamos National Laboratory}
	\acro{led}[LED]{Light-emitting diode} % Leuchtdiode
% M
	\acro{mems}[MEMS]{Micro Electro Mechanical Sensors}
	\acro{mir}[MIR]{Micropower Impulse Radar}
	\acro{mosi}[MOSI]{Master Output Slave Input}
	\acro{miso}[MISO]{Master Input Slave Output}
% N
	\acro{nlos}[NLos]{Non-line of Sight}
% O
	\acro{osd}[OSD]{Office of the Secretary of Defense}
% P
	\acro{pan}[PAN]{Personal Area Network}
% Q
% R
	\acro{rtls}[RTLS]{Real-Time Location System}
	\acro{rss}[RSS]{Received Signal Strength}
	\acro{rf}[RF]{Radio Frequency}
% S
% T
	\acro{tof}[ToF]{Time of Flight}
	\acro{toa}[ToA]{Time of Arrival}
	\acro{tdoa}[TDoA]{Time Difference of Arrival}
	\acro{twr}[TWR]{Two Way Ranging}
% U
	\acro{uwb}[UWB]{Ultra-Wideband} % Ultra WideBand (UWB),
	\acro{usaf}[USAF]{United States Air Force}
	\acro{usdod}[USDOD]{United States Department of Defense} % U.S. Department of Defense
	\acro{uart}[UART]{Universal Asynchronous Receiver Transmitter}
% V
% W
	\acro{wpan}[WPAN]{Wireless Personal Area Network}
% X
% Y
% Z

\end{acronym}
	\input{0_abbildungsverzeichnis}
	\input{0_tabellenverzeichnis}
	
	% Arabische Seitenummerierung
	\mainmatter
	
	\chapter{Einführung}

\section{Aufgabenstellung}

\section{Motivation}

\section{Zielsetzung}

\section{Gliederung}
	\chapter{Grundlagen}

\section{Verfahren für die Reichweiten-Bestimmung}

\section{Wahrscheinlichkeitstheorie}

\section{Bayes/Kalman/Partikel Filter}

\section{SLAM}

\section{ROS}
	%
% Forschungsstand:
%
% - Welche wissenschaftlichen Erkenntnisse liegen zu dem Thema bereits vor?
% - Grundsätzlich gibt es zwei Möglichkeiten, einen Forschungsstand zu schreiben: Entweder ordnen Sie Ihren Literaturüberblick nach Themenkomplexen oder Sie geben einen rein chronologischen Überblick über die wichtigsten Publikationen.
% - Auf keinen Fall sollten Sie den Forschungsstand zu voll packen. Es geht nicht darum, dem Leser zu zeigen, was Sie alles studiert haben (wie fleißig Sie waren), sondern um einen kompakten Überblick über die wichtigste Literatur.
% - Wichtig: Listen Sie die Literatur nicht nur auf, sondern erklären Sie, welchen Beitrag die jeweilige Publikation zum Erkenntnisgewinn geleistet hat. Also, zum Beispiel: Was hat der Autor als Erster erkannt oder hinterfragt? Es muss ja einen Grund geben, weshalb Sie die betreffende Publikation unter die Meilensteine reihen – und den sollten Sie dem Leser deutlich machen.
%
% Ferrein:
% - 4-5 Seiten in der Bachelorarbeit (deutlich umfangreicher als im Exposé)
% - Was unterscheidet meinen Ansatz von den anderen verhandenen Ansätzen?
%
\chapter{Stand der Forschung und Technik}

Einen guten Überblick über die Eigenschaften der Drahtlosen-Protokolle (engl. Wireless Protocols) Bluetooth, UWB, ZigBee und WiFi liefert die Arbeit \cite{lee2007comparative} von \citeauthor{lee2007comparative}.

In \cite{smith1987closed} wird das grundlegende Prinzip erklärt um aus mehreren bekannten Sensoren die Position eines beweglichen Empfängers zu berechnen.

Der theoretische Hintergrund des SLAM--Verfahrens wird in \cite{dissanayake2001solution} vorgestellt. Zusätzlich wird bewiesen das die Unsicherheit bei der Kartenerstellung und Lokalisierung eine untere Schranke erreicht.

\citeauthor{kantor2002preliminary} stellen in Ihrer Arbeit \cite{kantor2002preliminary} ein Lokalisierungsverfahren vor, welches die Roboterposition anhand von Entfernungsmessungen zu vorher bekannten Landmarken bestimmen kann. Im letzten Abschnitt wird SLAM--Verfahren vorgestellt, welches über einen Kalman--Filter die Unsicherheit der Landmarkenposition modellieren kann.

Die Autoren \citeauthor{blanco2008pure} gehen in ihren Arbeiten \cite{blanco2008pure, blanco2008efficient} einen Schritt weiter und bestimmen die unbekannte Roboterposition sowie die unbekannten Landmarkenpositionen. Hierzu nutzen Sie im ersten Schritt einen Partikelfilter (engl. Particle Filter) bis die Schätzung eine ausreichende Genauigkeit erreicht hat um dann im zweiten Schritt über einen Kalman--Filter ein Positionsverfolgung (engl. Position Tracking) durchzuführen.

Die Arbeit \cite{ledergerber2015robot} von \citeauthor{ledergerber2015robot} gehen auf die Roboterlokalisierung unter Verwendung einer One-Way Ultra-Wideband Kommunikation ein. Dieses hat den Vorteil, das mit sehr wenigen Landmarken eine große Anzahl von Roboter lokalisiert werden kann.

- The Cartesian EKF described above operates in the Cartesian space, we formulate our problem in polar coordinates.
- The use of this parameterization derives motivation from the polar coordinate system, where annuli, crescents and other ringlike shapes can be easily modeled. This parameterization is called Relative Over Parameterized (ROP) because it over parameterizes the state relative to an origin.

- EKF -> Polar EKF -> Multi-Hypothesis Filter
- Partikel Filter
	\chapter{Ultrabreitband}

\section{Historie}

\section{Alternative Technologien}

\section{Gegenüberstellung}

\section{Erstelle Hardware}

\subsection{Elektrischer Aufbau}

\subsection{Platinendesign}

\subsection{Steuersoftware}

\subsection{Entfernungsmessung und Auswertung}

\subsection{Kalibrierung}
	\input{ro_slam}
	\input{evaluation}
	\chapter{Zusammenfassung und Ausblick}

\section{Zusammenfassung}

\section{Ausblick}
	\cleardoublepage

%
% Literatur
%
% 	- http://linorg.usp.br/CTAN/macros/latex/contrib/biblatex/doc/biblatex.pdf
% 		- bibnumbered, bibintoc
%	- https://de.sharelatex.com/learn/Bibliography_management_with_biblatex
%
	\nocite{mcelroy2014comparison}
	\nocite{herranz2014comparison}
	\nocite{gonzalez2009mobile}
	\nocite{durrant2006simultaneous}
	\nocite{thrun2005probabilistic}
	\nocite{schroeder2005low}
	\nocite{smith2004tracking}
	\printbibliography[
		heading=bibintoc,
		title={Literaturverzeichnis}
	]
	\cleardoublepage

	% Beginn des Anhangs
	\appendix

	% Ende des Buches
	\backmatter

\end{document}