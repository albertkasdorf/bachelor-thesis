%
% Learn LaTeX in 30 minutes
% https://de.sharelatex.com/learn/Learn_LaTeX_in_30_minutes
%


%
% LaTeX documentclass options illustrated
% https://texblog.org/2013/02/13/latex-documentclass-options-illustrated/
%
\documentclass[
	%draft,
	11pt,
	a4paper,
	twoside,
	titlepage,
	openright,
%	BCOR = 8mm,
%	DIV = calc,
	ngerman,
]{scrbook}

%\usepackage[utf8]{inputenc}		% LuaTex do not need this
%\usepackage[T1]{fontenc}				% LuaTex do not need this
\usepackage{fontspec}					% LuaTex need this
\usepackage{lmodern}
\usepackage{comment}
%\usepackage[ngerman]{babel}
\usepackage{polyglossia}
	\setdefaultlanguage[spelling=new]{german}
\usepackage[autostyle]{csquotes}
%
% Layout I: Seitengröße und ‑ränder
% 	- https://tobiw.de/en/teotm/layout-1
% Diplomarbeit Koma 3 scrbook und layout Koma oder Geometry? Hilfe
%	- https://komascript.de/node/1047
% LaTeX: individuelle Kopf- und Fußzeilen
% 	- https://esc-now.de/_/latex-individuelle-kopf--und-fusszeilen
%
%\usepackage{showframe}
\usepackage[
	a4paper,
	top=25mm,
	bottom=25mm,
	left=30mm,
	right=25mm,
%	showframe,
]{geometry}
%
%
%
\usepackage[
%	singlespacing
	onehalfspacing
% doublespacing
]{setspace}
%
% fix widows (Hurenkinder) and orphans (Schusterjungen)
% http://mirror.hmc.edu/ctan/macros/latex/contrib/nowidow/nowidow.pdf
%
%\usepackage[defaultlines=2,all]{nowidow}
%
% use color package and define fh logo colors
%
%\usepackage{color}
%	\definecolor{fh-black}{RGB}{0,0,0}
%	\definecolor{fh-white}{RGB}{255,255,255}
%	\definecolor{fh-turquoise}{RGB}{0,177,172}
%
% Quelltexte (Programmlistings)
% Listings caption [duplicate]
% 		- https://tex.stackexchange.com/questions/54819/listings-caption
%
%\usepackage{listings}
%  \renewcommand{\lstlistingname}{Auflistung}
%
% http://ctan.math.utah.edu/ctan/tex-archive/macros/latex/contrib/minted/minted.pdf
%
\usepackage[chapter]{minted}
	\usemintedstyle{vs}
	\setminted{frame=single, breaklines}
	\renewcommand{\listingscaption}{Auflistung}
	%\renewcommand{\listoflistingscaption}{List of Program Code}
	\providecommand*{\listingautorefname}{Auflistung}
%
% Wie kann ich hyperlinks mit \url umbrechen lassen?
% http://texwelt.de/wissen/fragen/5361/wie-kann-ich-hyperlinks-mit-url-umbrechen-lassen
% Mit der Verwendung der Option hyphens wird der Umbruch nach einem Bindestrich aktiviert.
%
\usepackage[hyphens]{url}
%
% https://www.dante.de/events/Archiv/dante2012/Programm/Vortraege/vortrag-ferber.pdf
%
\usepackage[
	breaklinks,
	colorlinks,
	linkcolor=black,
	urlcolor=black,
	citecolor=black,
	pdfencoding=auto,	
]{hyperref}
	\hypersetup
	{
		pdftitle    = {Range-Only Simultaneous Localization and Mapping mittels Ultra-Wideband},
		pdfsubject  = {Range-Only Simultaneous Localization and Mapping mittels Ultra-Wideband},
		pdfauthor   = {Albert Kasdorf},
		pdfkeywords = {Range-Only Simultaneous Localization and Mapping, RO-SLAM, Ultra-Wideband, UWB},
		pdfcreator  = {LuaTex},
		pdfproducer = {LaTeX with hyperref}
	}
%
% bibtex vs. biber and biblatex vs. natbib
%	- https://tex.stackexchange.com/questions/25701/bibtex-vs-biber-and-biblatex-vs-natbib
% The biblatex Package - Programmable Bibliographies and Citations
% 	- http://ftp.math.purdue.edu/mirrors/ctan.org/macros/latex/exptl/biblatex/doc/biblatex.pdf
% Biblatex citation order
%	- https://tex.stackexchange.com/questions/51434/biblatex-citation-order
% 		- sorting=ydnt: Sort by year (descending), name, title.
%		- sorting=none: Do not sort at all. All entries are processed in citation order.
%
\usepackage[backend=biber,style=numeric-comp,sorting=none]{biblatex}
	\addbibresource{quellen.bib}
	\defbibfilter{papersonly}{not type=online and not type=manual}
%
% How can I use @author, @date, and @title after maketitle?
% https://tex.stackexchange.com/questions/27710/how-can-i-use-author-date-and-title-after-maketitle
%
\usepackage{titling}
	\title{Range-Only Simultaneous Localization and Mapping mittels Ultra-Wideband}
	\author{Albert Kasdorf}
%
% siunitx — A comprehensive (SI) units package
% http://www.bakoma-tex.com/doc/latex/siunitx/siunitx.pdf
% Dezibel Milliwatt: dBm
%	- https://de.wikipedia.org/wiki/Leistungspegel
%
\usepackage[mode=text]{siunitx}
	\sisetup{locale=DE, range-phrase=--, product-units=single, binary-units=true}
	\DeclareSIUnit{\mAh}{mAh}
	\DeclareSIUnit{\belmilliwatt}{Bm}
	\DeclareSIUnit{\dBm}{\deci\belmilliwatt}
\usepackage{graphicx}
  \graphicspath{ {bilder/} }
\usepackage{eso-pic}
%\usepackage{tikz}
%	\usetikzlibrary{positioning}
%	\usetikzlibrary{calc}
%	\usetikzlibrary{decorations.markings}
%	\usetikzlibrary{fit}
\usepackage{float}
%
%
%
\usepackage[headsepline]{scrlayer-scrpage}
	\clearpairofpagestyles
	\chead{\headmark}
	%\cfoot*{\pagemark}
	\ohead{\pagemark}
%	\ofoot{\pagemark}
	\setkomafont{pageheadfoot}{\sffamily}
	\setkomafont{pagination}{}
	\setkomafont{chapter}{\Huge\bfseries}
	\renewcommand*{\chapterformat}{%
  		\enskip\mbox{\scalebox{3}{\thechapter\autodot}}}
	\renewcommand\chapterlinesformat[3]{%
		\parbox[b]{\textwidth}{\hrulefill#2}\par%
		#3\par\bigskip
		\hrule}
%	\renewcommand\chapterlinesformat[3]{%
%  		\begin{center}#2\end{center}\par%
%  		\begin{center}#3\end{center}\par\bigskip}
%
% List of Figures, List of tables in toc
% https://tex.stackexchange.com/questions/299306/list-of-figures-list-of-tables-in-toc
%
\usepackage[nottoc]{tocbibind}
\usepackage[titletoc,toc,page]{appendix}
	\renewcommand{\appendixname}{Anhang}
	\renewcommand{\appendixtocname}{Anhang}
	\renewcommand{\appendixpagename}{Anhang}
%
% Generating dummy text/blindtext with LaTeX for testing
% https://texblog.org/2011/02/26/generating-dummy-textblindtext-with-latex-for-testing/
%
\usepackage{blindtext}
%
% Add source to figure caption
% https://tex.stackexchange.com/questions/95029/add-source-to-figure-caption
%
\usepackage{caption}
	\newcommand{\source}[1]{\caption*{Quelle: {#1}} }
	%\newcommand{\source}[1]{\caption*{\hfill Quelle: {#1}} }
	%\newcommand{\source}[1]{\vspace{-3pt} \caption*{ Quelle: {#1}} }
%
%
%
\usepackage{wrapfig}
%
%
%
\usepackage[bottom]{footmisc}
%
%
%
\usepackage{subcaption}
%
%
%
\usepackage[toc,acronyms,nonumberlist,nopostdot]{glossaries}
	\makeglossaries
	\loadglsentries{0_glossar}
%
% Euro Symbol
%
\usepackage{eurosym}
%
% Combine rows or columns in a tabular.
%
\usepackage{multirow}

%%%%%%%%%%%%%%%%%%%%%%%%%%%%%%%%%%%%%%%%%%%%%%%%%%%%%%%%%%%%%%%%%%%%%%%%%%%%%%%%
%
% https://www.rpi.edu/dept/arc/training/latex/LaTeX_symbols.pdf
%
\usepackage{wasysym}

%%%%%%%%%%%%%%%%%%%%%%%%%%%%%%%%%%%%%%%%%%%%%%%%%%%%%%%%%%%%%%%%%%%%%%%%%%%%%%%%
%
% Drehen einer Tabelle über die sidewaystable Umgebung.
%
\usepackage{rotating}

%%%%%%%%%%%%%%%%%%%%%%%%%%%%%%%%%%%%%%%%%%%%%%%%%%%%%%%%%%%%%%%%%%%%%%%%%%%%%%%%
%
% LATEX Mathematical Symbols
% https://reu.dimacs.rutgers.edu/Symbols.pdf
% https://en.wikibooks.org/wiki/LaTeX/Mathematics
%
\usepackage{amssymb}

%%%%%%%%%%%%%%%%%%%%%%%%%%%%%%%%%%%%%%%%%%%%%%%%%%%%%%%%%%%%%%%%%%%%%%%%%%%%%%%%
%
% http://osl.ugr.es/CTAN/macros/latex/contrib/draftwatermark/draftwatermark.pdf
%
\usepackage[
	nostamp,
	firstpage
]{draftwatermark}
	\SetWatermarkColor{red}
	%\SetWatermarkColor[rgb]{0,1,0}
	\SetWatermarkText{CONFIDENTIAL}
	\SetWatermarkScale{0.75}

%%%%%%%%%%%%%%%%%%%%%%%%%%%%%%%%%%%%%%%%%%%%%%%%%%%%%%%%%%%%%%%%%%%%%%%%%%%%%%%%
%
% algorithm2e.sty — package for algorithms
% http://ctan.mirrors.hoobly.com/macros/latex/contrib/algorithm2e/doc/algorithm2e.pdf
%
\usepackage[
	ngerman,
	linesnumbered,
	boxed,
	algochapter,
%	rightnl,
%	figure,
]{algorithm2e}
	% Then you can adjust the spacing between the body of the algorithm and its
	% caption through the command \SetAlCapSkip.
	\SetAlCapSkip{1em}
	% Restyling the caption in an algorithm created with algorithm2e
	% https://tex.stackexchange.com/questions/112294/restyling-the-caption-in-an-algorithm-created-with-algorithm2e/112295
	\SetAlgoCaptionSeparator{:}
	\renewcommand\AlCapFnt{\normalfont}
	% Algorithm2e modify line numbers
	% https://tex.stackexchange.com/questions/100145/algorithm2e-modify-line-numbers
	\SetNlSty{textbf}{}{:}
	% Sets the value of the space between the line numbers and the text, by default 1em.
	\SetNlSkip{2em}
	%\SetAlgoRefName{QXY}

%%%%%%%%%%%%%%%%%%%%%%%%%%%%%%%%%%%%%%%%%%%%%%%%%%%%%%%%%%%%%%%%%%%%%%%%%%%%%%%%
%
%
%
\usepackage{xparse}

%%%%%%%%%%%%%%%%%%%%%%%%%%%%%%%%%%%%%%%%%%%%%%%%%%%%%%%%%%%%%%%%%%%%%%%%%%%%%%%%
%
% Some packages from other authors may have problems with KOMA-Script.
% Package scrhack contains all those improvement proposals for
% other packages. This means, scrhack redefines macros of packages
% from other authors! The redefinitions are only activated, if
% those packages were loaded. Users may prevent scrhack from
% redefining macros of individual packages.
%
\usepackage{scrhack}



%%%%%%%%%%%%%%%%%%%%%%%%%%%%%%%%%%%%%%%%%%%%%%%%%%%%%%%%%%%%%%%%%%%%%%%%%%%%%%%%
%
% Befehlserweiterungen
%	- https://www.texdev.net/2010/05/23/from-newcommand-to-newdocumentcommand/
%	- https://tex.stackexchange.com/questions/67745/remove-citation-from-list-of-figures
%
\NewDocumentCommand\captionwithcite{mm}{\caption[#1]{#1 #2}}



%
% VIII Beispiel für den Aufbau einer Bachelorarbeit
% https://www.tu-chemnitz.de/hsw/psychologie/professuren/owpsy/Service/Leitfaden-Bachelorarbeiten.pdf
%
\begin{document}

	\frontmatter
 
	%
% http://tobiw.de/tbdm/titelseiten
%
\begin{titlepage}
	\newlength
	\logoheight
	\settoheight
	\logoheight{\includegraphics[]{fh_logo_right_black.pdf}}
	\AddToShipoutPicture*{
		\makebox[\paperwidth][r]{
			\raisebox{\dimexpr(\paperheight-\height-20mm)}
			{\includegraphics[]{fh_logo_right_black.pdf}}
		}
	}
	\newgeometry{margin=20mm,bottom=25mm}
	\thispagestyle{empty}

	\vspace*{\fill}
	\centering
	\Huge
	\thetitle\par
	\normalsize
	Bachelorarbeit\par
	\vspace{0.25cm}
	\normalsize
	Fachhochschule Aachen\par
	Fachbereich Elektrotechnik und Informationstechnik\par
	Ingenieur-Informatik\par
	\vspace{0.5cm}
	Albert Kasdorf\par
	geb. am 29.12.1984 in Pawlodar\par
	Matr.-Nr.: 3029294\par
	%\vspace{0.5cm}
	%März -- Juli 2017\par
	%01.03.2017 bis 31.07.2017\par
	\vspace{0.5cm}
	Gutachter:\par
	Prof. Dr. rer. nat. Alexander Ferrein\par
	Prof. Dipl.-Inf. Ingrid Scholl\par
	Prof. Dr.-Ing. Thorsten Ringbeck\par
	\vspace{\fill}
\end{titlepage}
	\cleardoublepage
	
	% front matter
	%\input{abstract}

	%
% Einführung in LATEX
% http://www.it-designers-gruppe.de/uploads/media/20130911_LaTeXFerienkurs2013_Skript.pdf
%

% Ein Kapitel das nicht mitgezählt wird beginnen
\chapter*{Eidesstattliche Erklärung}

% Diese Seite soll keine Kopf-/Fußzeile enthalten
\thispagestyle{empty}

%Hiermit erkläre ich, dass ich die vorliegende Arbeit selbstständig angefertigt habe. Es wurden nur die in der Arbeit ausdrücklich benannten Quellen und Hilfsmittel benutzt. Wörtlich oder sinngemäß übernommenes Gedankengut habe ich als solches kenntlich gemacht.

Ich versichere hiermit, dass ich die vorliegende Arbeit selbständig verfasst und keine anderen als die im Literaturverzeichnis angegebenen Quellen benutzt habe.

Stellen, die wörtlich oder sinngemäß aus veröffentlichten oder noch nicht veröffentlichten Quellen entnommen sind, sind als solche kenntlich gemacht.

Die Zeichnungen oder Abbildungen in dieser Arbeit sind von mir selbst erstellt worden oder mit einem entsprechenden Quellennachweis versehen.

Diese Arbeit ist in gleicher oder ähnlicher Form noch bei keiner anderen Prüfungsbehörde eingereicht worden.

% Etwas Abstand setzen
\vspace{1cm}

% Bereich für Datum, Name und Unterschrift
\begin{center}
\begin{tabular}[h]{lp{2cm}p{5.5cm}}
& & \\
\cline{1-1}\cline{3-3}
Ort, Datum& & Albert Kasdorf\\
\end{tabular}
\end{center}
	%\setcounter{tocdepth}{4}
	\tableofcontents
	
	%\printglossary[title=Glossar,toctitle=Glossar]
	%\printglossary[
	%	type=\acronymtype,
	%	title=Abkürzungsverzeichnis,
	%	toctitle=Abkürzungsverzeichnis]
	%\printglossary[type=symbols]
	
	\listoffigures
	\listoftables
	
	\mainmatter
	
	\chapter{Einführung}

Mit dem vorliegenden Artikel sollen die Einsatzmöglichkeiten der seriellen Kommunikation mit Peripheriegeräten mittels \ac{spi} verdeutlicht werden.

Das \ac{spi} ist ein in den frühen 1980er Jahren von Motorola entwickeltes Bus-System mit einem „lockeren“ Standard für einen synchronen seriellen Datenbus (Synchronous Serial Port), mit dem digitale Schaltungen nach dem Master-Slave-Prinzip miteinander verbunden werden können.

\begin{figure}[h]
    \centering
    \includegraphics[width=0.25\textwidth]{surface3d_demo4}
    \caption{a nice plot}
    \label{fig:mesh1}
\end{figure}
 
As you can see in the figure \ref{fig:mesh1}, the function grows near 0. Also, in the page \pageref{fig:mesh1} is the same example.

The table \ref{table:1} is an example of referenced \LaTeX elements.
 
\begin{table}[h!]
\centering
\begin{tabular}{||c c c c||} 
 \hline
 Col1 & Col2 & Col2 & Col3 \\ [0.5ex] 
 \hline\hline
 1 & 6 & 87837 & 787 \\ 
 2 & 7 & 78 & 5415 \\
 3 & 545 & 778 & 7507 \\
 4 & 545 & 18744 & 7560 \\
 5 & 88 & 788 & 6344 \\ [1ex] 
 \hline
\end{tabular}
\caption{Table to test captions and labels}
\label{table:1}
\end{table}

\section{Aufgabenstellung}

\section{Motivation}

\section{Zielsetzung}

\section{Gliederung}

\section{???Problemstellung??}

In der Zeit vor den Navigationsgeräten wurden auf deutschen Straßen noch regelmäßig faltbare Straßenkarten von den Beifahrern verwendet um den Fahrer den Weg zu weisen. Bevor eine Straßenkarten verwendet werden kann, muss diese Erstellt werden. Dieser Prozess ist unter dem Begriff Kartenerstellung (engl. Mapping) bekannt. Der Detailgrad hängt dabei stark vom Verwendungszweck ab. Der erste Schritt nach dem entfalten der Straßenkarten bestand in der Lokalisierung (engl. Localization), also der Bestimmung der ungefähren Fahrzeugposition und dem Ziel der Reise auf der Straßenkarte. Darauf aufbauend wurde vom Beifahrer dann eine Route zwischen der aktuellen Fahrzeugposition und dem Ziel geplant und während der Fahrt weiter verfolgt, was auch als Pfad-Planung (engl. Path-Planning) bekannt ist.

Genauso wie der menschliche Agent muss auch jeder mobile Roboter für sich diese grundlegende Frage beantworten können. \glqq Wo bin ich?\grqq{}, \glqq Wo bin ich bereits gewesen?\grqq, \glqq Wohin gehe ich?\grqq{} und \glqq Welcher ist der beste Weg dahin?\grqq{}\cite{murphy2000introduction}.

Außerhalb von geschlossenen Räumlichkeiten (engl. Outdoor) erfolgt die Lokalisierung in der Regel mittels GPS, unter der Voraussetzung das eine ungehinderte Verbindung zu den GPS-Satelliten möglich ist. Die Lokalisierung ist in diesem Fall sehr einfach, da die GPS Koordinaten eindeutig sind und das Kartenmaterial bereits im gleichen Koordinatensystem vorliegt.

Innerhalb geschlossener Räumlichkeiten (engl. Indoor), wie in öffentlichen Gebäuden, Logistikhallen oder auch in Bergwerken, ist eine Lokalisierung mittels GPS nicht mehr möglich. Erschwerend kommt dazu, dass es in der Regel zu diesen Räumlichkeiten keine öffentlich verfügbaren Karten gibt oder diese sich wie im letzten Beispiel häufig ändern. Aus diesem Problemfeld haben sich Algorithmen für die Simultane Lokalisierung und Kartenerstellen (engl. Simultaneous Localization and Mapping (SLAM)) entwickelt.

Häufig werden SLAM Algorithmen verwendet um aus Kamerabildern oder \SI{360}{\degree} Abstandsmessungen eine Karte der Umgebung zu erstellen und sich in der gleichen zu lokalisieren. Der Fokus dieser Arbeit liegt jedoch auf den reinen Entfernungsbasieren SLAM (engl. Range Only SLAM (RO--SLAM)) Algorithmen. Hierbei werden nur die Informationen der Eigenbewegung und die Entfernungen zu mehreren, vorher unbekannten, Basisstationen genutzt um sich selbst zu Lokalisieren und eine Karte mit den Positionen der Basisstationen zu erstellen.
	%%%%%%%%%%%%%%%%%%%%%%%%%%%%%%%%%%%%%%%%%%%%%%%%%%%%%%%%%%%%%%%%%%%%%%%%%%%%%%%%
%
%	- Introduction to Mobile Robotics - SS 2017
%		- http://ais.informatik.uni-freiburg.de/teaching/ss17/robotics/
%
%	- Robot Mapping - WS 2016/17 - Uni-Freiburg
%		- http://ais.informatik.uni-freiburg.de/teaching/ws16/mapping/
%		- https://www.youtube.com/playlist?list=PLgnQpQtFTOGQrZ4O5QzbIHgl3b1JHimN_&feature=g-list
%	
% 	- Probabilistic Robotics
%		- http://robots.stanford.edu/probabilistic-robotics/
%
%%%%%%%%%%
\chapter{Grundlagen}


%%%%%%%%%%%%%%%%%%%%%%%%%%%%%%%%%%%%%%%%%%%%%%%%%%%%%%%%%%%%%%%%%%%%%%%%%%%%%%%%
%
% https://de.wikipedia.org/wiki/Fading_(Elektrotechnik)
%
%%%%%%%%%%
\section{Verfahren für die Entfernungsbestimmung}

Bei der \textit{Triangulation} werden die Winkel zwischen mehreren Referenzpunkten bestimmt und dann die dazu gehörige Entfernung mittels trigonometrischer Funktionen berechnet. Dieses Verfahren ist auch unter den Namen \Gls{aoa} bzw. \Gls{doa} bekannt. Um eine genau Ortsbestimmung durchzuführen müssen die Winkel sehr genau bestimmt werden. Um das zu bewerkstelligen werden im Empfänger mehrere Antennen zu einem Feld (engl. Antenna Array) zusammengefasst. Jedoch ist diese Konstruktion sehr teuer und empfindlich für Mehrwegeempfang (engl. Multipath) bzw. Signalabschattungen. \cite{gezici2005localization, liu2007survey, decawave2014rtls}

Im Gegensatz dazu werden bei der \textit{Trilateration} die Entfernungen zwischen mehreren Referenzpunkten betrachtet. Es werden dabei die Verfahren \Gls{toa} und \Gls{tdoa} unterschieden.
Bei dem \Gls{toa}-Verfahren wird zuerst die Zeitdifferenz zwischen dem Senden und Empfangen eines Funksignals berechnet. Mittels der Signallaufzeit (engl. \acrfull{tof}) und der Ausbreitungsgeschwindigkeit des Funksignals kann die Entfernung berechnet werden. Die Ortsbestimmung erfolgt dann über die Schnittpunkte von drei Kreisen (2D) bzw. vier Kugeln (3D) miteinander. Um dieses Verfahren anwenden zu können, ist es erforderlich das das Funksignal mit einem Zeitstempel des Startzeitpunktes versehen ist. Daraus folgt aber auch, das die Zeit zwischen Sender und Empfänger sehr genau synchronisiert werden müssen um den Fehler möglich klein zu halten.
Bei dem \Gls{tdoa}-Verfahren werden die Zeitdifferenz zwischen dem Empfang des Funksignals an mehreren Empfängern ausgewertet. Dies hat den großen Vorteil das nur noch die Zeit zwischen den Empfängern synchronisiert werden muss. \cite{zekavat2011handbook, decawave2014rtls}

Neben der \textit{Triangulation} und \textit{Trilateration} besteht auch die Möglichkeit auf Grund der empfangenen Signalstärke (engl. \acrfull{ss_radio}, \acrfull{rss} oder auch \acrfull{rssi}) Rückschlüsse über die Entfernung zu ziehen. Dazu muss die ursprüngliche Signalstärke und die Ausbreitungscharakteristik der elektromagnetischen Welle in der spezifischen Umgebung bekannt sein. \cite{gezici2005localization, decawave2014rtls}

In den nächsten zwei Abschnitten werden die \textit{DecaWave} Entfernungsmessverfahren vorgestellt. Diese haben den Vorteil, dass keine Synchronisierung der Zeit zwischen Sender und Empfänger notwendig ist. Weiterhin besitzen die \textit{DecaWave} \Gls{uwbt} zwei Eigenschaften die die Entfernungsmessung ideal ergänzt. Zum einen wird jede erhaltene Nachricht mit einem lokalen Zeitstempel versehen, der über eine minimale Auflösung von ungefähr \SI{15.65}{\pico\second} verfügt. Hiermit wäre eine theoretische Ortsauflösung von ungefähr \SI{5}{\milli\metre} möglich. Des Weiteren ist es möglich, den Sendezeitpunkt einer Nachricht in die Zukunft zu legen. Damit lässt sich die Zeitspanne zwischen dem Empfang und der Antwort auf eine Nachricht im Voraus berechnen. Die Zeitspanne kann dann der Antwortnachricht als Nutzlast mitgegeben werden, um beim Empfänger die Umlaufzeit zu berechnen.


%%%%%%%%%%%%%%%%%%%%%%%%%%%%%%%%%%%%%%%%%%%%%%%%%%%%%%%%%%%%%%%%%%%%%%%%%%%%%%%%
%
% 	- Wie lange dauert es bis eine Nachricht ausgetauscht worden ist?
%		- Beispiel mit einer konkreten Entfernung?
%	- Wie schnell drifted ein Quarz in einem µc?
%		- What is the ppm in the crystal oscillator?
%		- https://electronics.stackexchange.com/questions/15851/what-is-the-ppm-in-the-crystal-oscillator
%		- In the 1930s, such precise time measurements simply weren't possible; a clock of the required accuracy was difficult enough to build in fixed form, let alone portable. A crystal oscillator, for instance, drifts about 1 to 2 seconds in a month, or 1.4x10−3 seconds an hour.[1] This may sound small, but as light travels 3x108 m/s, this represents a drift of 400 m per hour. Only a few hours of flight time would render such a system unusable, a situation that remained in force until the introduction of commercial atomic clocks in the 1960s.
%	- Clock accuracy in ppm
%	- http://www.best-microcontroller-projects.com/ppm.html
%
%%%%%%%%%%
\subsection{Single-sided Two-way Ranging}

Das einfachste Verfahren, um aus der Umlaufzeit die Entfernung abzuschätzen, ist das \Gls{sstwr}-Verfahren. Dabei sendet der \Gls{tag} eine Nachricht an den \Gls{anchor} und wartet ab, bis eine entsprechende Antwortnachricht eintrifft. Beide Module erhalten einen Zeitstempel für den Versand und Empfang von Nachrichten. Aus diesen kann dann die Antwort- ($T_{reply}$) und Umlaufzeit ($T_{round}$) berechnet werden, siehe \autoref{fig:single_sided_two_way_ranging}. \cite{decawave2015twr, decawave2016dw1kusermanual}

%\begin{figure}[ht]
\begin{figure}
	\centering
	\includegraphics[width=0.75\linewidth]{single_sided_two_way_ranging}
	\captionwithcite{Ablauf des \acrlong{sstwr}.}{\cite{decawave2016dw1kusermanual}}
	\label{fig:single_sided_two_way_ranging}
\end{figure}

Die ungefähre \Gls{tof}-Zeitspanne ergibt sich aus der folgenden Gleichung:

\begin{equation}
T_{prop}=\frac{1}{2}\left(T_{round}-T_{reply}\right)
\end{equation}

Damit der \Gls{tag} die \Gls{tof}-Zeitspanne berechnen kann, benötigt er die $T_{reply}$-Zeitspanne. Zu diesem Problem gibt es mehrere Lösungen. Die einfachste Lösung ist eine feste Antwortzeit die jedem Modul bekannt ist. Alternativ kann der \Gls{anchor} in der Antwortnachricht seine individuelle Antwortzeit übermitteln. Oder der \Gls{tag} übermittelt dem \Gls{anchor} mit der initialen Nachricht wie lange der \Gls{anchor} warten muss bis er die Antwortnachricht verschickt. Hierbei muss die Zeitspanne groß genug gewählt sein, um den \Gls{anchor} die Möglichkeit zur Antwort zu lassen. Je nach Anforderung wird eine der vorherigen Methoden verwendet.

Der Nachteil bei diesem Verfahren besteht in dem Fehler der von der Antwortzeit abhängt. Bei Module verwenden zur Berechnung von $T_{round}$ und $T_{reply}$ ihre lokalen Zeitgeber. Beide Zeitgeber haben einen Offsetfehler $e_{A}$ und $e_{B}$ der von der Nennfrequenz abweicht. Die daraus abgeleitete \Gls{tof}-Zeitschätzung hat damit einen Fehler der mit der Antwortzeit wächst:

\begin{equation}
error\approx\frac{1}{2}\left(e_B-e_A\right)\times T_{reply}
\end{equation}


%%%%%%%%%%%%%%%%%%%%%%%%%%%%%%%%%%%%%%%%%%%%%%%%%%%%%%%%%%%%%%%%%%%%%%%%%%%%%%%%
%
% \cite{decawave2016dw1kusermanual}
%	- Where the clock in device A runs at ka times the desired frequency and the clock in device B runs at kb times the desired frequency and both ka & kb are close to 1.
%	- To give some idea of the size of this error, if devices A and B have clocks where each are 20 ppm away (the worst case specification) from the nominal clock in directions which make their combined error additive and equal to 40 ppm, then ka and kb might both be 0.99998 or 1.00002.
%	- Even with a relatively large UWB operating range of say 100 m, the TOF is just 333 ns, so the error is 20 × 10-6 × 333 × 10-9 seconds, which is 6.7 × 10-12 seconds or 6.7 picoseconds which is approximately 2.2 mm.
%
%%%%%%%%%%
\subsection{Double-sided Two-way Ranging}
\label{subsec:double_sided_two_way_ranging}

Das \Gls{dstwr}-Verfahren stellt eine Verbesserung gegenüber dem \Gls{sstwr}-Verfahren dar. Hierbei werden nun drei Nachrichten verwendet um jeweils die Umlaufzeiten zwischen \Gls{tag} und \Gls{anchor}, und \Gls{anchor} und \Gls{tag} zu berechnen, siehe \autoref{fig:double_sided_two_way_ranging}. Wenn die Umlaufzeit beim \Gls{tag} berechnet werden soll, müssen die Zeitspannen $T_{reply1}$ und $T_{round2}$ zum \Gls{tag} übermittelt werden. Für die letzte Zeitspanne erfolgt das mit einer vierten Nachricht die in dem Schaubild nicht abgebildet ist. \cite{decawave2015twr, decawave2016dw1kusermanual}

%\begin{figure}[ht]
\begin{figure}
	\centering
%	\includegraphics[width=0.7\linewidth]{double_sided_two_way_ranging}
	\includegraphics[width=0.8\linewidth]{double_sided_two_way_ranging}
	\captionwithcite{Ablauf des \acrlong{dstwr}.}{\cite{decawave2016dw1kusermanual}} 
	\label{fig:double_sided_two_way_ranging}
\end{figure}

Die ungefähre \Gls{tof}-Zeitspanne ergibt sich aus der folgenden Gleichung:

\begin{equation}
T_{prop} = \frac{\left(T_{round1}\times T_{round2}-T_{reply1}\times T_{reply2}\right)}{\left(T_{round1}+T_{round2}+T_{reply1}+T_{reply2}\right)}
\end{equation}

Der Fehler bei diesem Verfahren ergibt sich aus der folgenden Gleichung:

\begin{equation}
error=T_{prop}\times\left(1-\frac{k_a+k_b}{2}\right)
\end{equation}

Die Variablen $k_a$ und $k_b$ entsprechen hierbei den Offsetfehler der Zeitgeber von der Nennfrequenz und liegen beide sehr nahe bei eins.

Mit diesem Verfahren ist es auch möglich, gleichzeitig die Entfernungen zu mehr als einem \Gls{anchor} zu bestimmen, siehe \autoref{fig:double_sided_two_way_ranging_with_three_anchor}. Hierbei weist der \Gls{tag} in der initialen Nachricht jedem \Gls{anchor} eine individuelle Antwortzeit zu. Danach wartet er bist alle Antwortnachrichten angekommen sind um im der letzten Nachricht, dann jedem \Gls{anchor} seine Umlauf- und Antwortzeiten zu übermitteln. Jeder \Gls{anchor} kann nun individuell die \Gls{tof}-Zeitspanne für sich berechnen und dann dem \Gls{tag} übermitteln. Diese letzte Nachricht ist auf der \autoref{fig:double_sided_two_way_ranging_with_three_anchor} nicht aufgeführt.

%\begin{figure}[ht]
\begin{figure}
	\centering
	\includegraphics[width=\linewidth]{double_sided_two_way_ranging_with_three_anchor}
	\captionwithcite{Ablauf des \acrlong{dstwr} mit einem \Gls{tag} und drei \Gls{anchor}.}{\cite{decawave2016dw1kusermanual}}
	\label{fig:double_sided_two_way_ranging_with_three_anchor}
\end{figure}

In der \autoref{fig:double_sided_two_way_ranging_with_three_anchor} wurde in der initialen Nachricht jedem \Gls{anchor} eine individuelle Antwortzeit zugeordnet. Woher wusste der\Gls{tag} welche \Gls{anchor} vorhanden sind? Die Kommunikation zwischen \Gls{tag} und \Gls{anchor} kann in zwei Phasen unterteilt werden, siehe \autoref{fig:discovery_and_ranging_phase}. In der \textit{Discovery} Phase schickt der \Gls{tag} periodisch \textit{Blink} Nachrichten mit seiner Identifikationsnummer an alle Module in der Umgebung. Empfängt ein \Gls{anchor} diese Nachricht, nimmt er den \Gls{tag} in seiner Liste auf und übermittelt dem \Gls{tag} eine \textit{Ranging Init} Nachricht mit seiner Identifikationsnummer. Das \Gls{tag} seinerseits nimmt den \Gls{anchor} in seiner Liste auf und weiß nun auch welche \Gls{anchor} er in der nächsten Entfernungsmessung berücksichtigen muss. Die \textit{Ranging} Phase entspricht hierbei der bereits zuvor besprochenen Entfernungsmessung.

%\begin{figure}[ht]
\begin{figure}
	\centering
%	\includegraphics[width=0.4\linewidth]{discovery_and_ranging_phase2}
	\includegraphics[width=0.3\linewidth]{discovery_and_ranging_phase2}
	\captionwithcite{Ablauf der \textit{Discovery} und \textit{Range Phase} zwischen einem \Gls{tag} und mehreren \Gls{anchor}n.}{\cite{decawave2016dw1kusermanual}}
	\label{fig:discovery_and_ranging_phase}
\end{figure}


%%%%%%%%%%%%%%%%%%%%%%%%%%%%%%%%%%%%%%%%%%%%%%%%%%%%%%%%%%%%%%%%%%%%%%%%%%%%%%%%
%
%
%
%%%%%%%%%%
\section{Geometrie}


%%%%%%%%%%%%%%%%%%%%%%%%%%%%%%%%%%%%%%%%%%%%%%%%%%%%%%%%%%%%%%%%%%%%%%%%%%%%%%%%
%
%
%
%%%%%%%%%%
\subsection{Gleichseitiges Dreieck}

Ein gleichseitiges Dreieck zeichnet sich dadurch aus, dass alle drei Seiten gleich lang sind und jeder Innenwinkel einen Wert von \SI{60}{\degree} besitzt, siehe \autoref{fig:wikipedia_gleichseitiges_dreieck}. Ist der Umkreisradius $r_u$ eines gleichseitigen Dreiecks bekannt, so kann mit der \autoref{eq:dreieck_seitenlaenge_aus_umkreis} auch die Seitenlänge $a$ berechnet werden.

\begin{equation}
a = \frac{3}{\sqrt{3}} r_u \label{eq:dreieck_seitenlaenge_aus_umkreis}
\end{equation}


%%%%%%%%%%%%%%%%%%%%%%%%%%%%%%%%%%%%%%%%%%%%%%%%%%%%%%%%%%%%%%%%%%%%%%%%%%%%%%%%
%
%
%
%%%%%%%%%%
\subsection{Regelmäßiges Fünfeck}

Das regelmäßige Fünfeck zeichnet sich dadurch aus, das alle fünf äußeren Seiten gleich lang sind und jeder Innenwinkel einen von Wert von \SI{108}{\degree} besitzt, siehe \autoref{fig:wikipedia_regelmaessiges_fuenfeck}. Aus der Länge einer äußeren Seite $a$ lässt sich zum einen der Umkreisradius $r_u$, siehe \autoref{eq:fuenfeck_umkreisradius}, und die Diagonale $d$ zwischen zwei Spitzen, siehe \autoref{eq:fuenfeck_diagonale}, berechnen.

\begin{equation}
r_u = \frac{a}{10} \sqrt{50 + 10 \sqrt{5}} \label{eq:fuenfeck_umkreisradius}
\end{equation}

\begin{equation}
d = \frac{a}{2} \left(1 + \sqrt{5} \right) \label{eq:fuenfeck_diagonale}
\end{equation}


%%%%%%%%%%%%%%%%%%%%%%%%%%%%%%%%%%%%%%%%%%%%%%%%%%%%%%%%%%%%%%%%%%%%%%%%%%%%%%%%
%
%	- Bayesian statistics
%		- https://en.wikipedia.org/wiki/Bayesian_statistics
%	- Law of total probability
%		- https://en.wikipedia.org/wiki/Law_of_total_probability
%	- Rule of Total Probability and Bayes' Rule
%		- https://www.youtube.com/watch?v=VvThd5zRQC4
%		- https://www.youtube.com/watch?v=8SCmdX68pIk
%	- Zufallsvariable, Massenfunktion, Dichtefunktion und Verteilungsfunktion
%		- https://www.youtube.com/watch?v=DoHTsDrzAQk
%	- A modern introduction to probability and statistics.pdf
%
%%%%%%%%%%
\section{Wahrscheinlichkeitstheorie}

Ob sich ein Roboter an einer ganz bestimmten Position befindet, lässt sich in der Praxis nicht genau bestimmen. Auch durch das genauere Vermessen der Position bleibt eine gewisse Unsicherheit. Um diese Unsicherheit abzubilden, wird in der Robotik die Wahrscheinlichkeitstheorie angewendet. Zum Einsatz kommen hierbei kontinuierliche Zufallsvariablen und die mit ihnen verbundene \Gls{wdf}. In der Regel wird eine Gaußsche Normalverteilung eingesetzt, die sich über den Mittelwert $\mu$ und Varianz $\sigma^2$ beschreiben lässt, siehe \autoref{eq:normal_distribution}. Häufig wird die Normalverteilung auch über $\operatorname{\mathcal{N}}{(x; \mu, \sigma^2)}$ abgekürzt.

\begin{equation}
p(x) = \left( 2 \pi \sigma^2 \right)^{-\frac12}  \exp{ \left\{ -\frac12 \frac{(x - \mu)^2}{\sigma^2} \right\} } \label{eq:normal_distribution}
\end{equation}

Da es sich bei der Position nicht um einen skalaren Wert handelt, muss eine mehrdimensionale Normalverteilung eingesetzt werden, siehe \autoref{eq:multivariate_normal_distribution}. Diese wird über einen Vektor der Mittelwerte $\mu$ und eine symmetische Kovarianzmatrix $\Sigma$ beschrieben.

\begin{equation}
p(x) = \operatorname{det}{\left( 2 \pi \Sigma \right)}^{-\frac12}  \exp{ \left\{ -\frac12 (x - \mu)^T \Sigma^{-1} (x - \mu) \right\} } \label{eq:multivariate_normal_distribution}
\end{equation}

Die Multivariate Verteilung zweier Zufallsvariablen beschreibt die Wahrscheinlichkeit, mit welcher das Ereignis $x$ sowie das Ereignis $y$ auftritt. Sollten beide Zufallsvariablen unabhängig von einander sein, so lassen sich ihre Wahrscheinlichkeiten multiplizieren, siehe \autoref{eq:independent_joint_distribution}.

\begin{equation}
p(x, y) = p(x) \, p(y) \label{eq:independent_joint_distribution}
\end{equation}

Von der bedingten Wahrscheinlichkeit spricht man dann, wenn eine Zufallsvariable, Wissen über eine anderen Zufallsvariable besitzt, siehe \autoref{eq:conditional_probability}.

\begin{equation}
p(x \mid y) = \frac{p(x, y)}{p(y)} \label{eq:conditional_probability}
\end{equation}

Die bedingte Wahrscheinlichkeit von zwei Zufallsvariablen die unabhängig von einander sind, reduzieren sich zu der Wahrscheinlichkeit einer Zufallsvariable, siehe \autoref{eq:independent_joint_distribution} und \autoref{eq:independent_conditional_probability}.

\begin{equation}
p(x \mid y) = \frac{p(x) \, p(y)}{p(y)} = p(x) \label{eq:independent_conditional_probability}
\end{equation}

Der Satz von Bayes sagt aus, das ein Verhältnis zwischen der bedingten Wahrscheinlichkeit $p(x \mid y)$ und der umgekehrten Form $p(y \mid x)$ besteht, siehe \autoref{eq:bayes_rule}.

\begin{equation}
p(x \mid y) = \frac{p(y \mid x) \, p(x)}{p(y)} \label{eq:bayes_rule}
\end{equation}

Der Nenner $p(y)$ hängt nicht von $x$ ab, ist daher für jedes Ereignis konstant und wird als Konstante $\eta$ vor die Gleichung gesetzt, siehe \autoref{eq:bayes_rule_with_normalizer}.

\begin{equation}
p(x \mid y) = \eta \, p(y \mid x) \, p(x) \label{eq:bayes_rule_with_normalizer}
\end{equation}


%%%%%%%%%%%%%%%%%%%%%%%%%%%%%%%%%%%%%%%%%%%%%%%%%%%%%%%%%%%%%%%%%%%%%%%%%%%%%%%%
%
%
%
%%%%%%%%%%
\section{Zustandsschätzer}

Ein Roboter der sich in seiner Umwelt zurechtfinden soll, muss diese modellieren. Das Ergebnis der Modellierung bezeichnet man als Zustand. Ein Zustand beschreibt dabei alle Aspekte des Roboters und seiner Umwelt die einen Einfluss auf die Zukunft haben können. Ein Zustand kann dabei aus statischen sowie dynamischen Komponenten bestehen. Aus dem letzteren ergibt sich zwangsweise, dass ein Zustand sich auch über die Zeit verändern kann. Ein Zustand kann z.B. die Pose des Roboters sein, die Positionen von stationären Landmarken oder aber auch der Ladezustand der Energieversorgung. Beschrieben wird er Zustand mit dem Symbol $x$.

Die meisten Roboter verfügen über Sensoren um Änderungen an Ihrer Umwelt wahrzunehmen. Diese Wahrnehmungen werden dabei mit dem Symbol $z$ beschrieben. Zusätzlich kann der Roboter aktiv Änderungen an seiner Umwelt vornehmen, in dem er Steuerbefehle and die Aktorik sendet und sich somit durch seine Umwelt bewegt. Steuerbefehle werden mit dem Symbol $u$ beschrieben.

Um Auszudrücken, dass die Information des Zustandes, der Wahrnehmung und der Steuerbefehle, zu einem bestimmen Zeitpunkt gehören wird der Index $x_t$ verwendet. Eine Zeitspanne wird über den Index $x_{1:t}$ ausgedrückt. Im mathematischen Sinne werden alle drei Größe als Spaltenvektoren beschrieben, die nicht über die gleiche Dimension verfügen müssen. Zum Beispiel könnte der Zustand aus den X-/Y-Position und der Orientierung $\theta$ zusammengesetzt sein, während die Wahrnehmung aus vielen hunderten Entfernungsmessungen bestehen.

Der Zustand der Umwelt kann nicht direkt gemessen werden, stattdessen muss dieser aus den Daten der Wahrnehmung und der Steuerbefehle geschlussfolgert werden. Das interne Wissen des Roboters über den Zustand seiner Umwelt wird dabei als Belief $bel(x_t)$ bezeichnet. Der Belief wird dabei durch eine bedingte Wahrscheinlichkeitsverteilung repräsentiert. Jeder Zustandshypothese ordnet der Belief dabei eine Wahrscheinlichkeit für ihre Gültigkeit zu.


%%%%%%%%%%%%%%%%%%%%%%%%%%%%%%%%%%%%%%%%%%%%%%%%%%%%%%%%%%%%%%%%%%%%%%%%%%%%%%%%
%
%	- A Short Introduction to the Bayes Filter and Related Models
%		- http://ais.informatik.uni-freiburg.de/teaching/ws12/mapping/pdf/slam02-bayes-filter-short.pdf
%	- Bayes Filter
%		- https://www.tu-chemnitz.de/informatik/KI/edu/robotik/ws2012/robotik_6_2.pdf
%
%%%%%%%%%%
\subsection{Bayes Filter}

Bei dem Bayes Filter Algorithmus handelt es sich um einen sehr abstrakte Beschreibung eines rekursiven Zustandsschätzer, siehe \autoref{fig:algorithm_bayes_filter}. Das heißt, aufbauend  auf der aktuellen Zustandsschätzung wird mit dem eintreffen neuer Wahrnehmungs- oder Steuerbefehl-Daten eine neue Zustandschätzung generiert. Abhängig vom Vorwissen des initialen Zustandes $bel(x_0)$ liefert diese entweder eine punktförmige Verteilung durch den initialen Zustand zurück oder eine stetige Gleichverteilung über den kompletten Zustandsraum. 

\begin{figure}
	\centering
	\includegraphics[width=0.7\linewidth]{algorithm_bayes_filter}
	\captionwithcite{Bayes Filter Algorithmus}{\cite{thrun2005probabilistic}}
	\label{fig:algorithm_bayes_filter}
\end{figure}

Der Bayes Filter geht dabei in zwei Schritt vor. Im ersten Schritt wird eine Prognose (engl. prediction) für den neuen Zustand erstellt, siehe Zeile 3. Der neue Zustand wird dabei aus dem vorherigen Zustand und dem aktuellen Steuerbefehl gebildet, siehe $p(x_t \mid, u_t, x_{t-1})$. Die daraus resultierende Verteilung wird mit der vorherige Belief Verteilung $bel(x_{t-1})$ multipliziert und ergibt die Prognose $\overline{bel}(x_t)$ für den neuen Zustand. Im zweite Schritt wird die erstellt Prognose mit der aktuellen Wahrnehmung korrigiert (engl. correction), siehe Zeile 4. Hierbei ist zu beachten, das für jede Zustandshypothese $x_t$ die erstelle Prognose $\overline{bel}(x_t)$ mit der Wahrscheinlichkeit multiplilziert wird, das die Wahrnehmung zu der Zustandprognose zutrifft $p(z_t \mid x_t)$.


%%%%%%%%%%%%%%%%%%%%%%%%%%%%%%%%%%%%%%%%%%%%%%%%%%%%%%%%%%%%%%%%%%%%%%%%%%%%%%%%
%
%	- Kalman fitering
%		- Multivariate Gaussian distribution (Mehrdimensionale Normalverteilung)
%		- \url{https://de.wikipedia.org/wiki/Mehrdimensionale_Normalverteilung}
%		- Kompakte Beschreibung der Normalverteilung über den Erwartungswert $\mu$ und die Kovarianzmatrix $\Sigma$ ($\mu$ und $\sigma^2$)
%		- \url{https://matheguru.com/stochastik/normalverteilung.html}
%	- Special Topics - The Kalman Filter (1 of 55) What is a Kalman Filter?
%		- https://www.youtube.com/watch?v=CaCcOwJPytQ
%	- Introduction_to_Kalman_Filtering.pdf
%
%%%%%%%%%%
\subsection{Kalman Filter}

Die Familie der Gauß Filter, zu denen auch der Kalman Filter gehört, ist eine der frühsten konkreten Implementierungen des Bayes Filters. Die Idee hinter jedem Gauß Filter liegt dabei, das der Belief durch eine mehrdimensionale Normalverteilung repräsentiert wird, die eindeutig durch die ersten beiden Momente Mittelwert $\mu$ und Kovarianz $\Sigma$ beschrieben wird. Der große Vorteil einer normalverteilten Zufallsvariable ist, dass das Ergebnis einer Lineartransformation mit einer anderen normalverteilten Zufallsvariable wieder zu einer normalverteilten Zufallsvariable führt. Zu den Nachteilen zählt, dass der Gauß Filter nur im kontinuierlichen Zustandsraum angewendet werden kann und durch die unimodale Verteilung nur für die Positionsverfolgung (engl. Position Tracking Problem) in Frage kommt.

Genauso wie der Bayes Filter geht auch der Kalman Filter in zwei Schritten vor, siehe \autoref{fig:algorithm_kalman_filter}. Im ersten Schritt wird aus dem vorherigen Zustand $\mu_{t-1}$ und $\Sigma_{t-1}$ und den aktuellen Steuerbefehlen $u_t$ die Prognose für den neuen Zustand berechnet, siehe Zeile 2--3. Über die Matrix $A_t$ wird festgelegt, wie sich der Zustand von $t-1$ nach $t$ entwickelt, ohne die Einwirkungen der Steuerbefehle oder von Rauschen. Die Matrix $B_t$ beschreibt wie die Steuerbefehle die Zustandsänderung von $t-1$ nach $t$ beeinflussen. Im zweiten Schritt wird zuerst der Kalman Gain $K_t$ berechnet, siehe Zeile 4. Bei dem Kalman Gain handelt es sich um ein Gewicht, das entscheidet wie stark die Prognose mit der Wahrnehmung korrigiert wird, siehe Zeile 5--6. Bei der Matrix $C_t$ wird definiert wir die Wahrnehmungen auf den Zustand abgebildet werden. Die beiden Matrizen $R_t$ und $Q_t$ bilden dabei das unabhängige und zufällig normalverteilte Rauschen ab.

\begin{figure}
	\centering
	\includegraphics[width=0.7\linewidth]{algorithm_kalman_filter}
	\captionwithcite{Kalman Filter Algorithmus}{\cite{thrun2005probabilistic}}
	\label{fig:algorithm_kalman_filter}
\end{figure}


%%%%%%%%%%%%%%%%%%%%%%%%%%%%%%%%%%%%%%%%%%%%%%%%%%%%%%%%%%%%%%%%%%%%%%%%%%%%%%%%
%
%
%
%%%%%%%%%%
\subsection{Extended Kalman Filter}

Der Kalman Filter geht von der Annahme aus, dass die Zustandsänderung und die Wahrnehmungen durch eine lineare Funktion beschrieben werden können. Diese Annahme ist entscheidend für das ordnungsgemäße Funktionieren des Kalman Filters. In der \autoref{fig:random_variable_linear_transformation} ist im rechten unteren Bereich die Zufallsvariable $x$ mit ihrer Normalverteilung dargestellt. Wird diese über eine lineare Funktion $y=ax+b$ auf die Zufallsvariable $y$ abgebildet. Es ist klar zu erkennen, dass die Normalverteilung für die Zufallsvariable $y$ erhalten bleibt.

Das Hauptproblem bei der linearen Annahme ist, dass viele Zustandsänderung und Wahrnehmungen nicht mit einer linearen Funktion beschrieben werden können, z.B. rotatorische Zustandsänderungen. Die \autoref{fig:random_variable_non_linear_transformation} verdeutlich, was mit der Normalverteilung einer Zufallsvariable $x$ passiert, wenn diese durch eine nicht lineare Funktion $g(x)$ abgebildet wird. Von der Normalverteilung der Zufallsvariable $y$ bleibt nicht mehr viel übrig.

Der \Gls{ekf} löst das Problem in dem eine Linearisierung der nicht linearen Funktion vorgenommen wird, siehe \autoref{fig:random_variable_taylor_linearization}. Hierzu wird eine Taylorreihe des ersten Grades gebildet. Der Funktionswert der Mittelwertes $g(\mu)$ dient dabei als Entwicklungsstelle. Durch die Linearisierung wird die Normalverteilung der Zufallsvariable $x$ wieder ordnungsgemäß auf die Zufallsvariable $y$ abgebildet. Da es sich bei der Linearisierung nicht um die Originalfunktion handelt entsteht ein Linearisierungsfehler der am Unterschied zwischen der durchgezogen und gestrichelten Kurve zu erkennen ist.

\begin{figure}
	\centering
	\begin{subfigure}{0.49\linewidth}
		\centering
		\includegraphics[width=\linewidth]{random_variable_linear_transformation}
		\caption{}
		\label{fig:random_variable_linear_transformation}
	\end{subfigure}
	\hfill
	\begin{subfigure}{0.49\linewidth}
		\centering
		\includegraphics[width=\linewidth]{random_variable_non_linear_transformation}
		\caption{}
		\label{fig:random_variable_non_linear_transformation}
	\end{subfigure}
	\par
	\bigskip
	\begin{subfigure}{0.49\linewidth}
		\centering
		\includegraphics[width=\linewidth]{random_variable_taylor_linearization}
		\caption{}
		\label{fig:random_variable_taylor_linearization}
	\end{subfigure}
	\captionwithcite{Lineare und nicht lineare Transformation einer normalverteilten Zufallsvariable.}{\cite{thrun2005probabilistic}}
	\label{fig:random_variable_transformation}
\end{figure}


%%%%%%%%%%%%%%%%%%%%%%%%%%%%%%%%%%%%%%%%%%%%%%%%%%%%%%%%%%%%%%%%%%%%%%%%%%%%%%%%
%
% Belief: A belief reflects the robot’s internal knowledge about the state of the environment.
% Posterior:
%	- is the conditional probability that is assigned after the relevant evidence or background is taken into account.
% sample: Stichprobe, Abtasten
% a posteriori:
% a priori:
% non-parametric: parameterfreie, nichtparametrische, verteilungsfreie
% nonlinear transformations of random variables:
%
%%%%%%%%%%
\section{\glsentrylong{pf}}

Bisherige Filter haben eine parametrisierte Form genutzt, um die Verteilung zu modellieren. Der \Gls{pf} geht einen anderen Weg und verwendet stattdessen eine Menge von Stichproben (engl. Samples) aus der gewünschten Verteilung. Dadurch ist es möglich multimodale Verteilungen abzubilden, die jedoch nur einer Annäherung entsprechen. Weiterhin ist es möglich Zufallsvariablen durch eine nicht lineare Funktion abzubilden, siehe \autoref {fig:particle_filter_with_non_linear_function}. Im unteren rechten Bild wird eine Menge von Stichproben aus einer Normalverteilung dargestellt. Diese werden dann durch die nicht lineare Funktion $g(x)$ auf eine multimodale Verteilung abgebildet. Je dichter die Stichproben an einander liegen desto höher ist an dieser Stelle auch die Wahrscheinlichkeit.

\begin{figure}
	\centering
	\includegraphics[width=0.6\linewidth]{particle_filter_with_non_linear_function}
	\captionwithcite{Darstellung einer Zufallsvariable vor und nach der Abbildung, durch eine nicht lineare Funktion, mittels einem \glsentrylong{pf}.}{\cite{thrun2005probabilistic}}
	\label{fig:particle_filter_with_non_linear_function}
\end{figure}

Die Stichproben aus der a posteriori Verteilung werden als Partikel (engl. Particle) bezeichnen. Jeder Partikel~$x_t$ entspricht dabei einer Hypothese des Weltzustandes zum Zeitpunkt~$t$. Über ein Gewicht~$w_t$ wird die Relevanz (engl. Importance Weight) eines Partikels ausgedrückt. Die Idee ist es dabei, durch ein Set von Partikeln den Belief~$bel(x_t)$ möglichst gut Anzunähern, siehe \autoref{eq:particle_filter_particle_set}. Der Parameter $M$ entspricht dabei der Anzahl der Partikel und wird sehr großzügig gewählt, z.B. $M=1000$.

\begin{equation}
\mathcal{X}_t := x^{\lbrack 1 \rbrack}_t, x^{\lbrack 2 \rbrack}_t, \ldots, x^{\lbrack M \rbrack}_t \label{eq:particle_filter_particle_set}
\end{equation}

Bei dem \Gls{pf} handelt es sich, wie bei dem Bayes Filter, ebenfalls um einen rekursiven Algorithmus, dem das vorherige Partikel-Set $\mathcal{X}_{t-1}$, die aktuellen Steuerbefehle $u_t$ und Wahrnehmungen $z_t$ übergeben werden, siehe \autoref{fig:algorithm_particle_filter}. Dabei wird im ersten Schritt auf jedem Partikel, aus dem vorherigen Set, die aktuellen Steuerbefehle angewendet und somit eine neue Prognose für den aktuellen Zustand gebildet, siehe Zeile 4. Weiterhin wird mit der neuen Zustandsprognose und den aktuellen Wahrnehmungen das Gewicht des Partikels bestimmt, siehe Zeile 5. Die aktuelle Zustandsprognose sowie das Gewicht werden danach in dem temporären Set $\overline{\mathcal{X}}_t$ gesammelt, siehe Zeile 6. Im zweiten Schritt findet das sogenannte \textit{Resampling} oder \textit{Importance Sampling} statt, siehe Zeile 9. Hierbei werden die Partikel mit einem hohen Gewicht beibehalten und die mit einem geringen durch neue ersetzt, die im Bereich der Partikel mit den hohen Gewichten liegen. Dadurch ist gewährleistet, dass sich die Partikel nicht in Bereichen mit einer geringen a posteriori Wahrscheinlichkeit aufhalten. Das Ergebnis aus dem letzten Schritt wird in dem Partikel-Set $\mathcal{X}_t$ gespeichert und dem Aufrufer zurückgegeben.

\begin{figure}
	\centering
	\fbox{	\includegraphics[width=0.5\linewidth]{algorithm_particle_filter}}
	\captionwithcite{\glsentrylong{pf} Algorithmus}{\cite{thrun2005probabilistic}}
	\label{fig:algorithm_particle_filter}
\end{figure}


%%%%%%%%%%%%%%%%%%%%%%%%%%%%%%%%%%%%%%%%%%%%%%%%%%%%%%%%%%%%%%%%%%%%%%%%%%%%%%%%
%
%	- \cite{kurth2003experimental}
%		- Monte Carlo localization, or particle filtering, provides a method of representing multimodal distri-butions for position estimation [4, 12], with the ad-vantage that the computational requirements can be scaled. The main advantage of these methods is their ability to recover robustly from a poor initial condition.
%	- \cite{fox1999monte}
%
%%%%%%%%%%
\subsection{\glsentrylong{mcl}}

Das Bestimmen der Pose eines Roboters relativ zu einer gegebenen Karte der Umwelt wird als Lokalisierung bezeichnet. Die Lokalisierung bestimmt dabei die Transformation zwischen dem lokalen Koordinatensystem des Roboters und dem globalen Koordinatensystem der Karte. Ist die initiale Pose des Roboters nicht gegeben, wird dieses Problem als globale Lokalisierung bezeichnet. Je größer die Karte der Umwelt ist, desto wahrscheinlicher wird es, dass die Wahrnehmung, Mehrdeutigkeiten in Bezug zu der Karte aufweist, die sich in einer multimodalen Verteilung wiederspiegeln und damit ideal durch einen \Gls{pf} abgebildet werden kann. Bei der \Gls{mcl} handelt es sich um genau so einen \Gls{pf}, der mittlerweile zu den Standardalgorithmen der Robotik gehört.

Die zwingende Voraussetzung, für die Lokalisierung, ist das vorhanden sein einer sehr genauen Karte der Umwelt. Die Kartendaten $m$ werden bei der Berechnung der Gewichte für die Zustandsprognosen eingesetzt. Somit unterscheidet sich der Monte Carlo Algorithmus nur in der fünften Zeile von dem generische Partikel Filter Algorithmus, siehe \autoref{eq:monte_carlo_measurement_model}. 

\begin{equation}
w^{\lbrack m \rbrack}_t = p(z_t \mid x^{\lbrack m \rbrack}_t, m) \label{eq:monte_carlo_measurement_model}
\end{equation}

% TODO: Erklären wie die MCL abläuft?
%
%- 8. Monte Carlo Localization
%	- Figure 8.11 - Monte Carlo Localization, a particle filter applied to mobile robot localization.
%	- The initial global uncertainty is achieved through a set of pose particles drawn at random and uniformly over the entire pose space, as shown in Figure 8.11a.
%	- As the robot senses the door, MCL assigns importance factors to each particle. The resulting particle set is shown in Figure 8.11b.
%	- The height of each particle in this figure shows its importance weight.
%	- It is important to notice that this set of particles is identical to the one in Figure 8.11a—the only thing modified by the measurement update are the importance weights.
%	- Figure 8.11c shows the particle set after resampling and after incorporating the robot motion. This leads to a new particle set with uniform importance weights, but with an increased number of particles near the three likely places.
%	- The new measurement assigns non-uniform importance weights to the particle set, as shown in Figure 8.11d.
%	- At this point, most of the cumulative probability mass is centered on the second door, which is also the most likely location.

%\begin{figure}
%	\centering
%	\begin{subfigure}{0.8\linewidth}
%		\centering
%		\includegraphics[width=\linewidth]{thrun2005probabilistic_fig811a}
%		\caption{}
%		\label{fig:thrun2005probabilistic_fig811a}
%	\end{subfigure}
%	\par\bigskip
%	\begin{subfigure}{0.8\linewidth}
%		\centering
%		\includegraphics[width=\linewidth]{thrun2005probabilistic_fig811b}
%		\caption{}
%		\label{fig:thrun2005probabilistic_fig811b}
%	\end{subfigure}
%	\par\bigskip
%	\begin{subfigure}{0.8\linewidth}
%		\centering
%		\includegraphics[width=\linewidth]{thrun2005probabilistic_fig811c}
%		\caption{}
%		\label{fig:thrun2005probabilistic_fig811c}
%	\end{subfigure}
%	\par\bigskip
%	\begin{subfigure}{0.8\linewidth}
%		\centering
%		\includegraphics[width=\linewidth]{thrun2005probabilistic_fig811d}
%		\caption{}
%		\label{fig:thrun2005probabilistic_fig811d}
%	\end{subfigure}
%	\par\bigskip
%	\begin{subfigure}{0.8\linewidth}
%		\centering
%		\includegraphics[width=\linewidth]{thrun2005probabilistic_fig811e}
%		\caption{}
%		\label{fig:thrun2005probabilistic_fig811e}
%	\end{subfigure}
%	\captionwithcite{Beispiel für die Positionsbestimmung eines Roboter mittels der \glsentrylong{mcl}.}{\cite{thrun2005probabilistic}}
%	\label{fig:thrun2005probabilistic_fig811}
%\end{figure}


%%%%%%%%%%%%%%%%%%%%%%%%%%%%%%%%%%%%%%%%%%%%%%%%%%%%%%%%%%%%%%%%%%%%%%%%%%%%%%%%
%
% 	- \cite{kurth2003experimental}
%		- We are currently developing a batch localization method, which considers all the data collected by the robot and finds the best path estimate given all the data. Although time consuming computationally, this will produce the theoretically optimal result obtainable from the collected data; we can then evaluate the results of our online localization method by comparing to this optimal solution.
%	- \cite{}
%		- Comparison of Batch and Kalman Filtering for Radar Tracking
%		- http://www.dtic.mil/dtic/tr/fulltext/u2/p011192.pdf
%		- The EKF propagates the filter state between measurements, and incorporates measurements sequentially, with the error dynamics and observation models respectively. It is an implementation of the often referenced linear-minimum-variance (Kalman) formulas, adapted by first-variation approximations of the nonlinear models, centered at the current state estimate, and described in Reference [4] (Jazwinski). 	
%		- The Batch algorithm, in comparison, processes those same measurements simultaneously via iterative least-squares estimation. An overview of the comparison of measurement processing among the two algorithms is shown in Figure-1. Batch algorithm estimations were made in parallel with those of the EKF for comparison of estimation error time histories.
%
%%%%%%%%%%
%\section{Batch optimization [todo,optional]}


%%%%%%%%%%%%%%%%%%%%%%%%%%%%%%%%%%%%%%%%%%%%%%%%%%%%%%%%%%%%%%%%%%%%%%%%%%%%%%%%
%
%	- \cite{kurth2003experimental}
%		- Additionally, we will extend the batch method to produce a variable dimension filter, as used by Deans for the case of bearing-only sensors [3], which would consider some window of previous robot states and optimize the position estimates based on the data in that window.
%
%%%%%%%%%%
%\section{Variable Dimension Filter [todo,optional]}


%%%%%%%%%%%%%%%%%%%%%%%%%%%%%%%%%%%%%%%%%%%%%%%%%%%%%%%%%%%%%%%%%%%%%%%%%%%%%%%%
%
%	- SLAM
%		- Verwenden der Sensordaten um sich zu Lokalisierung...
%		- und eine Karte der Landmarken zu erzeugen.
%		- Bisher Winkel und Entferung zu einer Landmarke gegeben
%			- Computer Vision, Structure from Motion
%	- Embodied Localisation and Mapping
%		- http://elib.suub.uni-bremen.de/edocs/00103537-1.pdf
%
%%%%%%%%%%
\section{\glsentrylong{slam}}


% Was versteht man unter Mapping?
%- a priori map
%- And even if blueprints were accurate, they would not contain furniture and other items that, from a robot’s perspective, determine the shape of the environment just as much as walls and doors. Being able to learn a map from scratch can greatly reduce the efforts involved in installing a mobile robot, and enable robots to adapt to changes without human supervision
%- In this chapter, we first study the mapping problem under the restrictive assumption that the robot poses are known.
%	- is also known as mapping with known poses.
%	- Occupancy grid mapping addresses the problem of generating consistent maps from noisy and uncertain measurement data, under the assumption that the robot pose is known.
%	- The basic idea of the occupancy grids is to represent the map as a field of random variables, arranged in an evenly spaced grid. Each random variable is binary and corresponds to the occupancy of the location it covers. Occupancy grid mapping algorithms implement approximate posterior estimation for those random variables.


% Was versteht man unter SLAM?
Wenn der Roboter weder seine eigne Pose kennt noch eine Karte seiner Umwelt besitzt, beides jedoch aus den Steuerbefehlen $u_{1:t}$ und den Wahrnehmungen $z_{1:t}$ bestimmen soll, wird das als \Gls{slam}\footnotemark{} Problem bezeichnet. 

\footnotetext{Alternativ zu \acrfull{slam} wird auch \acrfull{cml} verwendet.}

% Unterschied zwischen Online/Full SLAM? {Graph-Modell, Eqaution}
Es gilt dabei zu unterscheiden zwischen dem Online- und Full-\gls{slam}. Beim Online-\gls{slam} wird die aktuelle Pose $x_t$ innerhalb der Karte geschätzt, siehe \autoref{eq:online_slam_problem}, während beim Full-\gls{slam} die komplette Trajektorie $x_{1:t}$ innerhalb der Karte geschätzt wird, siehe \autoref{eq:full_slam_problem}.

\begin{equation}
p(x_t, m \mid z_{1:t}, u_{1:t}) \label{eq:online_slam_problem}
\end{equation}

\begin{equation}
p(x_{1:t}, m \mid z_{1:t}, u_{1:t}) \label{eq:full_slam_problem}
\end{equation}

%\begin{figure}
%	\centering
%	\begin{subfigure}{0.4\linewidth}
%		\centering
%		\includegraphics[width=\linewidth]{thrun2005probabilistic_fig10_1}
%		\caption{}
%		\label{fig:thrun2005probabilistic_fig10_1}
%	\end{subfigure}
%	\hfill
%	\begin{subfigure}{0.4\linewidth}
%		\centering
%		\includegraphics[width=\linewidth]{thrun2005probabilistic_fig10_2}
%		\caption{}
%		\label{fig:thrun2005probabilistic_fig10_2}
%	\end{subfigure}
%	\captionwithcite{Die graphischen Modelle für den Online- bzw. Full-\gls{slam}.}{\cite{thrun2005probabilistic}}
%	\label{fig:thrun2005probabilistic_fig10_1_2}
%\end{figure}


%%%%%%%%%%%%%%%%%%%%%%%%%%%%%%%%%%%%%%%%%%%%%%%%%%%%%%%%%%%%%%%%%%%%%%%%%%%%%%%%
%
%
%
%%%%%%%%%%
\subsection{EKF-SLAM [todo]}

% Prediction aktulisierung nur die Robot Pose
% Corection werden auch die Landmarken aktualisiert

% Welche Annahmen werden getroffen? {2D plane, known data association, known number of landmarks}
% Wie sieht der Zustandsvektor aus?
% Wie sieht die Modellierung mit mean und covarianz aus?

%
%
%- SLAM with Extended Kalman Filters
%	- In a nutshell, the EKF SLAM algorithm applies the EKF to online SLAM using maximum likelihood data association. In doing so, EKF SLAM is subject to a number of approximations and limiting assumptions:
%		- FEATURE - BASED MAPS
%		- Maps, in EKF SLAM, are feature-based.
%		- They are composed of point landmarks.
%		- For computational reasons, the number of point landmarks is usually small (e.g., smaller than 1,000).
%		- Further, the EKF approach tends to work well the less ambiguous the landmarks are.
%		- For this reason, EKF SLAM re quires significant engineering of feature detectors, sometimes using artificial beacons as features.
%		- GAUSSIAN NOISE ASSUMPTION
%		- As any EKF algorithm, EKF SLAM makes a Gaussian noise assumption for robot motion and perception.
%		- The amount of uncertainty in the posterior must be relatively small, since otherwise the linearization in EKFs tend to introduce intolerable errors.
%		- POSITIVE INFORMATION
%		- can only process positive sightings of landmarks. It cannot process negative information that arises from the absence of landmarks in sensor measurements.
%	- combined state vector
%		- In addition to estimating the robot pose x t , the EKF SLAM algorithm also estimates the coordinates of all landmarks encountered along the way. This makes it necessary to include the landmark coordinates into the state vector.
%		- For convenience, let us call the state vector comprising robot pose and the map the combined state vector, and denote this vector y t .
%		- Here x, y, and θ denote the robot’s coordinates at time t (not to be confused with the state variables x t and y t ), m i,x , m i,y are the coordinates of the i-th landmark, for i = 1, . . . , N , and s i is its signature.
%		- EKF SLAM calculates the online posterior p(y t | z 1:t , u 1:t ).
%
%



%%%%%%%%%%%%%%%%%%%%%%%%%%%%%%%%%%%%%%%%%%%%%%%%%%%%%%%%%%%%%%%%%%%%%%%%%%%%%%%%
%
% - https://www.mrpt.org/tutorials/slam-algorithms/rangeonly_slam/
%
%%%%%%%%%%
\subsection{RO-SLAM [todo]}


%%%%%%%%%%%%%%%%%%%%%%%%%%%%%%%%%%%%%%%%%%%%%%%%%%%%%%%%%%%%%%%%%%%%%%%%%%%%%%%%
%
%	- Rao-Blackwellized Particle Filtering
%		- https://people.eecs.berkeley.edu/~pabbeel/cs287-fa12/slides/RBPF.pdf
%	- \cite{murphy2001rao}
%		- Rao-Blackwellised particle filtering for dynamic Bayesian networks
%
%%%%%%%%%%
\subsection{Rao-Blackwellized [todo]}

% Model the robot's path by sampling and compute the landmarks given the pose
% each particle represents a possible trajectory of the robot
% each particle maintains its own map (2d ekf)
% each particle updates it upon <mapping with known poses>

% SLAM mit Partikelfiltern
%		FastSLAM
%		Grid-basierte Verfahren mit Rao-Blackwellisierten Partikelfiltern



%%%%%%%%%%%%%%%%%%%%%%%%%%%%%%%%%%%%%%%%%%%%%%%%%%%%%%%%%%%%%%%%%%%%%%%%%%%%%%%%
%
%	- \url{https://www.heise.de/developer/artikel/Einfuehrung-in-das-Robot-Operating-System-3273655.html?seite=all}
%	- Node = Knoten
%	- Topic = Datenbus
%	- Service = Dienst
%	- Message = Nachricht
%	- Launch-Files = Startdateien
%
%%%%%%%%%%
\section{Robot Operating System}

Bei dem \Gls{ros} handelt es sich nicht im eigentlichen Sinne um ein Betriebssystem, sondern vielmehr um ein Framework das die Kommunikation zwischen verschiedenen Verarbeitungseinheiten regelt. Im Jahre 2007 begann die Entwicklung von \Gls{ros} an der Stanford University. Ab 2009 wurde dieses dann hauptsächlich an dem Robotik Institut Willow Garage weiterentwickelt. Durch die BSD Lizenz steht \Gls{ros} als Open-Source-Projekt sowohl der nicht-kommerziellen auch der kommerziellen Weiterentwicklung zur Verfügung. \cite{quigley2009ros}

Jedes \Gls{ros}-System muss über einen \Gls{ros}-Master verfügen. Dieser stellt den zentralen Punkt für die Registrierung von Knoten (engl. Nodes), Datenbusse (engl Topics) und Dienste (engl. Services) zur Verfügung.

Jeder Verarbeitungseinheit in \Gls{ros} wird durch einen Knoten repräsentiert. Ziel ist es möglichst kleine, wiederverwendbare und miteinander kombinierbare Einheiten zu bilden. Die Programmierung eines Knoten erfolgt dabei in den Programmiersprachen C++, Python oder Lisp.

Die Kommunikation zwischen den Knoten erfolgt dabei über ein \Gls{p2p}-Kanal auf der Basis von Nachrichten (engl. Messages). Bei einer Nachricht handelt es sich um eine Datenstruktur die primitive Datentypen, die Datentypstruktur anderer Nachrichten und Felder enthalten kann. Eine Nachricht wird dabei über das \textit{Publish-Subscribe} Muster in einem Datenbus veröffentlicht und kann von jedem Knoten empfangen werden der Nachrichten von diesem Datenbus abonniert hat. Der typische Anwendungsfall für dieses Verfahren ist die Bereitstellung von Sensordaten und Rückmeldungen über Statusänderungen.

Das zuvor vorgestellte \textit{Publish-Subscribe} Muster stellt eine asynchrone Kommunikation bereit. Mit Hilfe von Diensten ist auch ein synchroner Nachrichtenaustausch möglich. Hierfür wird eine Anfrage von dem \textit{Client} an den \textit{Server} gestellt, der seinerseits mit einem Ergebnis antwortet. Dieses Verfahren entspricht dem \textit{Request-Response} Muster. Die Erstellung einer inversen Transformation ist ein typischer Anwendungsfall hierfür.

Nach der Entwicklung neuer Algorithmen ist ein Vergleich mit den bereits bestehenden Algorithmen von großem Interesse. Mit dem Konzept der \textit{Bag}-Dateien können alle veröffentlichten Nachrichten aufgezeichnet werden und zu einem späteren Zeitpunkt wieder abgespielt werden.

Um die Wiederverwendbarkeit von Knoten, Nachrichten und Diensten zu fördern, wurde das Konzept der Pakete (engl. Packages) eingeführt. Ein Paket stellt die kleinste erstellbare Einheit dar und beinhaltet alle Teile eines Softwarepaketes wie z.B. Quellcode-, Konfigurationsdateien, Drittanbieter Bibliothek, Abhängigkeitslisten usw.

Eine Roboterplattform besteht aus vielen Sensoren und Aktuatoren die abgefragt und gesteuert werden müssen. Dementsprechend viele Knoten müssen gestartet werden. Diese Aufgabe erfüllen die Startdateien (engl. Launch-Files). Neben der Definition der Knoten die gestartet werden sollen, können Einzelne oder eine Gruppe von Knoten parametrisiert werden. Mittels der Verschachtelung von Startdateien ist eine Wiederverwendung von knotenspezifischen Startdateien möglich.

Zu den häufigsten Operationen der Verarbeitungseinheit einer Roboterplattform ist das Transformieren der Sensordaten aus dem Sensorkoordinatensystem in das Koordinatensystem des Robotermittelpunktes. In \Gls{ros} wird hier für ein Transformationsbaum (engl. TF-Tree) verwendet, der aus statischen und dynamischen Transformationen besteht. Statische Transformationen werden dort eingesetzt, wo sich die Pose zweier Koordinatensysteme zur Laufzeit nicht ändert, z.B. zwischen dem Robotermittelpunkt und einem fest montierten Sensor. Dynamische hingegen bei Koordinatensystemen die sich zur Laufzeit ändern, z.B. zwischen dem Robotermittelpunkt und den Inkrementalgebern der Antriebseinheit. Statische Transformationen können mittels einer \Gls{urdf}-Datei modelliert und allen Knoten bereitgestellt werden.


	%
% Forschungsstand:
%
% - Welche wissenschaftlichen Erkenntnisse liegen zu dem Thema bereits vor?
% - Grundsätzlich gibt es zwei Möglichkeiten, einen Forschungsstand zu schreiben: Entweder ordnen Sie Ihren Literaturüberblick nach Themenkomplexen oder Sie geben einen rein chronologischen Überblick über die wichtigsten Publikationen.
% - Auf keinen Fall sollten Sie den Forschungsstand zu voll packen. Es geht nicht darum, dem Leser zu zeigen, was Sie alles studiert haben (wie fleißig Sie waren), sondern um einen kompakten Überblick über die wichtigste Literatur.
% - Wichtig: Listen Sie die Literatur nicht nur auf, sondern erklären Sie, welchen Beitrag die jeweilige Publikation zum Erkenntnisgewinn geleistet hat. Also, zum Beispiel: Was hat der Autor als Erster erkannt oder hinterfragt? Es muss ja einen Grund geben, weshalb Sie die betreffende Publikation unter die Meilensteine reihen – und den sollten Sie dem Leser deutlich machen.
%
% Ferrein:
% - 4-5 Seiten in der Bachelorarbeit (deutlich umfangreicher als im Exposé)
% - Was unterscheidet meinen Ansatz von den anderen verhandenen Ansätzen?
%
\begin{comment}
------------------------------------------------------------------------------------------
\end{comment}
\chapter{Stand der Forschung und Technik}


\begin{comment}
------------------------------------------------------------------------------------------
\end{comment}
\section{\citefield{kantor2002preliminary}{title}(189)}


\begin{comment}
------------------------------------------------------------------------------------------
\end{comment}
\section{\citefield{kurth2003experimental}{title}(82)}


\begin{comment}
------------------------------------------------------------------------------------------
- Wie funktioniert die Exploration Strategy?
	- Die Ungünstigeste strategy ist das geradeausfahren mit einem beacon links und rechts.
	- Das Gradientenfeld der abstanddifferenz zwischen den beiden beacons führt einen auf den optimalen weg um die Abstandsdifferenz zu maximieren. (Aktive Exploration)
\end{comment}
\section{\citefield{olson2004robust}{title}(264)}

In \citetitle{olson2004robust}\cite{olson2004robust} werden autonome Unterwasserfahrzeuge (engl. \gls{auv}) verwendet um den Ozeanen der Welt die letzten Geheimnisse zu entlocken. Hierfür ist eine genaue Lokalisierung der \gls{auv} notwendig. Dies wird über die Messung der Signallaufzeit zwischen dem \gls{auv} und mehreren stationären akustischen Transponder--\Gls{beacon} erreicht. Vor der eigentlichen Messung müssen jedoch die \Glspl{beacon} im Messbereich verteilt werden, deren Position genau bestimmt werden und zusätzlich dafür gesorgt werden, dass die \Glspl{beacon} ihre Position nicht ändern. Dadurch handelt es sich um ein sehr kostspieliges Prozedere und soll in dem vorgestellt Verfahren dahingehend optimierte werden, dass die \Gls{beacon}--Position vorher nicht bekannt sein muss. Dadurch ist eine autonome Verteilung der \Glspl{beacon} möglich und zusätzlich kann auch detektiert werden ob ein \Gls{beacon} seine Position verändert hat.

% TODO: Spectral Graph Partitioning
Bevor jedoch die Laufzeitmessungen verwendet werden können, müssen diese um \Gls{outlier} (dt. Ausreißer) bereinigt werden. Diese entstehen z.B. durch unterschiedliche Ausbreitungsgeschwindigkeiten der Schallwellen im Wasser, durch Reflektionen am Meeresgrund (engl. \Gls{multipath}) oder durch Interferenzen der Nutzlast (engl. \Gls{payload}) mit den Sensoren. Die Bereinigung der Messdaten erfolgt hierbei über das \textit{Spectral Graph Partitioning} Verfahren. Hierbei wird jede Messung als Kreis dargestellt. Überschneiden sich zwei Kreise gelten diese beiden Messungen als konsistent. Jede der Messungen wird in einem Graphen zu einem Vertex und jede konsistente Messung zu einer Kante. Bei dem fertigen Graphen ist nun zu beobachten, dass \Gls{outlier} weniger stark untereinander verbunden sind als \Gls{inlier}. Diese durchschnittliche Konnektivität der \Gls{inlier} wir als Metrik für den Partitionierungsalogorithmus verwendet um die Messdaten zu filtern.

Nach dem die Messwerte pro \Gls{beacon} gefiltert worden sind, werden zwei Messwerte von zwei verschiedenen \Glspl{beacon} wieder mit einandern überschnitten. Hieraus entstehen zwei mögliche Positionen des \gls{auv}. In einem Grid erhöhen diese beiden Positionen dann den Ihnen zugeordneten Zähler. Dieser Vorgang wird für weiter Messwerte wiederholt. Durch das daraus resultierende Voting--System entstehen zwei Gipfel (engl. Peak) die die mögliche Position des \gls{auv} approximieren. Ist der Höhenunterschied zwischen den Gipfeln ausreichend groß, wird die \gls{auv}--Position mit dem höchsten Gipfel an einen EKF--SLAM übergeben.


\begin{comment}
------------------------------------------------------------------------------------------
\end{comment}
\section{\citefield{smith2004tracking}{title}(547)}



\begin{comment}
------------------------------------------------------------------------------------------
\end{comment}
\section{Übersicht der Meilensteine [Remove in final version]}

\begin{description}

\item[2001]
\begin{itemize}
\item !A solution to the simultaneous localization and map building (SLAM) problem (Zitiert von: 2803)
\item +Auxiliary variable based particle filters (115)
\item +Factor graphs and the sum-product algorithm (5869)
\item +Rao-Blackwellised particle filtering for dynamic Bayesian networks (1264)
\end{itemize}

\item[2002]
\begin{itemize}
\item +FastSLAM: A factored solution to the simultaneous localization and mapping problem (2469)
\item !Preliminary results in range-only localization and mapping (187)
\item Indoor geolocation science and technology (985)
\end{itemize}

\item[2003]
\begin{itemize}
\item !Experimental results in range-only localization with radio (82)
\item !Pure range-only sub-sea SLAM (174)
\item +Recent results in extensions to simultaneous localization and mapping (18)
\end{itemize}

\item[2004]
\begin{itemize}
\item !Tracking moving devices with the cricket location system (547)
\item !A probabilistic approach to inference with limited information in sensor networks (42)
\item +An introduction to factor graphs (672)
\item ?An efficient multiple hypothesis filter for bearing-only SLAM (114)
\end{itemize}

\item[2006]
\begin{itemize}
\item !Simultaneous localization and mapping: part I (2982)
\item !Simultaneous localization and mapping (SLAM): Part II (1479)
\item Further results with localization and mapping using range from radio (69)
\item !Range-only slam for robots operating cooperatively with sensor networks (146)
\item !Range-only slam with interpolated range data (28)
\end{itemize}

\item[2007]
\begin{itemize}
\item !Application of UWB and GPS technologies for vehicle localization in combined indoor-outdoor environments (52)
\item +Rao-Blackwellized particle filter for multiple target tracking (264)
\end{itemize}

\item[2008]
\begin{itemize}
\item !Efficient probabilistic range-only SLAM (Zitiert von: 70)
\item !A pure probabilistic approach to range-only SLAM (Zitiert von: 36)
\end{itemize}

\item[2009]
\begin{itemize}
\item !Navigating with ranging radios: Five data sets with ground truth (18)
\item !Mobile robot localization based on ultra-wide-band ranging: A particle filter approach (90)
\item !A robust method of localization and mapping using only range (45)
\item !Range-only SLAM with a mobile robot and a wireless sensor networks (109)
\end{itemize}

\item[2010]
\begin{itemize}
\item !Geolocation with range: Robustness, efficiency and scalability (11)
\item !Studying of WiFi range-only sensor and its application to localization and mapping systems (3)
\item !tinySLAM: A SLAM algorithm in less than 200 lines C-language program (62)
\end{itemize}

\item[2011]
\begin{itemize}
\item Ultra wide-band localization and SLAM: A comparative study for mobile robot navigation (17)
\item !A new state vector for range-only SLAM (10)
\end{itemize}

\item[2013]
\begin{itemize}
\item A Spectral Learning Approach to Range-Only {SLAM} (13)
\end{itemize}

\item[2014]
\begin{itemize}
\item !A comparison of slam algorithms with range only sensors (2)
\item !Efficient robot-sensor network distributed seif range-only slam (16)
\end{itemize}

\item[2015]
\begin{itemize}
\item ?A robot self-localization system using one-way ultra-wideband communication (18)
\item ?Fusing ultra-wideband range measurements with accelerometers and rate gyroscopes for quadrocopter state estimation (30)
\end{itemize}

\item[2016]
\begin{itemize}
\item !Indoor robot positioning using an enhanced trilateration algorithm (5)
\end{itemize}

\item[2017]
\begin{itemize}
\item ?Ultra-Wideband Aided Fast Localization and Mapping System (1)
\item !A system for indoor positioning using ultra-wideband technology (0)
\item ?Range-only SLAM schemes exploiting robot-sensor network cooperation (0)
\end{itemize}

%\item[]
%\begin{itemize}
%\item 
%\end{itemize}

\end{description}


\begin{comment}
------------------------------------------------------------------------------------------
\end{comment}
\section{Informationen aus dem Expose [Remove in final version]}

Einen guten Überblick über die Eigenschaften der Drahtlosen-Protokolle (engl. Wireless Protocols) Bluetooth, UWB, ZigBee und WiFi liefert die Arbeit \cite{lee2007comparative} von \citeauthor{lee2007comparative}.

In \cite{smith1987closed} wird das grundlegende Prinzip erklärt um aus mehreren bekannten Sensoren die Position eines beweglichen Empfängers zu berechnen.

Der theoretische Hintergrund des SLAM--Verfahrens wird in \cite{dissanayake2001solution} vorgestellt. Zusätzlich wird bewiesen das die Unsicherheit bei der Kartenerstellung und Lokalisierung eine untere Schranke erreicht.

\citeauthor{kantor2002preliminary} stellen in Ihrer Arbeit \cite{kantor2002preliminary} ein Lokalisierungsverfahren vor, welches die Roboterposition anhand von Entfernungsmessungen zu vorher bekannten Landmarken bestimmen kann. Im letzten Abschnitt wird SLAM--Verfahren vorgestellt, welches über einen Kalman--Filter die Unsicherheit der Landmarkenposition modellieren kann.

Die Autoren \citeauthor{blanco2008pure} gehen in ihren Arbeiten \cite{blanco2008pure, blanco2008efficient} einen Schritt weiter und bestimmen die unbekannte Roboterposition sowie die unbekannten Landmarkenpositionen. Hierzu nutzen Sie im ersten Schritt einen Partikelfilter (engl. Particle Filter) bis die Schätzung eine ausreichende Genauigkeit erreicht hat um dann im zweiten Schritt über einen Kalman--Filter ein Positionsverfolgung (engl. Position Tracking) durchzuführen.

Die Arbeit \cite{ledergerber2015robot} von \citeauthor{ledergerber2015robot} gehen auf die Roboterlokalisierung unter Verwendung einer One-Way Ultra-Wideband Kommunikation ein. Dieses hat den Vorteil, das mit sehr wenigen Landmarken eine große Anzahl von Roboter lokalisiert werden kann.

- The Cartesian EKF described above operates in the Cartesian space, we formulate our problem in polar coordinates.
- The use of this parameterization derives motivation from the polar coordinate system, where annuli, crescents and other ringlike shapes can be easily modeled. This parameterization is called Relative Over Parameterized (ROP) because it over parameterizes the state relative to an origin.

- EKF -> Polar EKF -> Multi-Hypothesis Filter
- Partikel Filter
	\begin{comment}
Fragestellung:
- Welche elektrische Beschaltung ist notwendig um das DWM1000 Modul von DecaWave in Betrieb nehmen zu können?
- Wie erfolgt die Entfernungsmessung zwischen den einzelnen UWB--Modulen?
- Wie erfolgt der Datenaustausch zwischen einem UWB--Modul und der Verarbeitungseinheit?
\end{comment}

\begin{comment}
------------------------------------------------------------------------------------------
\end{comment}
\chapter{Ultrabreitband}

\begin{comment}
- Was ist UWB überhaupt?
- Wie unterscheidet es sich von der bisherigen Verfahren?
- Auflisten einige Eigenschaften
\end{comment}

\begin{comment}
"C:\Users\Albert\Documents\Studium\Bachelor WS17\02133r1P802-15-WG_Ultra-Wideband-Tutorial.ppt"
=> Möglicherweise hilfreich!

Ultra wideband wireless positioning systems
	- \cite{yavari2014ultra}
	- The most important characteristic of UWB is large bandwidth in comparison with
prevalent narrow-band systems.
	- One result of the large bandwidth of UWB is that due to the inverse relatioship of
time and frequency, the life-time of UWB signals is very short. Consequently, the
time resolution of UWB signals is high and UWB is a good candidate for positioning
systems.

Ultra-wideband communications: an idea whose time has come
	- \cite{yang2004uwbcom}
	- UWB applications: short-range very high-speed broadband access to the Internet, covert communication links, localization at centimeter-level accuracy, high-resolution ground-penetrating radar, through-wall imaging, precision navigation and asset tracking, just to name a few.
	- UWB characterizes transmission systems with instantaneous spectral occupancy in excess of 500 MHz or a fractional bandwidth of more than 20%.
	- Such systems rely on ultra-short (nanosecond scale) waveforms that can be free of sine-wave carriers and do not require IF processing because they can operate at baseband. As information-bearing pulses with ultra-short duration have UWB spectral occupancy, UWB radios come with unique advantages that have long been appreciated by the radar and communications communities: i) enhanced capability to penetrate through obstacles; ii) ultra high precision ranging at the centimeter level; iii) potential for very high data rates along with a commensurate increase in user capacity; and iv) potentially small size and processing power.
	- This huge “new bandwidth” opens the door for an unprecedented number of bandwidth-demanding position-critical low-power applications in wireless communications, networking, radar imaging, and localization systems [64].
	- These include short-range, high-speed access to the Internet, accurate personnel and asset tracking for increased safety and security, precision navigation, imaging of steel reinforcement bars in concrete or pipes hidden inside walls, surveillance, and medical monitoring of the heart’s actual contractions.
	- For wireless communications in particular, the FCC regulated power levels are very low (below -41.3 dBm), which allows UWB technology to overlay already available services such as the global positioning system (GPS) and the IEEE 802.11 wireless local area networks (WLANs) that coexist in the 3.6--10.1 GHz band. Although UWB signals can propagate greater distances at higher power levels, current FCC regulations enable high-rate (above 110 MB/s) data transmissions over a short range (10--15 m) at very low power.
	- Wireless personal area networks (WPANs): Also known as in-home networks, WPANs address short-range (generally within 10--20 m) ad hoc connectivity among portable consumer electronic and communication devices. They are envisioned to provide high-quality real-time video and audio distribution, file exchange among storage systems, and cable replacement for home entertainment systems. UWB technology emerges as a promising physical layer candidate for WPANs, because it offers high-rates over short range, with low cost, high power efficiency, and low duty cycle.
	- Sensor networks: Sensor networks consist of a large number of nodes spread across a geographical area. The nodes can be static, if deployed for, e.g., avalanche monitoring and pollution tracking, or mobile, if equipped on soldiers, firemen, or robots in military and emergency response situations. Key requirements for sensor networks operating in challenging environments include low cost, low power, and multifunctionality. High data-rate UWB communication systems are well motivated for gathering and disseminating or exchanging a vast quantity of sensory data in a timely manner. Typically, energy is more limited in sensor networks than in WPANs because of the nature of the sensing devices and the difficulty in recharging their batteries. Studies have shown that current commercial Bluetooth devices are less suitable for sensor network applications because of their energy requirements [62] and higher expected cost [2]. In addition, exploiting the precise localization capability of UWB promises wireless sensor networks with improved positioning accuracy. This is especially useful when GPSs are not available, e.g., due to obstruction.
	- Imaging systems: Different from conventional radar systems where targets are typically considered as point scatterers, UWB radar pulses are shorter than the target dimensions. UWB reflections off the target exhibit not only changes in amplitude and time shift but also changes in the pulse shape. As a result, UWB waveforms exhibit pronounced sensitivity to scattering relative to conventional radar signals. This property has been readily adopted by radar systems (see e.g., [5] and references therein) and can be extended to additional applications, such as underground, through-wall and ocean imaging, as well as medical diagnostics and border surveillance devices [55], [57].
	- Vehicular radar systems: UWB-based sensing has the potential to improve the resolution of conventional proximity and motion sensors. Relying on the high ranging accuracy and target differentiation capability enabled by UWB, intelligent collision-avoidance and cruise-control systems can be envisioned. These systems can also improve airbag deployment and adapt suspension/braking systems depending on road conditions. UWB technology can also be integrated into vehicular entertainment and navigation systems by downloading high-rate data from airport off ramp, road-side, or gas station UWB transmitters.
		- By its rulemaking proposal in 2002, the Federal Communications Commission (FCC) in the United States essentially unleashed huge “new bandwidth’’ (3.6–10.1 GHz) at the noise floor, where UWB radios overlaying coexistent RF systems can operate using low-power ultra-short information bearing pulses.
	
\cite{win1998impulse}
	- Impulse radio: How it works
	
\cite{fontana2004recent}
	- Recent system applications of short-pulse ultra-wideband (UWB) technology	
	- In its infancy, UWB was commonly referred to as “carrier-free,” “baseband,” or “impulse,” reflecting the fact that the underlying signal generation strategy was the result of a broad-band extremely fast rise time, step, or impulse, which shock, or impulse, excited a wide-band antenna (e.g., TEM, mode horn).
	- The origins of UWB technology stem from work in time-domain electromagnetics begun in the early 1960s to fully describe the transient behavior of certain classes of microwave networks by examining their characteristic impulse response [7]–[12].
	- For up until 1962, there were no convenient means to observe, let alone measure, waveforms having subnanosecond durations, as were required to suitably approximate an ideal impulsive excitation. Fortuitously, at about the same time [15], Hewlett-Packard introduced the time-domain sampling oscilloscope, which greatly facilitated these measurements.
	- The last element that needed to be developed before real system development could begin was the short-pulse, or threshold, receiver. In the early 1970s, both avalanche transistor and tunnel diode detectors were constructed in attempts to detect these very short duration signals. The tunnel diode, invented in 1957 by Esaki who would later receive the Nobel Prize in physics in 1973 for this accomplishment, was the first known practical application of quantum physics. This unique device, with its extremely wide bandwidth (at the time, tens of gigahertz) permitted not only subnanosecond pulse generation essential for impulse excitation, but also could be used as a sensitive thresholding device for the detection of short-pulse waveforms.
	
Medical applications of ultra-wideband (UWB)
	- \cite{pan2007medical}
	
\cite{lakkundi2006ultra}
	- Ultra wideband communications: History, evolution and emergence
	- Figure 1 und 2 sind gut
	- The development of the sampling oscilloscope in the early 1960s and the corresponding techniques for generating sub-nanosecond baseband pulses speed up the development of UWB [1].
	
\cite{aiello2006ultra}
	- Ultra wideband systems: technologies and applications
	- The FCC's definition of the criteria for devices operating in the UWB spectrum purposely did not specify the techniques related to the generation and detection of RF energy; rather, it mandated compliance with emission limits that would enable coexistence and minimize the threat of harmful interference with legacy systems, thus protecting the Global Positioning System (GPS), satellite receivers, cellular systems, and others.
	- Interest in the technology has been steady, with more than 200 technical papers published in journals between 1960 and 1999 on the topic and more than 100 U.S. patents issued on UWB or UWB-related technology[3].
\end{comment}


\begin{comment}
------------------------------------------------------------------------------------------
\end{comment}
\section{Historie}

Als Vater der \ac{uwb} Kommunikation kann der italienische Funkpionier Guglielmo Marconi angesehen werden. In den späten 1890er Jahren entwickelte er den Knallfunkensender, der über eine Funkenstrecke ein hochfrequentes Signal zur Übertragung von Morsezeichen erzeugt. Mit dieser Apparatur gelang es Ihm, im Jahre 1901 einen Nachrichtenaustausch zwischen Nordamerika und Europa über den Nordatlantik durchzuführen.\cite{fontana2004recent}

% Weitere Quellen:
% http://www.ieee.ca/millennium/radio/radio_differences.html
% https://de.wikipedia.org/wiki/Knallfunkensender

Bis in die Anfänge der 1960er Jahre dominierte jedoch die sinusförmige Funkübertragungsform. Dies änderte sich als die Forscher vom \ac{llnl} und \ac{lanl} begangen die Ausbreitung elektromagnetischer Wellen nicht zur im Frequenz- sondern auch im Zeitbereich zu untersuchen. Grundlegende Erkenntnisse wurden dabei im Bereich der Impulssender, -empfänger und -antennen gesammelt.\cite{eltaher2004positioning, fontana2004recent, lakkundi2006ultra, aiello2006ultra}

% TODO: time-domain sampling oscilloscopes
Durch die Einführung der zeitbereichs basierten Abtast--Oszilloskope im Jahre 1962 durch Tektronix bzw. Hewlett-Packard war es zum ersten Mal möglich eine \ac{uwb} Wellenform aufzufangen und anzuzeigen. Ermöglicht wurde dies erst durch den Einsatz von Tunneldioden und Avalanchetransistoren. \cite{fontana2004recent, lakkundi2006ultra, aiello2006ultra}

Ab dem Jahre 1964 produzierten beide Hersteller Messgeräte für die Diagnose im Zeitbereich. \cite{barrett2001technical}

Ab den Anfängen der 1970er Jahre waren alle wichtigen Grundsteine für ein \ac{uwb} System für Kommunikation- bzw. Radaranwendungen gelegt. Dazu zählten auch diverse eingereichte Patente von Harmuth an der \ac{cua}, Ross und Robbins bei der Sperry Rand Corporation und Paul van Etten an der \ac{usaf} im Rome Air Development Center.\cite{barrett2001technical, fontana2004recent, yang2004uwbcom} Hervorzuheben ist das eingereichte Patent von Ross im Jahre 1973, siehe \cite{g1973transmission}.

% INFO:
% Rome Laboratory (Rome Air Development Center until 1991) is the US
% "Air Force 'superlab' for command, control, and communications"[4] research and
% development and is responsible for planning and executing the USAF science and
% technology program.

% TODO: Warum ist gerade diese Patent hervorzuheben?
% US 3728632 A
% Transmission and reception system for generating and receiving base-band pulse
% duration pulse signals without distortion for short base-band communication system
% Veröffentlichungsnummer: US3728632 A
% Publikationstyp: Erteilung
% Veröffentlichungsdatum	: 17. Apr. 1973
% Eingetragen: 12. März 1971
% Erfinder:	Ross G
% Ursprünglich Bevollmächtigter: Sperry Rand Corp
% ZUSAMMENFASSUNG
% An electromagnetic signal communication system utilizing short base-band pulse
% signals of sub-nanosecond duration employs dispersionless, broad band antenna
% transmission line elements for generating and preserving the character of the
% short base-band pulses in respective transmitter and receiver sub-systems.

Kurz darauf im Jahre 1974 wurde die \ac{uwb} Technologie kommerziell erfolgreich von Morey bei der \ac{gssi} für ein Bodenradar (engl. \acf{gpr}) angewendet. \cite{barrett2001technical}

Im Zeittraum von 1977 is 1989 wurden mehrere Programme und Workshops organisiert um die Entwicklung von \ac{uwb} Systemen voranzutreiben, darunter auch bei der \ac{usaf} und dem \ac{usdod}. Ebenfalls gabe es mehrere akademische Programme an diversen Instituten, darunter auch am \ac{llnl}, \ac{lanl}, University of Michigan, University of Rochester und
Polytechnic University, mit dem Fokus auf den physikalischen Unterschieden zwischen der Kurzimpulsübertragung und den Langimpulssignalen bzw. kontinuierlichen Impulssignalen bei der Interaktion mit verschiedenen Materialien.\cite{barrett2001technical}

% TODO: Der letzte Satz ist zu lang. Aufspalten!

Ab dem Jahre 1989 wurde der Name \ac{uwb} durch das \ac{usdod} geprägt. Diese Definition galt für alle Geräte die mindestens eine Bandbreite von \SI{1.5}{\GHz} bzw. \SI{25}{\percent} der \ac{fbw} belegten. Vorher war die \ac{uwb} Technologie nur unter den Synonymen ``baseband communication'', ``carrier free communication'', ``impulse radio'', ``large relative bandwidth communication'', ``nonsinusoidal communication'', ``orthogonal functions'', ``sequency theory'', ``time domain'', ``large-relative-bandwidth radio/radar signals'', ``video-pulse transmission'' und/oder ``Walsh waves communication'' bekannt. \cite{eltaher2004positioning, fowler1990assessment, yang2004uwbcom, aiello2006ultra, fontana2004recent}

% TODO: Fachbegriffe Übersetzen?

% TODO: Sollen die Militärischen Anwendungfälle erwähnt werden? Vielleicht besser bei der FCC regulierung.
% Das Militär hatte dabei für sich die Anwendungsfälle im Bereich Radar und hochsicherheitskommunkation entdeckt.\cite{eltaher2004positioning}

Im Jahre 1994 wurde von McEwan an der \ac{llnl} das \ac{mir} konstruiert. Hierbei handelte es sich um ein \ac{uwb} Radarsystem mit bemerkenswerten Eigenschaften. Das Radarsystem verfügte über eine sehr hohe Signalsensitivität, einen kompakten Aufbau, eine kostengünstige Herstellung und einen der geringer Energieverbrauch, der sich im Bereich von Mikrowatt befanden und daher ideal für batteriebetriebene Anwendung eignete. \cite{barrett2001technical}

% TODO: Zwei mal bemerkenswerte Eigenschaften. Korrigieren bzw. Umformulieren!!!

Vor dem Jahre 2002 war die Verwendung von \ac{uwb} auf Radarssytem beschränkt, die größtenteils in militärischen Anwendungen aufzufinden waren. \cite{yang2004uwbcom} Das änderte sich ab dem Jahre 1998, als die \ac{fcc} mit der Standardisierung der \ac{uwb} Nutzung begann. Im Jahre 2002 wurden durch die \ac{fcc} in den Vereinigten Staten von Amerika große Frequenzbereiche (\SIrange{3.6}{10.1}{\GHz}) für die kommerzielle Nutzung freigegeben hat, siehe First Report and Order (R\&O). Danach wurden erstmals auch eine nicht militärische Anwendungen im Bereich ``Imaging systems'', ``communication and measurement systems'' und ``vehicular radar systems'' möglich. \cite{yang2004uwbcom}

% TODO: Zitieren des FCC R&O
% [12] FCC First Report and Order: In the matter of Revision of Part 15 of the
% Commission’s Rules Regarding Ultra-Wideband Transmission Systems,
% FCC 02–48, April 2002.
% Chrome: g FCC 02-48 bibtex
% https://transition.fcc.gov/Bureaus/Engineering_Technology/Orders/2002/fcc02048.pdf

Weitere Staten folgten der \ac{fcc} Regulierung/Standardierung und gaben ebenfalls große Frequenzbereiche für die \ac{uwb} Technologie frei. Details zu den Regularien der einzelnen Staten können unter \cite{decawave2015uwbreg} eingesehen werden.

	
\begin{comment}
------------------------------------------------------------------------------------------
\end{comment}
\section{Alternative Technologien}

\begin{comment}
Welche alternativen Technologien gibt es zu UWB?
\end{comment}

Einen guten Überblick über die Eigenschaften der Drahtlosen-Protokolle (engl. Wireless Protocols) Bluetooth, UWB, ZigBee und WiFi liefert die Arbeit \cite{lee2007comparative} von \citeauthor{lee2007comparative}.

qigao2015tightly - Tightly Coupled Model for Indoor Positioning based on UWB/INS


\begin{comment}
------------------------------------------------------------------------------------------
\end{comment}
\section{Gegenüberstellung}

\begin{comment}
Welche Eigenschaften haben die alternativen Technologien?
Warum hab ich mich für UWB entschieden?
\end{comment}


\begin{comment}
------------------------------------------------------------------------------------------
\end{comment}
\section{Erstelle Hardware}

\begin{comment}
------------------------------------------------------------------------------------------
- Datenübertragung zum Host
- Batteriebetrieb
- TODO: Erweiterbare Hardwareplattform
\end{comment}
\subsection{Anforderungen}

An die zu erstellende Hardware werden mehrere Anforderungen gestellt.

Um eine Entfernungsmessung durchzuführen wird immer ein Marker und mindestens ein Anker benötigt. Sowohl der Marker als auch der Anker sollen aus den gleichen elektrischen Komponenten bestehen, also eine gemeinsame Harewareplattform bilden. Die unterschiedliche Funktionalität pro Modul soll sich dann aus verschiedenen Software-Ständen der Firmware herausbilden.

Die Anker sollen im Bedarfsfall frei im Raum verteilt werden können. Nicht an jeder Stelle steht eine Stromversorgung zur Verfügung, daher muss jedes Modul über eine separate Energiequelle verfügen.

Zusätzlich muss der Marker über eine bidirektional Kommunikationsschnittstelle zur Verarbeitungseinheit verfügen. Über diese sollen zum einen Steuerbefehle an das UWB--Modul geschickt werden und zum anderen sollen die gemessenen Entfernungen zwischen dem Marker und den Ankern an die Verarbeitungsanleitung übertragen werden.


\begin{comment}
------------------------------------------------------------------------------------------
\end{comment}
\subsection{Hardware Zusammenstellung}


\begin{comment}
------------------------------------------------------------------------------------------
\end{comment}
\subsubsection{\ac{uwb}--Transceiver}
\label{subsec:uwb_transceiver}

Als \ac{uwb}--Transceiver werden die Kompontenten der Firma DecaWave verwendet. Bei dem DW1000 handelt es sich nur um dem \ac{ic} der für die Erzeugen und Verarbeiten der \ac{uwb}--Funksignale zuständig ist. Der DWM1000 beinhaltet neben dem DW1000 auch die notwendige Beschaltung und zusätzlich eine Antenne für die Übertragung, siehe \figurename~\ref{fig:pin_assignment}.

\begin{figure}
	\begin{subfigure}[t]{0.4\textwidth}
		\includegraphics[width=\textwidth]{dw1000_pin_assignments.png}
		\caption{DW1000 \ac{ic}}
		\label{fig:dw1000_pin_assignments}
		\source{\cite{decawave2016dw1kdatasheet}}
	\end{subfigure}
	\hfill
	\begin{subfigure}[t]{0.4\textwidth}
		\includegraphics[width=\textwidth]{dwm1000_pin_assignments.png}
		\caption{DWM1000 Modul}
		\label{fig:dwm1000_pin_assignments}
		\source{\cite{decawave2016dwm1kdatasheet}}
	\end{subfigure}
	\caption{DecaWave \ac{ic} Pin Belegung}
	\label{fig:pin_assignment}
\end{figure}

Der DWM1000 kann mit einer Spannung von \SIrange{2.8}{3.6}{\volt}\cite{decawave2016dwm1kdatasheet} betrieben werden, idealerweise mit \SI{3.3}{\volt}. Das bedeutet aber auch, dass die Logikpegelspannung für die \ac{spi} Schnittstelle \SI{3.3}{\volt} beträgt. Dieser Umstand muss bei der Auswahl des Mikrocontrollers berücksichtig werden.

Die Kommunikation mit dem DWM1000 erfolgt über die \ac{spi} Schnittstelle, hierfür sind die Pins \ac{sclk}, \ac{mosi}, \ac{miso} und \ac{ss} zu verwenden \cite{decawave2016dwm1kdatasheet}. Bei der \ac{spi}--Schnittstelle handelt es sich um eine Master-Slave Architektur, das bedeutet das Daten vom Master gesendet und angefragt werden können. Der Slave kann jedoch nur Daten auf Anfrage senden. Um zu verhindern, das der Master periodisch auf das Eintreffen einer Nachrichten anfragen muss, kann der \ac{irq}--Pin des Slaves verwendet werden. Um zu verhindern das kurzfristige Spannungsspitzen einen Interrupt auslösen, muss der \ac{irq}--Pin über einen Pulldown--Widerstand auf Masse gezogen werden.

Um das DWM1000 erfolgreich zu initalisieren muss zusätzlich der RSTn--Pin durch den Mikrocontroller angesteuert werden. Zusätzlich ergibt über die Beschaltung dieses Pins die Möglichkeit den DWM1000 per Hardware im laufenden Betrieb neuzustarten.

Zusätzliche Informationen, wie der Versand und Empfang von Nachrichten, könnten über Status--Leuchtdioden ausgegeben werden. Hierfür wird jeder der Pins GPIO1 bis GPIO3 jeweils mit einem Vorwiderstand und einer Leuchtdiode verbunden.

%TODO: Wie wird der Vorwiderstand berechnet? ca. 10mA bei 1.8V, R=(U_0-U_LED)/I_LED


\begin{comment}
------------------------------------------------------------------------------------------
\end{comment}
\subsubsection{Mikrocontroller}

Wie bereits im vorherigen Abschnitt~\ref{subsec:uwb_transceiver} erwähnt beträgt die Logikpegelspannung \SI{3.3}{\volt}. Durch diesen Umstand entfallen alle Mikrocontroller die mit einer \SI{5}{\volt} Versorgungsspannung, wie z.B. der beliebte Arduino Uno, betrieben werden. Die Entscheidung viel auf den Pro Trinket der Firma Adafruit, der als Hauptprozessor den Atmel ATmega328/P verwendet. Dieser hat den Vorteil, dass er jeweils in einer \SI{5}{\volt} und \SI{3.3}{\volt} Variante existiert. Zusätzlich ist die \SI{3.3}{\volt} Variante mit einem Systemtakt von \SI{12}{\MHz} schneller als der vergleichbare Arduino Pro Mini \SI{3.3}{\volt} der nur mit \SI{8}{\MHz} getaktet ist.

%\begin{wrapfigure}{r}{0.5\textwidth}
\begin{figure}
	\centering
	\includegraphics[width=0.5\textwidth]{adafruit_pro_trinket_5v.png}
	\caption[Adafruit Pro Trinket]{Adafruit Pro Trinket\protect\footnotemark}
	\label{fig:adafruit_pro_trinket}
	\source{\url{https://learn.adafruit.com/introducing-pro-trinket/pinouts}}
\end{figure}

\footnotetext{Der Adafruit Pro Trinket \SI{3.3}{\volt} ist zum Großteil pinkompatibel zu der \SI{5}{\volt} Variante. Nur der Pin \textit{BAT+} benötigte eine Batteriespannung von \SIrange{3.5}{16}{\volt} und der drei Reihen weiter unten liegende \SI{5}{\volt} Pin liefert nur \SI{3.3}{\volt}.}

Um eine Kommunikationsverbindung zwischem dem DWM1000 und dem Mikrocontroller herzustellen, müssen die Pins anhand der \tablename~\ref{tab:pin_assignment_between_dwm1k_and_pro_trinket} verbunden werden.

\begin{table}
	\centering
	\begin{tabular}{||c|c|c||} 
		\hline
		DWM1000 (Pin)&Pro Trinket (Pin)&Bedeutung\\\hline
		\hline
		SPICLK (20)&SCK (13)&SPI\\\hline
		SPIMISO (19)&MISO (12)&SPI\\\hline
		SPIMOSI (18)&MOSI (11)&SPI\\\hline
		SPICSn (17)&SS (10)&SPI\\\hline
		\hline
		IRQ (22)&INT1 (3)&Interrupt\\\hline
		\hline
		RSTn (3)&PB1 (9)&Hardware Reset\\\hline
	\end{tabular}
	\caption{Pinbelegung zwischen dem DWM1000 und Pro Trinket.}
	\label{tab:pin_assignment_between_dwm1k_and_pro_trinket}
\end{table}


\begin{comment}
% Lithium Ion Cylindrical Battery - 3.7v 2200mAh
% https://www.adafruit.com/product/1781
% Lithium Ion Polymer Battery - 3.7v 2500mAh
% https://www.adafruit.com/product/328
% Adafruit Pro Trinket LiPoly/LiIon Backpack
% https://learn.adafruit.com/adafruit-pro-trinket-lipoly-slash-liion-backpack?view=all
------------------------------------------------------------------------------------------
\end{comment}
\subsubsection{Energieversorgung}

Um den \ac{uwb}--Transceiver und den Pro Trinket mit Energie zu versorgen wird ein Lithiumionenakku mit einer Spannung von \SI{3.7}{\volt} und einer Kapazität von \SI{2200}{\mAh} verwendet. Die Verbindung zwischen den beiden wird über einen Lithiumionenakku Lade--Chip hergestellt. Diesen gibt es als fertiges Modul von Adafruit mit der Bezeichnung Pro Trinket LiPoly/LiIon Backpack.

Bevor jedoch dieses Modul eingesetz werden kann, müssen noch zwei Modifikationen durchgeführt werden. Zum einen kann die Energiequelle mittels eines Schalters vom Verbraucher getrennt werden. Per Standard sind jedoch diese zwei Pins mit einander verbunden und müssen mit einem schwarfen Messer unterbrochen werden, siehe \figurename~\ref{fig:pro_trinket_liion_backpack_top}. Zum anderen wird der Lithiumionakku nur mit einem Strom von \SI{100}{\mA} geladen. Bei einer Kapazität von \SI{2200}{\mAh} würde ein vollständiger Ladezyklus ca. \SI{22}{\hour} dauern. Um diese Zeit zu verkürzen, müssen die zwei Lötpads, siehe \figurename~\ref{fig:pro_trinket_liion_backpack_bottom}, miteinander verbunden werden. Danach wird der Lithiumionakku mit einem Strom von \SI{500}{\mA} geladen und demenstrechend verkürzt sich die Ladedauer auch auf ca. \SI{2.5}{\hour}.
%TODO: 2.5 Stunden? Häh? Wie wäre es mit 2 1/2, laut Duden korrekt.

\begin{figure}
	\centering
	\begin{subfigure}[t]{0.4\textwidth}
		\includegraphics[width=\textwidth]{adafruit_lipoly_backpad_top_with_marker}
		\caption{Modifikation für den Schalter.}
		\label{fig:pro_trinket_liion_backpack_top}
	\end{subfigure}
	\qquad
	\begin{subfigure}[t]{0.4\textwidth}
		\includegraphics[width=\textwidth]{adafruit_lipoly_backpad_back_with_marker}
		\caption{Modifikation für einen höheren Ladestrom.}
		\label{fig:pro_trinket_liion_backpack_bottom}
	\end{subfigure}
	\caption{Adafruit Pro Trinket LiPoly/LiIon Backpack}
	\source{\url{https://learn.adafruit.com/adafruit-pro-trinket-lipoly-slash-liion-backpack}}
	\label{fig:pro_trinket_liion_backpack}
\end{figure}
%TODO: URL ist zu lang und wird nicht umgebrochen.

Um eine Verbindung zwischem dem LiIon Backpack und dem Pro Trinket herzustellen, müssen die Pins anhand der \tablename~\ref{tab:pin_assignment_between_liion_backpack_and_pro_trinket} verbunden werden.

\begin{table}
	\centering
	\begin{tabular}{||c|c|c||} 
		\hline
		LiIon Backpack&Pro Trinket&Bedeutung\\\hline
		\hline
		BAT&BAT+&Batteriespannung\\\hline
		5V&BUS&Ladespannung\\\hline
		G&GND&Masse\\\hline
		\hline
		SW1&&Schalter\\\hline
		SW2&&Schalter\\\hline
	\end{tabular}
	\caption{Pinbelegung zwischen dem LiIon Backpack und dem Pro Trinket.}
	\label{tab:pin_assignment_between_liion_backpack_and_pro_trinket}
\end{table}


\begin{comment}
- FTDI {Data Transfer}
	- Wofür?
	- Warum separat?
	- USB to Serial
	
	- 
	- 
	
% Adafruit CP2104 Friend - USB to Serial Converter
% https://www.adafruit.com/product/3309
% https://www.silabs.com/documents/public/data-sheets/cp2104.pdf
% Universal Asynchronous Receiver Transmitter
% https://de.wikipedia.org/wiki/Universal_Asynchronous_Receiver_Transmitter
------------------------------------------------------------------------------------------
\end{comment}
\subsubsection{Datenaustausch}

% TODO: Computer? PC? Verarbeitungseinheit?
Der ATmega328/P verfügt nicht über einen eingebauten USB--Controller, daher ist ein direkter Datenaustausch zwischen dem Mikrocontroller und einem Computer nicht möglich. Jedoch verfügt der ATmega328/P über eine \ac{uart}--Schnittstelle, mit der Daten seriell über die Leitungen RX und TX übertragen und empfangen werden können. Mittels einem zusätzlichen Modul kann diesen Datenstrom aufgefangen und über die USB--Schnittstelle übertragen werden. Das Adafruit CP2104 Friend erledigt genau diese Aufgabe. Angeschlossen wird es über den  \ac{ftdi}--Header, siehe \figurename~\ref{fig:adafruit_pro_trinket}. Dadurch ist es möglich die Module die einen Datenaustausch benötigen mit einem entsprechenden Modul auszurüsten.


% GND & GND & Masse\\\hline
% CTS & GND & Masse\\\hline
% 5V & 5V & Versorgungsspannung\\\hline
% TXD & RXD & Datenaustausch\\\hline
% RXD & TXD & Datenaustausch\\\hline
% RTS & RTS & Reset\\\hline







\begin{comment}

Zusätzlich werden Erfahrungsberichte aus dem Internet ausgewertet um die Beschaltung weiter zu verfeinern, siehe \cite{Trojer2015, Holder2016, Holder2016a}.
	- Kosten für den Aufbau
	- Schaltplan-Skizze
		- Besonderheiten (NetLabels)
		- SVG/PNG/PDF-Export
		- Gruppierung nach Funktionsgruppen
\subsection{Prototypen}
	- 1. Prototyp Aufbau auf einem Breadboard
		- UWB-Adapter von ...?
		- SMD Löttechnik
		- Funktionstest
		- Skript als Anhang
	- 2. Prototyp Aufbau auf einem Lochstreifen
		- Kommunikations- und Entfernungsmessungstest
	- Der initiale Aufbau erfolgt zu Evaluationszwecken auf einem Steckboard und zusätzlich auf einer separaten Lochstreifenplatine um das Zusammenspiel zweier UWB--Module zu testen. Nach dem erfolgreichen Systemtest wird aus dem erstellten Schaltplan, ein PCB--Layout erstellt, mehrere PCB--Boards bestellt und nach der Lieferung zusammengebaut und noch mal getestet.
\subsection{Platinen-Design}
	- Antenne
	- Aufrecht stehend
	- Batterie auf der Rückseite bietet stabilität
		- Flachere Akkus können auch verwendet werden
	- Ansteckbares FTDI
	- Preis pro Platinengröße
	- Footprint of DWM1000
	- AutoRoute
	- TODO: Ground Fill with Copper
\subsection{Steuersoftware}
	- Klassendiagramme der wichtigsten Elemente
	- Basisscript
	- Ranging (Verfahren)
	- Datenaustausch zwischen Host und µC
\subsection{Entfernungsmessung und Auswertung}
	- Versuchsaufbau
	-
\subsection{Kalibierung}
	- Kalibierung nach DecaWave
		- FlowCharts erklären
		- Ergebnisse auswerten
	- Wo liegen die Problem
	- Script im Anhang
\end{comment}


\begin{comment}
------------------------------------------------------------------------------------------
\end{comment}
\subsection{Elektrischer Aufbau}

\begin{comment}
Wie lange halten die Batterien durch?
	- 4.7 m, LOS, Start 13:50-23:50, 12:50-20:00 => 10+7 => 17 Stunden
	
LED Vorwiderstand berechnen
	- https://www.youtube.com/watch?v=iNZj91TSRUg
	- DW1000 Datasheet - 5.9 General Purpose Input Output (GPIO)
\end{comment}


\begin{comment}
------------------------------------------------------------------------------------------
\end{comment}
\subsection{Platinendesign}

\begin{comment}
\end{comment}


\begin{comment}
------------------------------------------------------------------------------------------
\end{comment}
\subsection{Steuersoftware}


\begin{comment}
------------------------------------------------------------------------------------------
\end{comment}
\subsection{Entfernungsmessung und Auswertung}

\begin{comment}
- Mit welchen Einstellungen kommt man auf die Entfernungsmessung?
- Streuung?
- LOS/NLOS {Holz, Bücher, Menschlicher Körper}
	- Welcher Fehler ergibt zwischen LOS/NLOS?
- Wie verändert sich die Genauigkeit der Entfernungsmessung bei einer direkten Sichtverbindung (engl. Line--of--sight (LOS)) und indirekten Sichtverbindung (engl. Non--line--of--sight (NLOS))?
\end{comment}

Der Überblick und Vergleich der verschiedenen Abstandsbestimmungsverfahren erfolgt über eine klassische Literatursuche, siehe \cite{lee2007comparative, herranz2010studying, zekavat2011handbook}.


isaacs2009optimal - Optimal sensor placement for time difference of arrival localization


\begin{comment}
------------------------------------------------------------------------------------------
\end{comment}
\subsection{Kalibrierung}

\begin{comment}
- Kalibierungsalgorithmus nach decaWave
	- Hab ich den Überhaupt richtig implementiert?
- Kalibierung über die Anpassung der einer Antennen Delay für alle.
- Kann über die Kalibrierung der Antennenverzögerung eine genauere Entfernungsmessung erreicht werden?
\end{comment}


Das Verfahren zur Kalibrierung der Antennenverzögerung kann ebenfalls der Hersteller--Dokumentation entnommen werden, siehe \cite{decawave2014calibration}. Hierfür muss ein Versuchsaufbau erstellt werden. Zusätzlich wird eine Anpassung der Steuer--/Auswerte-Software notwendig, um die Verzögerung zu berechnen.

Die Genauigkeitsbestimmung der Entfernungsmessung mit LOS und NLOS wird über einen Versuchsaufbau realisiert. Hierfür werden mehrere Messreihen in verschiedenen Abständen aufgenommen und mit der tatsächlichen Entfernung verglichen.
	\begin{comment}
------------------------------------------------------------------------------------------
- \cite{kurth2003experimental}
	- Numerically, we can evaluate the performance of the dead reckoning and Kalman lter localization methods by considering the cross-track error (XTE). That is, for each pose we measure how far left or right of the true position our estimation is, orthogonal to the true heading. We compile these errors for every point along the path, then nd the maximum value along with the mean and standard deviation of the errors to produce the evaluative statistics in Table 1.
	- https://de.wikipedia.org/wiki/Querabweichung
- Diagramme
	- \cite{kurth2003experimental}
		- Fig. 5: (1) The ground truth path with tags indicated by circles. The numbers indicate how many range measurements were received from each tag over the duration of Test 1. (2) The path estimate from dead reckoning alone. (3) The path estimate from localization using a Kalman lter. The lter fuses data from odometry and a gyro with absolute measurements from RF tags to produce this path estimate. Numerical results are given in Table 1. (X: position in x(m), Y: position in y(m), Ground truth path with tag locations, Dead reckoning path, Kalman filter localization path)
\end{comment}
\chapter{RO-SLAM}\label{ch:ro_slam}


\begin{comment}
------------------------------------------------------------------------------------------
\end{comment}
\section{Roboterplattform [todo]}


\begin{comment}
------------------------------------------------------------------------------------------
\end{comment}
\section{Softwarearchitektur [todo]}


\begin{comment}
------------------------------------------------------------------------------------------
\end{comment}
\subsection{ROS Module [todo]}


\begin{comment}
------------------------------------------------------------------------------------------
\end{comment}
\subsubsection{hector\_trajectory\_server}

Mit diesem Modul ist es möglich die gefahrene Trajektorie eine Roboters in einem bestimmten Koordinatensystem auszugeben. Hierfür muss im TF--Tree eine Verbindung zwischen dem \textit{source\_frame\_name}(base\_link) und dem \textit{target\_frame\_name}(odom oder map) bestehen. Die Trajektorie bezieht sich mit ihren Koordinaten auf das target\_frame und ist vom Datentyp nav\_msgs/Path.

Über einen Service lässt die sich Trajektorie auch abfragen: \textit{rosservice call /trajectory}

\url{http://wiki.ros.org/hector_trajectory_server}

\url{http://docs.ros.org/api/nav_msgs/html/msg/Path.html}

\begin{lstlisting}[
	frame=shadowbox,
	breaklines=true,
	caption={Konfiguration der hector\_trajectory\_server--Nodes.},
	captionpos=b,
	label={lst:hector_trajectory_server_node},
	columns=fullflexible,
	language=XML,r
	numbers=none,
	float,
]
<node 
  name="hector_trajectory_server"
  pkg="hector_trajectory_server"
  type="hector_trajectory_server"
  output="screen">

  <param name="target_frame_name" value="map"/>
  <param name="source_frame_name" value="base_link" />
  <param name="trajectory_update_rate" value="10.0" />
  <param name="trajectory_publish_rate" value="10"/>
</node>
\end{lstlisting}


\begin{comment}
------------------------------------------------------------------------------------------
\end{comment}
\subsubsection{rf2o\_laser\_odometry}

\url{http://wiki.ros.org/rf2o}


\begin{comment}
------------------------------------------------------------------------------------------
\end{comment}
\subsubsection{robotino\_node}


\begin{comment}
------------------------------------------------------------------------------------------
\end{comment}
\subsubsection{robotino\_odometry\_node}

Sorgt dafür, dass die Odometry--Nachrichten vom Robotino ins ROS--System veröffentlicht werden.


\begin{comment}
------------------------------------------------------------------------------------------
\end{comment}
\subsubsection{Vergleich der Trajektorie von Odom und rf2o}

- Trajektorie der Robotino--Odometry bestimmen.
- Herausfiltern der Odometry Nachrichten aus den Bag--Dateien
- Odometry aus den Laser--Scans bestimmen und Trajektorie aufzeichen.
- Trajektorien vergleichen




\begin{comment}
------------------------------------------------------------------------------------------
\end{comment}
\subsection{MRPT Module [todo]}
	\begin{comment}
--------------------------------------------------------------------------------
\end{comment}
\chapter{Evaluation}


\begin{comment}
--------------------------------------------------------------------------------
\section{Versuchsaufbau [todo]}
\section{Ergebnisse und Auswertung [todo]}
\end{comment}


\begin{comment}
--------------------------------------------------------------------------------
- TODO: Stimmen die 10Hz?
- Start 13:50-23:50, 12:50-20:00 => 10+7 => 17 Stunden
\end{comment}
\section{Batterielaufzeit}

Beim Test der Batterielaufzeit wurden zwei \gls{uwbm} in einem Abstand von \SI{4.7}{\metre} aufgestellt. Beide \gls{uwbm} hatten eine direkte \gls{los} zu einandern. Über den kompletten Zeitraum wurden Entfernungsmessungen mit einer Rate von \SI{10}{\hertz} durchgeführt. Als Testprogramme wurden dabei \textit{DW1000Ranging\_ANCHOR} und \textit{DW1000Ranging\_TAG} aus dem GitHub--Projekt \cite{Trojer2015} verwendet.

Der \Gls{anchor} hatte nach ca. \SI{17}{\hour} seinen Dienst eingestellt, wenig später folgte Ihm der \Gls{tag}. Deutlich höhere Batterielaufzeiten können dadurch erzielt werden, dass die Senderate reduziert wird und die Stromsparfunktionen sowohl des DWM1000 als auch des ATmega328/P genutzt werden.


\begin{comment}
--------------------------------------------------------------------------------
- Mit welchen Einstellungen kommt man auf die Entfernungsmessung?
- Streuung?
- LOS/NLOS {Holz, Bücher, Menschlicher Körper}
	- Welcher Fehler ergibt zwischen LOS/NLOS?
- Wie verändert sich die Genauigkeit der Entfernungsmessung bei einer direkten Sichtverbindung (engl. Line--of--sight (LOS)) und indirekten Sichtverbindung (engl. Non--line--of--sight (NLOS))?
- isaacs2009optimal - Optimal sensor placement for time difference of arrival localization
- Diagramme
	- \cite{kurth2003experimental}
		- Fig. 2: Sample PDFs showing the true ranges associated with 20, 30, and 50 ft measured ranges. (X: true range, Y:count)
		- Fig. 3: The mean true distances to RF tags vs. measured distances (X:measured range, Y: true range)
		- Fig. 4: The variance in true distances to RF tags vs. measured distances (X:measured range (ft), Y: variance (ft^2))
	
- https://matheguru.com/stochastik/standardfehler.html
- https://de.wikipedia.org/wiki/Standardfehler
	
\end{comment}
\section{Entfernungsmessung}

Um die Charakteristik der Entfernungsmessung zu bestimmen, wurde der Versuchsaufbau aus der \figurename~\ref{fig:entfernungsmessung_versuchsaufbau} verwendet. Dabei wird der \Gls{tag} an einem fixen Ort befestigt und die Entfernung zu dem \Gls{anchor} gemessen. Es wurden dabei acht Entfernungen mit einem Abstand von einem Meter gemessen. Zu jeder Entfernung wurden \num{249} Messungen aufgezeichnet. Die tatsächliche Entfernung wird mit einem Laser Entfernungsmesser, der eine Genauigkeit von $\pm$~\SI{2}{\milli\meter} besitzt, bestimmt.

\begin{figure}[ht!]
  \centering
  \includegraphics[width=0.5\linewidth]{entfernungsmessung_versuchsaufbau}
	\caption{Versuchsaufbau der Entfernungsmessung.}
	\label{fig:entfernungsmessung_versuchsaufbau}
\end{figure}

Die Ergebnisse der Entfernungsmessung können der \tablename~\ref{tab:entfernungsmessung_stochastik} entnommen werden. Auffällig sind die zum Teil großen Abweichung der Mittelwerte von der tatsächlichen Entfernungen, siehe Entfernung \SI{3}{\meter} und \SI{7}{\meter}. Mit \SI{10}{\centi\meter} sind die größten Ausreißer vom Mittelwert bei \SI{7}{\meter} zu verzeichnen. Die restlichen liegen im Bereich von \SIrange{4}{8}{\centi\meter}. Die Standardabweichung liegt mit \SI{3}{\centi\meter} in einem sehr guten Bereich, siehe auch \figurename~\ref{fig:entfernungsmessung_punktwolke}.

\begin{table}[h!]
	\centering
	\begin{tabular}{||c||c|c|c|c|c|c||}
		\hline
		Entfernung [\si{\meter}] & $\overline{x}_{arithm}$ & $\sigma$ & $\sigma^2$ & $SE_{\overline{x}}$ & Min & Max\\\hline
		\hline
		\num{1.00} & \num{1.0401} & \num{0.0298} & \num{0.0009} & \num{0.0019} & \num{0.96} & \num{1.12}\\\hline
		\num{2.00} & \num{2.0766} & \num{0.0164} & \num{0.0003} & \num{0.0010} & \num{2.03} & \num{2.12}\\\hline
		\num{3.00} & \num{3.1288} & \num{0.0218} & \num{0.0005} & \num{0.0014} & \num{3.07} & \num{3.18}\\\hline
		\num{4.00} & \num{3.9104} & \num{0.0221} & \num{0.0005} & \num{0.0014} & \num{3.86} & \num{3.97}\\\hline
		\num{5.00} & \num{5.0746} & \num{0.0383} & \num{0.0015} & \num{0.0024} & \num{5.00} & \num{5.19}\\\hline
		\num{6.00} & \num{6.0965} & \num{0.0177} & \num{0.0003} & \num{0.0011} & \num{6.05} & \num{6.16}\\\hline
		\num{7.00} & \num{7.1509} & \num{0.0324} & \num{0.0010} & \num{0.0021} & \num{7.08} & \num{7.25}\\\hline
		\num{8.00} & \num{7.9356} & \num{0.0191} & \num{0.0004} & \num{0.0012} & \num{7.89} & \num{7.98}\\\hline
	\end{tabular}
	\caption{Stochastische Eigenschaften der Entfernungsmessungen.}
	\label{tab:entfernungsmessung_stochastik}
\end{table}

\begin{figure}[h!]
  \centering
  \includegraphics[width=0.5\linewidth]{entfernungsmessung_punktwolke}
	\caption{Verteilung der Messpunkte der ungeraden Entfernungsmessungen.}
	\label{fig:entfernungsmessung_punktwolke}
\end{figure}

In der \figurename~\ref{fig:entfernungsmessung_los_16440} wurden die ungeraden Entfernungsmessungen als Histogramm dargestellt. Gut zu erkennen ist die Normalverteilung der Messwerte um den Mittelwert.

\begin{figure}[h!]
	\centering
	\begin{subfigure}[b]{0.45\textwidth}
		\centering
		\includegraphics[width=\textwidth]{entfernungsmessung_los_1_16440}
		\caption{1 Meter}
		\label{fig:entfernungsmessung_los_1_16440}
	\end{subfigure}
	\hfill
	\begin{subfigure}[b]{0.45\textwidth}
		\centering
		\includegraphics[width=\textwidth]{entfernungsmessung_los_3_16440}
		\caption{3 Meter}
		\label{fig:entfernungsmessung_los_3_16440}
	\end{subfigure}
	\bigskip
	\begin{subfigure}[b]{0.45\textwidth}
		\centering
		\includegraphics[width=\textwidth]{entfernungsmessung_los_5_16440}
		\caption{5 Meter}
		\label{fig:entfernungsmessung_los_5_16440}
	\end{subfigure}
	\hfil
	\begin{subfigure}[b]{0.45\textwidth}
		\centering
		\includegraphics[width=\textwidth]{entfernungsmessung_los_7_16440}
		\caption{7 Meter}
		\label{fig:entfernungsmessung_los_7_16440}
	\end{subfigure}
	\caption{Histogramm und Wahrscheinlichkeitsdichtefunktion der ungeraden Entfernungsmessungen.}
	\label{fig:entfernungsmessung_los_16440}
\end{figure}


\begin{comment}
--------------------------------------------------------------------------------
- Diagramme
	- \cite{kurth2003experimental}
		- Fig. 5: (1) The ground truth path with tags indicated by circles. The numbers indicate how many range measurements were received from each tag over the duration of Test 1. (2) The path estimate from dead reckoning alone. (3) The path estimate from localization using a Kalman Filter. The Filter fuses data from odometry and a gyro with absolute measurements from RF tags to produce this path estimate. Numerical results are given in Table 1. (X: position in x(m), Y: position in y(m), Ground truth path with tag locations, Dead reckoning path, Kalman filter localization path)


- Versuchsdurchführung (Kalibierung)
	+ Warten bis alle Beacons registiert sind (10 Sekunden)
		+ rostopic echo "/beacon/sensed_data[0]/id"
	+ Anzahl der Messungen festlegen
		+ Wie dauert eine volle Messung
		+ 250=50s, 500=>100s, 1000=>200s
	+ ROS
		+ roscore
		+ beacon_publisher
		- beacon_writer	
			- filename
			- append
			- check measurement id
			- count per beacon
			- debug print
	+ Beacons mit einem Null Antennenverzögerung programmieren
		- Projekte erstellen mit Tag und Anchor
		- Programmierung von Linux aus
	+ Aufzeichen und Auswerten
		- Matlab Script erstellen
	+ Beacons mit der persönlichen Antennenverzögerung programmieren
		- Genauigkeit verifizieren
	- Ersatz Beacon erstellen
	- Gleichgewichtige Verkabelung für die Datenübertragung

- Versuchsdurchführung (LOS/NLOS)
	+ Nicht RF-Transparentes Material bestimmen
		+ Alumniumplatte (stark Dämpfung)
		+ Holzplatten (schwache Dämpfung)
		+ Behälter mit Wasser (mittlere Dämpfung)
	+ Messanzahl festlegen
		+ Dauer
			+ 1-1 Beacon => 8Hz
			+ 1-2 Beacon => 13Hz
			+ 1-3 Beacon => 17Hz
			+ 1-4 Beacon => 20Hz
		+ 250 => 31s, 500 => 62s, 1000 => 125s
	- Messabstände festlegen
		- 1:1:15 Meter? => 15 Messungen
		- 1:0.5:15 Meter? => 30 Messungen
		- 1:0.1:8 Meter? => 80 Messungen

- Aufzeichnungsrichtlinien (RO-SLAM)
	+ Robotino Odom zurücksetzen
		+ rosservice call /reset_odometry 0 0 0
	+ Warten bis alle Beacons registiert sind (10 Sekunden)
		+ rostopic echo "/beacon/sensed_data[0]/id"
	- Beacon Platzierung
		+ Beacons auf keinen Fall symmetrisch plazieren.
		+ Beacon-Position muss im Nachhinein bestimmt werden können.
		- Beacons erhöht positionieren, damit sie vom Laser erkannt werden.
		- Karte erstellen mit Beacon-Positionen
	- Robotino Platzierung
		+ Horizontale wird an dem Türrahmen ausgerichtet
		+ Rückkehren zum Startpunkt, sinnvoll wegen Drift.
		- Startpunkt festlegen auf der Karte
		- Startpunkt mit Tape abkleben
	- Skizze des Raums anfertigen mit Maßen
	- Platzierung von Kisten um Features für den LaserScan zu haben
	- Robotino Geschwindigkeit anpassen
		- Custom joystick teleop
			- [todo]
			- http://wiki.ros.org/teleop_twist_joy
			- http://yardbot.ca/2014/10/writing-custom-joystick-teleop-node-ros/
	- URDF-Modell anpassen
		- [todo] Position des Beacons anpassen
		
		
- Versuchsbeschreibung:
	- Warum wurden die uwbm da platziert wo sie jetzt stehen?
	
- Bauhaus
	- Rohr mit einer Höhe von x und einem Durchmesser 4-5 cm mit Kappe
	- Schrauben
	- Malerband?

\end{comment}
\section{RO-SLAM [todo]}


\begin{comment}
--------------------------------------------------------------------------------
\end{comment}
\subsection{Trajektorie [todo]}


\begin{comment}
--------------------------------------------------------------------------------
\end{comment}
\subsection{Trajektorie [todo]}


\begin{comment}
--------------------------------------------------------------------------------
\end{comment}
\subsection{Vergleich von MC und SOG [todo]}


	%%%%%%%%%%%%%%%%%%%%%%%%%%%%%%%%%%%%%%%%%%%%%%%%%%%%%%%%%%%%%%%%%%%%%%%%%%%%%%%%
%
%	- [Wikipedia, Wireless USB]
%		- Eine aktuelle Recherche (3. Januar 2017) in den einschlägigen deutschen Bestellportalen (Ebay, Amazon, Conrad, Euronics, …) ergab, dass Geräte mit CWUSB-Unterstützung dort derzeit in Deutschland nicht bestellbar sind.
%		- Bei Amazon ließen sich in den Kundenrezensionen Spuren finden, dass im Jahr 2010 entsprechende Geräte auch vertrieben wurden.
%		- Zwei Entwicklungen machen es den Geräten schwer, sich am Markt zu behaupten:
%			- Einerseits wurde mit USB 3.0 die Datendurchsatzrate deutlich angehoben, was die Anforderungen an den Wireless-USB-Standard verschärft.
%			- Andererseits hat die Marktentwicklung bei den Smartphones die Verbreitung des Bluetooth-Standards stark ausgebaut.
%			- Während das Bluetooth-Konsortium seinen Standard laufend weiterentwickelt (zuletzt 2016 mit Version 5), datiert die letzte Version des USBCV-Tools für den Test und die Entwicklung von Wireless USB auf den 17. Juli 2009. Vor diesem Hintergrund erscheint es derzeit fraglich, ob CWUSB noch einmal aus der Versenkung auftauchen wird. 
%	
%	- Vielleicht sollte man sich diese Einschätzung für das Fazit aufgewahren? Komnsumer Markt nein, Spezial Markt ja.
%	
%
%%%%%%%%%%
\chapter{Zusammenfassung und Ausblick}

Im Grundlagen-Kapitel wurden zu erste der Unterschied zwischen der Entfernungsmessung mittels Triangulation und der Trilateration beschrieben. Die Triangulation bestimmt die Entfernung durch das Messen der Winkel zwischen mehreren Referenzpunkten, während die Trilateration die Entfernungen anhand der Signallaufzeit bestimmt. Die Trilateration wird von den \glsuseri{uwbm} verwendet um Nachrichten auszutauschen. Durch den Nachrichtenaustausch ist es auch möglich die Entfernung zwischen zwei \glspl{uwbm} zu bestimmen. Dazu wird der Nachrichtenversand zu einem zukünftigen Zeitpunkt geplant, um den Zeitstempel des Sendevorgangs in die Nachricht einzubetten. Das empfangende \gls{uwbm} ist nun im Besitz aller Informationen um die Entfernung zu errechnen. Dies ist unter dem Namen \gls{sstwr}-Verfahren bekannt. Eine Verbesserung stellt das \gls{dstwr}-Verfahren dar, das für die Entfernungsmessung verwendet wird, da es den Fehler der lokalen Zeitgeber minimiert.

Im Geometrie-Abschnitt wurden die mathematischen Gleichungen für das Konstruieren und Berechnen der relevanten Längen des gleichseitigen Dreiecks und eines regelmäßigen Fünfecks beschrieben.

Die Wahrscheinlichkeitstheorie legt den Grundstein um die Funktionsweise der verschieden \gls{slam}-Varianten zu verstehen. Dabei wurden die Konzepte der Zufallsvariablen, der einfachen und mehrdimensionalen Normalverteilung und deren Gesetzmäßigkeiten wie die bedingte Wahrscheinlichkeit, die Abhängigkeiten zwischen Zufallsvariablen und der Satz von Bayes vorgestellt.

Mit einem Zustandschätzer ist es möglich, Abschätzungen über den zukünftigen Zustand eines Systems zu erstellen. Der Zustand beschreibt dabei alle Aspekte eines Roboters und seiner Umwelt die einen Einfluss auf die Zukunft haben können. Die Abschätzungen werden dabei durch die Steuerbefehle, die der Roboter an die Aktorik sendet, und die Wahrnehmungen der Umwelt gesteuert. Das interne Wissen des Roboters über den Zustand seiner Umwelt wird dabei als Belief bezeichnet. Die Basis aller Zustandschätzer bildet dabei der rekursive Bayes Filter, der jeden Zustand in zwei Schritten schätzt. Im Prognose-Schritt wird der nächste Zustand mit den Steuerbefehlen vorhergesagt und dann im Korrektur-Schritt mit den Wahrnehmungen korrigiert. Bei dem Bayes Filter handelt es sich um eine sehr abstrakten Algorithmus, der durch den Kalman Filter konkret umgesetzt wird. Der Kalman Filter nutzt dabei eine mehrdimensionale Normalverteilung um den Belief zu repräsentieren. Dadurch entstehen bei der Nutzung von nicht linearen Funktionen, die bei Rotationsbewegungen auftreten, aber auch Probleme. Diese werden durch den \gls{ekf} mittels einer Linearisierung der nicht linearen Funktion gelöst.

Bedingt durch die Verwendung einer Normalverteilung, können sowohl der Kalman Filter als auch der \gls{ekf} keine multimodalen Verteilungen darstellen. Diese Einschränkungen können durch den \gls{pf} umgangen werden. Dieser nutzt eine Menge von Partikeln um den Belief darzustellen. Dadurch ist es möglich eine beliebige Verteilung abzubilden. Jedes Partikel wird mit einem Gewicht versehen, um die Relevanz zu beschreiben. Im Resampling-Schritt werden die Partikel aussortiert, bei denen die Gewichtung unter einen festgelegten Grenzwert fällt, und durch neue Partikel ersetzt.

Wenn es notwendig wird, das nicht nur der Zustand des Roboters geschätzt, sondern auch gleichzeitig eine Karte der Umwelt erstellt wird, spricht man von einem \gls{slam}-Verfahren. Der \gls{ekf}-\gls{slam} gehört dabei zu den Standardverfahren der Robotik. Hierbei wird der Zustand des Roboters, als auch der der Umwelt mittels eines kombinierten Zustandsvektors modelliert. Das \gls{slam}-Verfahren kann auch mithilfe eines \gls{pf} gelöst werden, dazu muss der Zustandsvektor aufgespalten werden da \gls{pf} in großen Zustandsräumen nicht praktikabel sind. Der \gls{pf} stellt somit durch seine Partikel eine Hypothese des Pfades dar und jedem Partikel wird dann eine Menge von Landmarken zugewiesen.

Das Grundlagen-Kapitel schließt mit einer kurzen Betrachtung der \gls{ros}-Begrifflichkeiten ab. Hierzu gehört der \gls{ros}-Master, der die Verbindung zwischen verschieden Verarbeitungseinheiten, auch als Knoten bezeichnet, bereitstellt. Damit eine Kommunikation zwischen den Knoten stattfinden kann, werden Nachrichten über Datenbusse übertragen.

Die ersten beiden Veröffentlichungen im Kapitel Stand der Forschung und Technik werden die Positionen der \glspl{beacon} zuerst solange beobachtet, bis die Unsicherheit so gering ist, dass diese problemlos mit einem \gls{ekf}-\gls{slam} verarbeitet werden können. In der ersten Veröffentlichung werden die ungefähren \gls{beacon}-Positionen durch die Kombination von \gls{propgrid} angenähert. Anders geht die zweite Veröffentlichung vor, hier werden die beobachteten Entfernungen in einem Gitter eingetragen. Sobald sich ein Gipfel gebildet hat, wird die Positionsschätzung an den \gls{ekf}-\gls{slam} übergeben.

Die nächsten zwei Veröffentlichungen nutzen den \gls{pf}-\gls{slam}. Beim ersten wird für jedes Partikel ein Hilfspartikel Filter eingesetzt, um die radiale Verteilung zu modellieren. Der Zweite verwendet anstatt einem Hilfspartikel Filter eine Menge von Normalverteilungen die ebenfalls radial angeordnet und mit Gewichten versehen sind. Sobald beide Verfahren gegen eine \gls{beacon}-Positionen konvergiert sind, wird der Hilfspartikel Filter und die Menge von Normalverteilungen durch einen \gls{ekf} ersetzt.

Einen vollkommen anderen Weg beschreibt die letzte Veröffentlichung, hier werden die \gls{beacon}-Positionen nicht in Kartesische Koordinaten, sondern in Polarkoordinaten beschrieben.

Das Ultra-Wideband-Kapitel beginnt mit der Beschreibung der Unterschiede zwischen der Übertragung von Informationen durch das Aufmodulieren auf eine sinusförmige Trägerfrequenz und der Übertragung von Informationen im Basisband durch das Erzeugen von kurzen Impulsen im Nanosekundenbereich. Die Ultra-Wideband Technologie verwendet das letzte Verfahren und kann dank der hohen Bandbreite auch entsprechend hohe Datenmengen übertragen.

Danach geht es an die Erstellung der Hardware, sprich der \glspl{uwbm}. Zu den Hauptanforderungen gehört die gemeinsame Hardwareplattform, eine separate Stromversorgung und eine optionale Kommunikationsschnittstelle. Das Herzstück der \glspl{uwbm} ist dabei der \gls{uwbt} von \textit{DecaWave}. Dieser \gls{ic} sorgt für die komplette Verarbeitung und Auswertung der \gls{uwb}-Signal. Gesteuert wird dieser über die \gls{spi}-Schnittstelle durch einen Arduino kompatiblen Mikrocontroller, dem Pro Trinket. Da der Mikrocontroller über keine eigene Kommunikationsschnittstelle zur Verarbeitungseinheit verfügt, wird diese über eine separate Kommunikationsschnittstelle gelöst. Im Vorfeld wurden zwei Prototypen der \gls{uwbm} aufgebaut. Zum einen um die Beschaltung der elektrischen Komponenten zu testen und zum anderen um den Nachrichtenaustausch mit gleichzeitiger Entfernungsmessung auszuprobieren. Nach der erfolgreichen Prototypenphase wurde ein Platinendesign erstellt und durch einen entsprechenden Dienstleister gefertigt.

Als Software für die Steuerung des \gls{uwbt} durch den Mikrocontroller wurde eine GitHub-Projekt verwendet. Dieses stellte die Basis für die Kommunikation mit dem \gls{uwbt} bereit. Zusätzlich waren die Kommunikationsprotokolle für die Entfernungsmessung bereits implementiert. Lediglich die Kommunikationsschnittstelle zwischen dem Mikrocontroller und der Verarbeitungseinheit, und die Integration der Entfernungsmessungsdaten in das \gls{ros}-System mussten entwickelt werden.

Nach dem die \glspl{uwbm} erstellt und die Kommunikationsschnittstelle bereitstand, musste die Antennenverzögerung durch eine Kalibierung der \glspl{uwbm} bestimmt werden. Hierfür wurde zum einen das herstellerspezifische Verfahren implementiert, das auf Basis eines genetischen Algorithmus die Antennenverzögerung bestimmte. Durch die Erkenntnis, dass das Verfahren nur unzureichende Ergebnisse lieferte, wurde ein weiteres Verfahren auf der Basis eines linearen Gleichungssystems entwickelt.







% Umsetzung des RO-SLAM in ROS
%- Roboterplattform
%	- Robotino 2, 2D-Laser-Entfernungsmesser, holonomer Antrieb mit Inkrementalgeber, Verarbeitungseinheit
%	- Softwarearchitektur
%		- Transformatinosbaum mit Odometrie + Entfernungsmessungen
%	- ROS-Hauptmodule
%		- Robotino-Steuerung
%		- Koordindatentransformationen
%		- Teleoperation
%		- UWB-Entfernungsmessung
%	- ROS-Hilfsmodule
%		- 2D-Laser-Entfernungsmesser
%		- Belegtheitskarten
%		- Trajektorie
%		- Laser-Odometrie
%	- MRPT

% Evaluation
%- Batterielaufzeit
%- Kalibierung
%	- LGS
%	- DecaWave
%	- Normalverteilung der Messwerte
%	- Fünfeck
%- Entfernungsmessung
%	- LOS/NLOS
%	- 1-9m in 0.5 schritten
%	- Blech/Wasser
%- Trajektorie
%	- Testen der verschieden Odometriequellen
%	- Inkrementalgeber + laserbasierte
%- Positionsschätzung
%	- RF2O
%	- virtuelle + reale uwbm
%- Positionsschätzung der UWB-Module
%	- virtuelle werden auf ca. 0.15-0.97m geschätzt
%	- reale werden von ca 0.49-1,03m geschätzt
%- Konvergenz der WDF


%%%%%%%%%%%%%%%%%%%%%%%%%%%%%%%%%%%%%%%%%%%%%%%%%%%%%%%%%%%%%%%%%%%%%%%%%%%%%%%%
%
%	- UWB-Module
%		- Stärker Prozessor um den DWM mit maximaler SPI-Geschwindigkeit anzusteuern (20 Mhz)
%		- Identifier über DIP Schalter einstellbar
%		- Größerer Speicher um mehr Anchor/Tags verwalten zu können.
%		- Stromsparfunktionen
%		- vergleich mit dem kommerziellen Produkt
%	- RO-SLAM
%		- 
%
%%%%%%%%%%
\section{Ausblick}


%%%%%%%%%%%%%%%%%%%%%%%%%%%%%%%%%%%%%%%%%%%%%%%%%%%%%%%%%%%%%%%%%%%%%%%%%%%%%%%%
%
% Wurden alle Fragen aus dem Expose geklärt/beantwortet?
%
%%%%%%%%%%
\section{Fazit}


	\cleardoublepage


%
% Literatur
%
% 	- http://linorg.usp.br/CTAN/macros/latex/contrib/biblatex/doc/biblatex.pdf
% 		- bibnumbered, bibintoc
%	- https://de.sharelatex.com/learn/Bibliography_management_with_biblatex
%
	%\nocite{mcelroy2014comparison}
	%\nocite{herranz2014comparison}
	%\nocite{gonzalez2009mobile}
	%\nocite{durrant2006simultaneous}
	%\nocite{thrun2005probabilistic}
	%\nocite{schroeder2005low}
	%\nocite{smith2004tracking}
	\printbibliography[heading=bibintoc,filter=papersonly,title={Literaturverzeichnis}]
	\printbibliography[heading=subbibintoc,type=online,title={Internetquellen}]
	\printbibliography[heading=subbibintoc,type=manual,title={Handbücher}]
	\cleardoublepage

	\begin{appendices}

\chapter{Blindtext Kapitel}
\blindtext

\section{Blindtext Abschnitt}
\Blindtext

\end{appendices}

	\backmatter

\end{document}