%
% Forschungsstand:
%
% - Welche wissenschaftlichen Erkenntnisse liegen zu dem Thema bereits vor?
% - Grundsätzlich gibt es zwei Möglichkeiten, einen Forschungsstand zu schreiben: Entweder ordnen Sie Ihren Literaturüberblick nach Themenkomplexen oder Sie geben einen rein chronologischen Überblick über die wichtigsten Publikationen.
% - Auf keinen Fall sollten Sie den Forschungsstand zu voll packen. Es geht nicht darum, dem Leser zu zeigen, was Sie alles studiert haben (wie fleißig Sie waren), sondern um einen kompakten Überblick über die wichtigste Literatur.
% - Wichtig: Listen Sie die Literatur nicht nur auf, sondern erklären Sie, welchen Beitrag die jeweilige Publikation zum Erkenntnisgewinn geleistet hat. Also, zum Beispiel: Was hat der Autor als Erster erkannt oder hinterfragt? Es muss ja einen Grund geben, weshalb Sie die betreffende Publikation unter die Meilensteine reihen – und den sollten Sie dem Leser deutlich machen.
%
% Ferrein:
% - 4-5 Seiten in der Bachelorarbeit (deutlich umfangreicher als im Exposé)
% - Was unterscheidet meinen Ansatz von den anderen verhandenen Ansätzen?
%
\begin{comment}
------------------------------------------------------------------------------------------
\end{comment}
\chapter{Stand der Forschung und Technik}


\begin{comment}
------------------------------------------------------------------------------------------
\end{comment}
\section{\citefield{kantor2002preliminary}{title}(189)}


\begin{comment}
------------------------------------------------------------------------------------------
\end{comment}
\section{\citefield{kurth2003experimental}{title}(82)}


\begin{comment}
------------------------------------------------------------------------------------------
- Wie funktioniert die Exploration Strategy?
	- Die Ungünstigeste strategy ist das geradeausfahren mit einem beacon links und rechts.
	- Das Gradientenfeld der abstanddifferenz zwischen den beiden beacons führt einen auf den optimalen weg um die Abstandsdifferenz zu maximieren. (Aktive Exploration)
\end{comment}
\section{\citefield{olson2004robust}{title}(264)}

In \citetitle{olson2004robust}\cite{olson2004robust} werden autonome Unterwasserfahrzeuge (engl. \gls{auv}) verwendet um den Ozeanen der Welt die letzten Geheimnisse zu entlocken. Hierfür ist eine genaue Lokalisierung der \gls{auv} notwendig. Dies wird über die Messung der Signallaufzeit zwischen dem \gls{auv} und mehreren stationären akustischen Transponder--\Gls{beacon} erreicht. Vor der eigentlichen Messung müssen jedoch die \Glspl{beacon} im Messbereich verteilt werden, deren Position genau bestimmt werden und zusätzlich dafür gesorgt werden, dass die \Glspl{beacon} ihre Position nicht ändern. Dadurch handelt es sich um ein sehr kostspieliges Prozedere und soll in dem vorgestellt Verfahren dahingehend optimierte werden, dass die \Gls{beacon}--Position vorher nicht bekannt sein muss. Dadurch ist eine autonome Verteilung der \Glspl{beacon} möglich und zusätzlich kann auch detektiert werden ob ein \Gls{beacon} seine Position verändert hat.

% TODO: Spectral Graph Partitioning
Bevor jedoch die Laufzeitmessungen verwendet werden können, müssen diese um \Gls{outlier} (dt. Ausreißer) bereinigt werden. Diese entstehen z.B. durch unterschiedliche Ausbreitungsgeschwindigkeiten der Schallwellen im Wasser, durch Reflektionen am Meeresgrund (engl. \Gls{multipath}) oder durch Interferenzen der Nutzlast (engl. \Gls{payload}) mit den Sensoren. Die Bereinigung der Messdaten erfolgt hierbei über das \textit{Spectral Graph Partitioning} Verfahren. Hierbei wird jede Messung als Kreis dargestellt. Überschneiden sich zwei Kreise gelten diese beiden Messungen als konsistent. Jede der Messungen wird in einem Graphen zu einem Vertex und jede konsistente Messung zu einer Kante. Bei dem fertigen Graphen ist nun zu beobachten, dass \Gls{outlier} weniger stark untereinander verbunden sind als \Gls{inlier}. Diese durchschnittliche Konnektivität der \Gls{inlier} wir als Metrik für den Partitionierungsalogorithmus verwendet um die Messdaten zu filtern.

Nach dem die Messwerte pro \Gls{beacon} gefiltert worden sind, werden zwei Messwerte von zwei verschiedenen \Glspl{beacon} wieder mit einandern überschnitten. Hieraus entstehen zwei mögliche Positionen des \gls{auv}. In einem Grid erhöhen diese beiden Positionen dann den Ihnen zugeordneten Zähler. Dieser Vorgang wird für weiter Messwerte wiederholt. Durch das daraus resultierende Voting--System entstehen zwei Gipfel (engl. Peak) die die mögliche Position des \gls{auv} approximieren. Ist der Höhenunterschied zwischen den Gipfeln ausreichend groß, wird die \gls{auv}--Position mit dem höchsten Gipfel an einen EKF--SLAM übergeben.


\begin{comment}
------------------------------------------------------------------------------------------
\end{comment}
\section{\citefield{smith2004tracking}{title}(547)}



\begin{comment}
------------------------------------------------------------------------------------------
\end{comment}
\section{Übersicht der Meilensteine [Remove in final version]}

\begin{description}

\item[2001]
\begin{itemize}
\item !A solution to the simultaneous localization and map building (SLAM) problem (Zitiert von: 2803)
\item +Auxiliary variable based particle filters (115)
\item +Factor graphs and the sum-product algorithm (5869)
\item +Rao-Blackwellised particle filtering for dynamic Bayesian networks (1264)
\end{itemize}

\item[2002]
\begin{itemize}
\item +FastSLAM: A factored solution to the simultaneous localization and mapping problem (2469)
\item !Preliminary results in range-only localization and mapping (187)
\item Indoor geolocation science and technology (985)
\end{itemize}

\item[2003]
\begin{itemize}
\item !Experimental results in range-only localization with radio (82)
\item !Pure range-only sub-sea SLAM (174)
\item +Recent results in extensions to simultaneous localization and mapping (18)
\end{itemize}

\item[2004]
\begin{itemize}
\item !Tracking moving devices with the cricket location system (547)
\item !A probabilistic approach to inference with limited information in sensor networks (42)
\item +An introduction to factor graphs (672)
\item ?An efficient multiple hypothesis filter for bearing-only SLAM (114)
\end{itemize}

\item[2006]
\begin{itemize}
\item !Simultaneous localization and mapping: part I (2982)
\item !Simultaneous localization and mapping (SLAM): Part II (1479)
\item Further results with localization and mapping using range from radio (69)
\item !Range-only slam for robots operating cooperatively with sensor networks (146)
\item !Range-only slam with interpolated range data (28)
\end{itemize}

\item[2007]
\begin{itemize}
\item !Application of UWB and GPS technologies for vehicle localization in combined indoor-outdoor environments (52)
\item +Rao-Blackwellized particle filter for multiple target tracking (264)
\end{itemize}

\item[2008]
\begin{itemize}
\item !Efficient probabilistic range-only SLAM (Zitiert von: 70)
\item !A pure probabilistic approach to range-only SLAM (Zitiert von: 36)
\end{itemize}

\item[2009]
\begin{itemize}
\item !Navigating with ranging radios: Five data sets with ground truth (18)
\item !Mobile robot localization based on ultra-wide-band ranging: A particle filter approach (90)
\item !A robust method of localization and mapping using only range (45)
\item !Range-only SLAM with a mobile robot and a wireless sensor networks (109)
\end{itemize}

\item[2010]
\begin{itemize}
\item !Geolocation with range: Robustness, efficiency and scalability (11)
\item !Studying of WiFi range-only sensor and its application to localization and mapping systems (3)
\item !tinySLAM: A SLAM algorithm in less than 200 lines C-language program (62)
\end{itemize}

\item[2011]
\begin{itemize}
\item Ultra wide-band localization and SLAM: A comparative study for mobile robot navigation (17)
\item !A new state vector for range-only SLAM (10)
\end{itemize}

\item[2013]
\begin{itemize}
\item A Spectral Learning Approach to Range-Only {SLAM} (13)
\end{itemize}

\item[2014]
\begin{itemize}
\item !A comparison of slam algorithms with range only sensors (2)
\item !Efficient robot-sensor network distributed seif range-only slam (16)
\end{itemize}

\item[2015]
\begin{itemize}
\item ?A robot self-localization system using one-way ultra-wideband communication (18)
\item ?Fusing ultra-wideband range measurements with accelerometers and rate gyroscopes for quadrocopter state estimation (30)
\end{itemize}

\item[2016]
\begin{itemize}
\item !Indoor robot positioning using an enhanced trilateration algorithm (5)
\end{itemize}

\item[2017]
\begin{itemize}
\item ?Ultra-Wideband Aided Fast Localization and Mapping System (1)
\item !A system for indoor positioning using ultra-wideband technology (0)
\item ?Range-only SLAM schemes exploiting robot-sensor network cooperation (0)
\end{itemize}

%\item[]
%\begin{itemize}
%\item 
%\end{itemize}

\end{description}


\begin{comment}
------------------------------------------------------------------------------------------
\end{comment}
\section{Informationen aus dem Expose [Remove in final version]}

Einen guten Überblick über die Eigenschaften der Drahtlosen-Protokolle (engl. Wireless Protocols) Bluetooth, UWB, ZigBee und WiFi liefert die Arbeit \cite{lee2007comparative} von \citeauthor{lee2007comparative}.

In \cite{smith1987closed} wird das grundlegende Prinzip erklärt um aus mehreren bekannten Sensoren die Position eines beweglichen Empfängers zu berechnen.

Der theoretische Hintergrund des SLAM--Verfahrens wird in \cite{dissanayake2001solution} vorgestellt. Zusätzlich wird bewiesen das die Unsicherheit bei der Kartenerstellung und Lokalisierung eine untere Schranke erreicht.

\citeauthor{kantor2002preliminary} stellen in Ihrer Arbeit \cite{kantor2002preliminary} ein Lokalisierungsverfahren vor, welches die Roboterposition anhand von Entfernungsmessungen zu vorher bekannten Landmarken bestimmen kann. Im letzten Abschnitt wird SLAM--Verfahren vorgestellt, welches über einen Kalman--Filter die Unsicherheit der Landmarkenposition modellieren kann.

Die Autoren \citeauthor{blanco2008pure} gehen in ihren Arbeiten \cite{blanco2008pure, blanco2008efficient} einen Schritt weiter und bestimmen die unbekannte Roboterposition sowie die unbekannten Landmarkenpositionen. Hierzu nutzen Sie im ersten Schritt einen Partikelfilter (engl. Particle Filter) bis die Schätzung eine ausreichende Genauigkeit erreicht hat um dann im zweiten Schritt über einen Kalman--Filter ein Positionsverfolgung (engl. Position Tracking) durchzuführen.

Die Arbeit \cite{ledergerber2015robot} von \citeauthor{ledergerber2015robot} gehen auf die Roboterlokalisierung unter Verwendung einer One-Way Ultra-Wideband Kommunikation ein. Dieses hat den Vorteil, das mit sehr wenigen Landmarken eine große Anzahl von Roboter lokalisiert werden kann.

- The Cartesian EKF described above operates in the Cartesian space, we formulate our problem in polar coordinates.
- The use of this parameterization derives motivation from the polar coordinate system, where annuli, crescents and other ringlike shapes can be easily modeled. This parameterization is called Relative Over Parameterized (ROP) because it over parameterizes the state relative to an origin.

- EKF -> Polar EKF -> Multi-Hypothesis Filter
- Partikel Filter